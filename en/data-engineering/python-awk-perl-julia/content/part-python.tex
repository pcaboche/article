% Copyright 2022 Pierre S. Caboche. All rights reserved.

\renewcommand{\currentPart}{Python}

\newpage
\part{Python} \label{python}

We will use the \python\ script as a reference. We will study its implementation details later, when we compare the Python implementation with the \awk\ and \perl\ scripts\dots \\


Below is our original implementation, in \python:

\begin{figure}[h]
	\caption{bucket.py}
	\lstinputlisting[language=python]{"files/bucket.py"}
\end{figure}  


\newpage


Make sure this script is executable by performing a:
\begin{lstlisting}[language=sh]
chmod u+x bucket.py
\end{lstlisting}



\section*{A few things to notice\dots}

Below are a few things I would like to point out regarding scripts in general, and this Python script in particular\dots


\subsection*{Specifying the interpreter}

Unlike Windows, Linux does not rely on a file extension to determine how to run a program. As long as a file is marked as ``executable", Linux will consider is as executable. \\

To determine which interpreter to use, the program loader looks at the first line of a script. \\

In a Linux script, the first characters of the first line are \cmd{\#!}, also called \emph{``shebang"} (because it comprises a ha\emph{sh}tag and an exclamation mark, also known as \emph{``bang!"} Hence the name \emph{``shebang"}) \\

Once a \emph{``shebang"} is detected, the program loader reads the rest of the first line to determine which interpreter to use (and with which parameters). \\

Note that we need to specify the full path of the interpreter executable. Usually, this will start with \cmd{/usr/bin} \footnote{\cmd{/usr/bin} contains binaries available to all users in multi-user mode, as opposed to \cmd{/bin} which is contains essential binaries required at boot up and single-user mode}. \\

\bigskip



In this article, we will use the following at the beginning of our scripts\dots \\

For \cmd{python}:
\begin{lstlisting}[language=sh]
#!/usr/bin/python3
\end{lstlisting}

For \gawk:
\begin{lstlisting}[language=sh]
#!/usr/bin/gawk -f
\end{lstlisting}

For \cmd{perl}:
\begin{lstlisting}[language=sh]
#!/usr/bin/perl
\end{lstlisting}

For \cmd{julia}:
\begin{lstlisting}[language=sh]
#!/usr/bin/julia
\end{lstlisting}


\newpage
\subsection*{Incrementing an integer}

\note{
	\python\ doesn't have the \cmd{++} operator for incrementation. \\
	
	In other languages, the misuse of the \cmd{++} operator might lead to unforeseen consequences, which \emph{may} be the reason why \cmd{++} has been left out of \python.
}


If you try using the \cmd{++} operator in \python\ (like you might be doing in other programming languages), then you'll be met with a syntax error, which can be confusing at first. \\

In any case, in \python, to increment an integer, you would need to do a \cmd{+= 1}:

\begin{lstlisting}[language=python]
# Python
file_num += 1
\end{lstlisting}

\medskip


In the other languages ( \awk\ and \perl\ ), we will use the \cmd{++} unary operator to increment integers. For example in \awk:

\begin{lstlisting}[language=awk]
# AWK
file_num++ 
\end{lstlisting}



\subsection*{Compiling the regular expression}

Near the beginning of our script, we have the following instructions:

\begin{lstlisting}[language=python]
pattern=r'^-- Table structure for table `?(?P<TABLE>\w+)`?'
parser = re.compile(pattern, re.IGNORECASE)
\end{lstlisting}

Here, we ``compile" the regular expression once, store the compiled expression in a variable (\cmd{parser}), and then use it again and again ( \cmd{parser.match(line)} ).

This is to avoid compiling the regular expression every time we need it.

\subsection*{Reading from \stdin\ }

We read the content of \stdin\ line by line, with the following loop:
\begin{lstlisting}[language=python]
	for line in sys.stdin :
	# loop instructions...
\end{lstlisting}

In this code, \cmd{sys.stdin} is an \emph{iterator} over \stdin, and our loop follows this pattern:
\begin{lstlisting}[language=python]
for <variable> in <iterator> :
# loop instructions...
\end{lstlisting}


\newpage

\section*{Result} 


The result of execution is as follows (\emph{note: some of the output was edited to fit in the page}):

\begin{lstlisting}[language=sh]
[user@gen-8 test]$ time pv mysql_dump.sql | ./bucket.py
7.42GiB 0:14:55 [8.49MiB/s] [====================================================>] 100%  

real    14m55.335s
user    14m29.250s
sys     0m28.531s
\end{lstlisting}

As you can see, our input file was 7.42 GiB in size, and Python took nearly 15 minutes to process it at an abysmal 8 MiB/s... ( = 7420 / 15 / 60 ) \\

When running the script for the first time, \pv\ showed a very low 10 MiB/s, and my first reaction was: \emph{``did I do something wrong?"}.
As we'll discover later, this was not the case\dots \\

In the meantime (and while the \python\ script was still running\dots), I wanted to try another implementation. So I rewrote the script in \awk\dots
