
% Copyright 2022 Pierre S. Caboche. All rights reserved.

% Note: use XeLaTeX for rendering
% !TEX encoding = UTF-8


\documentclass[twoside]{article}

\setlength{\parindent}{0pt}

%% Packages and their configuration
%%%%%% Packages and configuration

\usepackage{geometry}
\geometry{
	a4paper,
	total={170mm,257mm},
	left=35mm,
	right=35mm,
	top=30mm,
	bottom=30mm,
}


\usepackage{fontspec}
\setmainfont{DejaVu Serif} 
\setsansfont{DejaVu Sans} 


%% Use Sans-Serif by default
\renewcommand{\familydefault}{\sfdefault}



%% bibliography
\usepackage[authoryear]{natbib}


\usepackage{pgfplots}


\usepackage{titleps}
\usepackage{xstring}
\newcommand{\currentPart}{}

\newcommand{\formatPartTitleDefault}{
	\textbf{\IfStrEq{\thepart}{}{}{Part \thepart\ ---  }\currentPart}
}

\newcommand{\formatPartTitle}{\formatPartTitleDefault}

\newcommand{\copyrightNotice}{\color{gray}{\emph{\textcopyright\ 2022 Pierre S. Caboche. All rights reserved.}}}


\newpagestyle{main}{
	\sethead
		% even
		[\formatPartTitle]
		[]
		[]
		% odd
		{}
		{}
		{\formatPartTitle}
	
	\setfoot
		% even
		[\textbf{\thepage\ \color{lightgray}{|}}]
		[]
		[\copyrightNotice]
		% odd
		{\copyrightNotice}
		{}
		{\textbf{\color{lightgray}{|} \color{black}{\thepage} }} 
	
	\headrule
	%\footrule
}
\pagestyle{main}

\widenhead[25pt][25pt]{25pt}{25pt}




\usepackage{listings}
\lstset{
	basicstyle=\fontsize{10}{10}\selectfont\ttfamily
	,frame=lines
	,tabsize=2
	,keywordstyle=\bfseries%\itshape,
	,commentstyle=\itshape\color{teal}
	,stringstyle=\color{magenta}
	,breaklines=true,
	,postbreak=\mbox{\textcolor{red}{$\hookrightarrow$}\space}
}



\usepackage{hyperref}
\usepackage[super]{nth}

\usepackage{marginnote}
\setlength{\marginparwidth}{70pt}


\widowpenalties 1 10000
\raggedbottom




%% Custom macros


\newcommand{\insertBlankPage}[1][This page intentionally left blank.]{
\newpage
\thispagestyle{empty} % remove header and footer for current page
\vspace*{\fill}
\begin{center}
#1
\end{center}
\vspace*{\fill}
\newpage
}


\newmdenv[
	middlelinewidth=2pt,
	backgroundcolor=yellow!10
]{note}



\newcommand{\idxCmd}[1]{\index{\texttt{\textbackslash#1}}}
\newcommand{\idxEnv}[1]{\index{\texttt{#1 \footnotesize{(environment)}}}}
\newcommand{\idxPkg}[1]{\index{\texttt{#1 \footnotesize{(package)}}}}


\newcommand{\quoteCmd}[1]{\texttt{\textbackslash#1}\idxCmd{#1}}
\newcommand{\quoteEnv}[1]{\texttt{#1}\idxEnv{#1}}
\newcommand{\quotePkg}[1]{\texttt{#1}\idxPkg{#1}}

%%% Terms with custom rendering

\newcommand{\LoremIpsum}{\emph{Lorem Ipsum}\index{Lorem Ipsum}}

%% \TeXstudio
\newcommand{\TeXstudio}{\TeX studio\index{TeXstudio}}
\newcommand{\WYSIWYG}{WYSIWYG\index{WYSIWYG}}
\newcommand{\WYSIWYM}{WYSIWYM\index{WYSIWYM}}

\newcommand{\Iomega}{Iomega\textregistered\index{Iomega\textregistered}}
\newcommand{\Zip}{Zip\texttrademark\index{Zip\texttrademark}}

%% Note the use of non-breaking spaces (~) and long dash (---)
\newcommand*{\longref}[1]{\ref{#1}~---~\nameref{#1}}


\newcommand{\TeXworks}{\emph{TeXworks}\index{TeXworks}}

\newcommand{\Microsoft}{\emph{Microsoft\textregistered}\index{Microsoft\textregistered}}
\newcommand{\Word}{\emph{Word\textregistered}\index{Word\textregistered}}


\usepackage{hologo}

\newcommand{\XeTeX}{\hologo{XeTeX}\index{XeTeX}}
\newcommand{\XeLaTeX}{\hologo{XeLaTeX}\index{XeLaTeX}}
\newcommand{\pdfTeX}{\hologo{pdfTeX}\index{pdfTeX}}
\newcommand{\pdfLaTeX}{\hologo{pdfLaTeX}\index{pdfLaTeX}}
\newcommand{\LuaTeX}{\hologo{LuaTeX}\index{LuaTeX}}
\newcommand{\LuaLaTeX}{\hologo{LuaLaTeX}\index{LuaLaTeX}}
\newcommand{\ConTeXt}{\hologo{ConTeXt}\index{ConTeXt}}
\newcommand{\MiKTeX}{\hologo{MiKTeX}\index{MiKTeX}}



% Opening

\title{Python vs. AWK, Perl, Julia: \\ Processing large data files}
\author{Pierre S. Caboche}

%% Custom format to indicate revision date (if needed)
\date{
	July 2, 2022
	% \\	\medskip \footnotesize (Revised: \today)
}


\begin{document}
	
	\maketitle
	
	\begin{abstract}
		In this article we will learn how to perform a very useful task (processing a large text file, splitting data into ``buckets") using three different scripting languages: Python, AWK, and Perl.
		This will allow use to compare those four languages in terms of features, syntax, and performance. \\
		
		This will also give us the opportunity to learn more about these languages. Our examples and explanations should help you easily get started, even if you have no prior experience with Python, AWK, Perl, or Julia.
	\end{abstract}


\bigskip
\bigskip
\bigskip

%%% Intro

% Copyright 2022 Pierre S. Caboche. All rights reserved.

\renewcommand{\currentPart}{Preamble}

\section*{Background}
I started to use \LaTeX\ to write documents containing a lot of Japanese text and \furigana\footnote{one of my hobbies is to study the lyrics of the Japanese songs I like, then try to sing them at the \emph{karaoke}. \LaTeX\ allows me to quickly add \furigana\ to the lyrics, or any other Japanese text}. From my experience (and by using some of the techniques described in this article), adding \furigana\ was considerably faster to do in \LaTeX\ than in either \Word\ or \LibreOffice, as we'll discover towards the end of this  article\dots \\


However, when I switched to \Linux, I discovered that my \LaTeX\ documents didn't render at all. \\

When I tried to look for a solution online, I found a lot of documents whose advice were either:
\begin{itemize}
	\item \emph{outdated}, as they relied on the obsolete packages % \texttt{CJKutf8} package under \pdfLaTeX
	\item \emph{not portable}: they were written with \Windows\ in mind, and recommended the use of fonts that are not readily available on other systems (e.g. \Meiryo)
\end{itemize}

I eventually found a solution to those issues, I decided to share my findings in this article.




\section*{Goal}

Our goal in this article is to learn how to perform the following:

\begin{itemize}
	\item in \LaTeX, display texts in Chinese, Japanese, etc. \\
	\phantom{MM}\dots without relying on proprietary fonts, which might not be available on \Linux
	\item add \rruby\ characters, especially Japanese \furigana\ (e.g. \kabocha) and Chinese \ppinyin\ (e.g. \xpinyin*{南瓜})
\end{itemize}



\section*{Methodology}

To achieve our goals, we will do the following:

\begin{itemize}
	\item install the \emph{\Noto\ Fonts} for the relevant languages
	\item install \LaTeX
	\item render documents containing texts in Chinese, Japanese, Korean, etc.\\
		\phantom{MM}\dots in a way that works on Windows, Linux, and other systems
	\item add \furigana\ to text in Japanese with the \texttt{ruby} package, as well as a \emph{custom macro}
	\item add \ppinyin\ to text in Mandarin Chinese with the \xxpinyin package
	\item perform the same tasks in \LibreOffice\ (mini-guide included), and compare it with our solution in \LaTeX
\end{itemize}

The rest of this article goes into more details about \emph{what} those tools are, \emph{how} to use them, and \emph{why}.



\newpage

\section*{What are \emph{ruby} and \emph{furigana}?}

\emph{Ruby} characters are annotations usually placed on top of\footnote{or to the right, if the text is displayed vertically} Chinese, Japanese, or Korean characters\footnote{\rruby\ characters can technically be used in other languages too}, which are usually used to show the pronunciation of such characters\footnote{\rruby\ characters have other usages, but are mainly used to indicate pronunciation}. \\


\CJKfontspec{Noto Sans CJK SC}

When adding \rruby\ characters to texts in Standard Mandarin Chinese, \ppinyin\ (see below) are usually used as \rruby. \\


Below is an example of \ppinyin\ used as \rruby:

\begin{center}
	\begin{pinyinscope}
	雪花飄飄\ 北風蕭蕭\par
	天地\ 一片蒼茫\par
	\end{pinyinscope}
\end{center}

\bigskip

\CJKfontspec{Noto Sans JP}

In Japanese, \rruby\ characters are usually called \furigana.\\

Below is the word ``\furigana" (振り仮名), with \furigana\ added to it:
\begin{center}
	\begin{tabular}{c c}
		\furi{振/ふ,り/,仮/が,名/な}    \\
		\furi{振/fu,り/ri,仮/ga,名/na} \\
	\end{tabular}
\end{center}


\bigskip


\emph{Ruby} characters may also be referred to as \emph{rubi}, and may be plurialised as: \emph{ruby}, \emph{rubi}, or \rubies\ (I tend to use the phrase ``\emph{ruby} characters" to avoid the confusion between singular and plural).





\section*{What is \ppinyin?}

\CJKfontspec{Noto Sans CJK SC}

\ppinyin\ is the official romanization system for Standard Mandarin Chinese. \\

The name ``\ppinyin" comes from \emph{``Hànyǔ Pīnyīn"} (汉语拼音), literally: \emph{``to spell the sound of the Han language"}. \\

\ppinyin\ can be used on their own (e.g. ``\pinyin{han4yu3pin1yin1}") or as \rruby\ characters (e.g. \xpinyin*{汉语拼音}). \\

\CJKfontspec{Noto Sans JP}


\section*{About foreign loanwords}

Words of Chinese or Japanese origins are invariable in English. For example, the plural of \emph{anime} or \emph{manga} is ``\emph{anime}", ``\emph{manga}". \\

As such, the word ``\kanji" may refer to either one \kanji\ (i.e. Chinese character, also used in Japanese) or several \kanji. \\

Foreign loanwords (therefore, invariable in plural) used in this article include: \kanji, \hiragana, \katakana, \furigana, \ppinyin.




\section*{Conventions}


Some commands in this article need to be executed with \emph{administrator} privileges (on Linux, that means \emph{root}) to perform operations such as: installing new software on the system, modifying some system configuration, etc. \\

If you have \emph{administrator} privileges on your machine, please read on\ldots

\note{
In this article, we indicate that a Linux command needs to be executed with \emph{root} privileges by using: \\

\cmd{sudo} \\

However, please note that there are cases when the \cmd{sudo} command will not work; for example, if the current user does not appear in the list of \cmd{sudoers} (file \cmd{/etc/sudoers}). \\

If that is your case, please use whichever method you normally use to run a command as \emph{root} (e.g. \emph{doas}, \emph{su}), while keeping in mind that being logged in as \emph{root} is extremely bad practice.
}


If you do NOT have \emph{administrator} privileges on your machine, please contact your administrator.



\newpage
\section*{General Information}

This document was first published at: \\
\mbox{} \hfill \url{https://pcaboche.github.io} 

\subsection*{Legal}
\input{"READ_ME_(LEGAL).txt"}

%%% TOC
\newpage
\renewcommand{\currentPart}{Table of Contents}

\addtocontents{toc}{\setcounter{tocdepth}{3}}
\setcounter{tocdepth}{3}
\tableofcontents

%%% Article body
\newpage
% Copyright 2022 Pierre S. Caboche. All rights reserved.

\renewcommand{\currentPart}{Splitting a MariaDB dump file, on a per-table basis}

\newpage
\part{MySQL / MariaDB : splitting a database dump file, on a per-table basis}

This specific problem will serve as our example throughout the article:

\begin{itemize}
	\item we've been given a big \cmd{.sql} file, which is the result of running \mysqldump to ``backup" a MySQL/MariaDB database
	\item the \cmd{.sql} file contains the data from all our tables
	\item we need to split that data into several \cmd{.sql} files, one file per table
	\item the string \cmd{-- Table structure for table `table_name`} marks the beginning of the definition for table \cmd{table_name}, followed by its data. \
	\item Whenever we encounter such a string, we need to change the output of our script (to a file named \cmd{table_name.sql})
\end{itemize}


In this problem, each table constitutes a ``bucket". \\

If, for example, we wanted to separate the table structure from its actual data, this would constitute 2 buckets per table (one for the structure, and one for the data). To make our script easier to understand, we will keep 1 bucket per table. \\

What's important to remember is:
\begin{itemize}
	\item a ``bucket" represents how the data is separated \emph{logically}
	\item a file is how the bucket data is stored \emph{physically}
	\item we decide to make just 1 bucket per table (and therefore, 1 output file per table)
\end{itemize}



\newpage

\section*{Note on the example}

Our database ``dump" file actually contains more than the tables' definition and data. For example, it may contain definitions for views, stored procedures, and other definitions. \\

A fully working example would require more complex logic, and may also depend on external factors which are out of my control (such as: how the "dump" file was generated in the first place, the version of \mysqldump, and others).

We have deliberately kept the examples in this article simple and easy to follow. \\

The scripts provided in this article are FOR ILLUSTRATION PURPOSE ONLY, USE AT YOUR OWN RISK.


\section*{What the scripts will do}

Our scripts will:
\begin{itemize}
	\item read data from the standard input ( \stdin\ )
	\item write to different files based on some conditions
\end{itemize}


The output file names will follow the same form: \newline
\lstinline|${number}_${table_name}.sql| \\

For example: \cmd{0041_some_table_name.sql} \\


We are numbering our files in order to reassemble them later (and in the correct order), and compare the input with the output of our scripts (this will be important later\dots) \\


The output files will be stored in an output directory. 
We will use a different output directory for each of our implementations (\cmd{python/}, \cmd{awk/}, \cmd{perl/}). This will be useful later, to compare the output results.


\section*{Tools}

Here is a typical example of how we will run a script:

\begin{lstlisting}[language=sh]
$ time pv input_file | ./script.py
\end{lstlisting}

\note{
	We use the linux \cmd{time} command to measure how long a script took to execute.
}


\note{
	We use \emph{pipes} ( \cmd{|} ) to pass data from one process to another. The output of one process ( \stdout ) is passed to the standard input ( \stdin\ ) of the next. \\
	
	Our scripts will read data from the standard input ( \stdin\ ). This approach is generally more flexible that reading data from files.
}

\note{
	We prefer to use the \pv\ (\emph{pipe viewer}) command instead of \cat\ (or the \cmd{<} file indirection).
	
	\pv\ will not only read the file, but also provide extra information, such as: the file size, \emph{current} read speed (in a unit such as MiB/s), and overall progress. This immediately tells us when the read/processing speed is abnormally slow.
}

Please note however, that when talking about the \emph{overall} read speed, we divide the file size by the real time (as reported by \cmd{time}), not what is displayed by \pv\ (which is the \emph{current} read speed at the end of the file processing). 


\section*{Execution environment} \label{execution-environment}


I first ran the experiment on a computer equipped with a fast SSD on NVMe PCIe gen 3. 
We'll call this computer \emph{``gen-8"}, for it has an \nth{8} generation \Intel\ \Core~i5 processor. \\

Later on I got access to a more recent computer (\nth{11} generation \Intel\ \Core~i5) with a faster SSD (PCIe gen 4). I re-ran the tests, and included them for comparison.
We'll call this computer \emph{``gen-11"}. \\



Below is some information about the hardware and software used\ldots

\begin{table}[h]
	\caption{Hardware comparison}
	\centering
	\begin{tabular}{| c | c | c |}
		\hline
			& Computer 1 & Computer 2 \\
		\hline
			& & \\
		Name & gen-8 & gen-11 \\
			& & \\
		CPU & \Intel\ \Core\ i5-8259U CPU & \Intel\ \Core\ i5-1135G7 \\
		    & 8 $\times$ 2.30GHz (max: 3.80 GHz) & 8 $\times$ 2.40GHz (max: 4.20 GHz) \\
		    & \footnotesize Released: Q2'2018 & \footnotesize Released: Q3'2020 \\
			& & \\
		RAM & 16 GB & 16 GB \\
			& (2 $\times$ 8GB, DDR4-2400 MHz) & (2 $\times$ 8GB, DDR4-3200 MHz) \\
			& & \\
		%& ADATA 512 GB& Samsung 980 Pro 1 TB\\
     	SSD & NVMe PCIe \emph{gen 3} & NVMe PCIe \emph{gen 4} \\
		    & Seq. read: $\sim$2.3 GB/s & Seq. read: $\sim$6.5 GB/s \\
     		& Seq. write: $\sim$1.7 GB/s & Seq. write: $\sim$4 GB/s \\
			& \footnotesize Partition format: XFS & \footnotesize Partition format: XFS \\
			& & \\
		\hline
	\end{tabular}
\end{table}

\begin{table}[h]
	\caption{Software versions}
	\centering
	\begin{tabular}{ l  l }
		& \\
		OS & Linux (Fedora 35) \\
		\python & 3.10.5 \\
		GNU Awk &  GNU Awk 5.1.0, API: 3.0 (GNU MPFR 4.1.0-p13, GNU MP 6.2.0) \\
		\perl & v5.34.1 \\
		\julia & 1.7.3 \\
\end{tabular}
\end{table}	



\medskip

I also tested the scripts on a Windows 10 laptop, using WSL 2 (Windows Subsystem for Linux, version 2). \\

The scripts run well under WSL 2, but that laptop had a slower SSD, which impacted the results (some scripts were bottlenecked by the read speed of its SSD, of $\sim$700 MB/s).




\newpage
% Copyright 2022 Pierre S. Caboche. All rights reserved.

\renewcommand{\currentPart}{Python}

\newpage
\part{Python} \label{python}

We will use the \python\ script as a reference. We will study its implementation details later, when we compare the Python implementation with the \awk\ and \perl\ scripts\dots \\


Below is our original implementation, in \python:

\begin{figure}[h]
	\caption{bucket.py}
	\lstinputlisting[language=python]{"files/bucket.py"}
\end{figure}  


\newpage


Make sure this script is executable by performing a:
\begin{lstlisting}[language=sh]
chmod u+x bucket.py
\end{lstlisting}



\section*{A few things to notice\dots}

Below are a few things I would like to point out regarding scripts in general, and this Python script in particular\dots


\subsection*{Specifying the interpreter}

Unlike Windows, Linux does not rely on a file extension to determine how to run a program. As long as a file is marked as ``executable", Linux will consider is as executable. \\

To determine which interpreter to use, the program loader looks at the first line of a script. \\

In a Linux script, the first characters of the first line are \cmd{\#!}, also called \emph{``shebang"} (because it comprises a ha\emph{sh}tag and an exclamation mark, also known as \emph{``bang!"} Hence the name \emph{``shebang"}) \\

Once a \emph{``shebang"} is detected, the program loader reads the rest of the first line to determine which interpreter to use (and with which parameters). \\

Note that we need to specify the full path of the interpreter executable. Usually, this will start with \cmd{/usr/bin} \footnote{\cmd{/usr/bin} contains binaries available to all users in multi-user mode, as opposed to \cmd{/bin} which is contains essential binaries required at boot up and single-user mode}. \\

\bigskip



In this article, we will use the following at the beginning of our scripts\dots \\

For \cmd{python}:
\begin{lstlisting}[language=sh]
#!/usr/bin/python3
\end{lstlisting}

For \gawk:
\begin{lstlisting}[language=sh]
#!/usr/bin/gawk -f
\end{lstlisting}

For \cmd{perl}:
\begin{lstlisting}[language=sh]
#!/usr/bin/perl
\end{lstlisting}

For \cmd{julia}:
\begin{lstlisting}[language=sh]
#!/usr/bin/julia
\end{lstlisting}


\newpage
\subsection*{Incrementing an integer}

\note{
	\python\ doesn't have the \cmd{++} operator for incrementation. \\
	
	In other languages, the misuse of the \cmd{++} operator might lead to unforeseen consequences, which \emph{may} be the reason why \cmd{++} has been left out of \python.
}


If you try using the \cmd{++} operator in \python\ (like you might be doing in other programming languages), then you'll be met with a syntax error, which can be confusing at first. \\

In any case, in \python, to increment an integer, you would need to do a \cmd{+= 1}:

\begin{lstlisting}[language=python]
# Python
file_num += 1
\end{lstlisting}

\medskip


In the other languages ( \awk\ and \perl\ ), we will use the \cmd{++} unary operator to increment integers. For example in \awk:

\begin{lstlisting}[language=awk]
# AWK
file_num++ 
\end{lstlisting}



\subsection*{Compiling the regular expression}

Near the beginning of our script, we have the following instructions:

\begin{lstlisting}[language=python]
pattern=r'^-- Table structure for table `?(?P<TABLE>\w+)`?'
parser = re.compile(pattern, re.IGNORECASE)
\end{lstlisting}

Here, we ``compile" the regular expression once, store the compiled expression in a variable (\cmd{parser}), and then use it again and again ( \cmd{parser.match(line)} ).

This is to avoid compiling the regular expression every time we need it.

\subsection*{Reading from \stdin\ }

We read the content of \stdin\ line by line, with the following loop:
\begin{lstlisting}[language=python]
	for line in sys.stdin :
	# loop instructions...
\end{lstlisting}

In this code, \cmd{sys.stdin} is an \emph{iterator} over \stdin, and our loop follows this pattern:
\begin{lstlisting}[language=python]
for <variable> in <iterator> :
# loop instructions...
\end{lstlisting}


\newpage

\section*{Result} 


The result of execution is as follows (\emph{note: some of the output was edited to fit in the page}):

\begin{lstlisting}[language=sh]
[user@gen-8 test]$ time pv mysql_dump.sql | ./bucket.py
7.42GiB 0:14:55 [8.49MiB/s] [====================================================>] 100%  

real    14m55.335s
user    14m29.250s
sys     0m28.531s
\end{lstlisting}

As you can see, our input file was 7.42 GiB in size, and Python took nearly 15 minutes to process it at an abysmal 8 MiB/s... ( = 7420 / 15 / 60 ) \\

When running the script for the first time, \pv\ showed a very low 10 MiB/s, and my first reaction was: \emph{``did I do something wrong?"}.
As we'll discover later, this was not the case\dots \\

In the meantime (and while the \python\ script was still running\dots), I wanted to try another implementation. So I rewrote the script in \awk\dots


\newpage
\input{"content/part-gawk"}

\newpage
\input{"content/part-perl"}

\newpage
\input{"content/part-julia"}

\newpage
\input{"content/part-other-technical-considerations"}

\newpage
\input{"content/part-python-issues"}

\newpage
\input{"content/part-verifying-results"}

\newpage
\input{"content/part-results-and-conclusion"}

\bigskip

%%% Bibliography

\newpage
\renewcommand{\formatPartTitle}{}

\bibliography{content/biblio-python-awk-perl}
\bibliographystyle{apalike}

\bigskip
\bigskip

\listoffigures
\listoftables


% Label on the last page. Allows to easily get the page number
\label{LastPage}

\end{document}
