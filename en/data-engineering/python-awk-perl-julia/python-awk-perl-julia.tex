
% Copyright 2022 Pierre S. Caboche. All rights reserved.

% Note: use XeLaTeX for rendering
% !TEX encoding = UTF-8


\documentclass[twoside]{article}

\setlength{\parindent}{0pt}

%% Packages and their configuration
% Copyright 2022 Pierre S. Caboche. All rights reserved.

%% Packages and configuration

\usepackage{geometry}
\geometry{
	a4paper,
	total={170mm,257mm},
	left=35mm,
	right=35mm,
	top=30mm,
	bottom=30mm,
}


\usepackage{fontspec}
\setmainfont{DejaVu Serif} 
\setsansfont{DejaVu Sans} 

%% Use Sans-Serif by default
\renewcommand{\familydefault}{\sfdefault}


\usepackage{xeCJK}
\setCJKmainfont{Noto Serif JP}
\setCJKsansfont{Noto Sans JP}
\setCJKmonofont{Noto Sans Mono CJK JP}

\usepackage{xpinyin}


%% bibliography
\usepackage[authoryear]{natbib}

%% index
\usepackage{imakeidx}
\makeindex


\usepackage{xcolor}
\usepackage{graphicx}


\usepackage{titleps}
\usepackage{xstring}
\newcommand{\currentPart}{}

\newcommand{\formatPartTitleDefault}{
	\textbf{\IfStrEq{\thepart}{}{}{Part \thepart\ ---  }\currentPart}
}

\newcommand{\formatPartTitle}{\formatPartTitleDefault}

\newcommand{\copyrightNotice}{\color{gray}{\emph{\textcopyright\ 2022 Pierre S. Caboche. All rights reserved.}}}

\newpagestyle{main}{
	\sethead
	% even
	[\formatPartTitle]
	[]
	[]
	% odd
	{}
	{}
	{\formatPartTitle}
	
	\setfoot
	% even
	[\textbf{\thepage\ \color{lightgray}{|}}]
	[]
	[\copyrightNotice]
	% odd
	{\copyrightNotice}
	{}
	{\textbf{\color{lightgray}{|} \color{black}{\thepage} }} 
	
	\headrule
	%\footrule
}
\pagestyle{main}

\widenhead[25pt][25pt]{25pt}{25pt}



\usepackage{listings}
\lstset{
	basicstyle=\fontsize{9}{9}\selectfont\ttfamily
	,frame=lines
	,tabsize=2
	,keywordstyle=\bfseries%\itshape,
	,commentstyle=\itshape\color{teal}
	,stringstyle=\color{magenta}
	,breaklines=true,
	,postbreak=\mbox{\textcolor{red}{$\hookrightarrow$}\space}
}



\usepackage{hyperref}
\usepackage{longtable}
\usepackage[super]{nth}



\widowpenalties 1 10000
\raggedbottom

\setlength{\parindent}{0pt}



%% Custom macros
% Copyright 2022 Pierre S. Caboche. All rights reserved.

\usepackage{mdframed}
\mdfsetup{
	%innertopmargin=10pt,
	frametitleaboveskip=-\ht\strutbox,
	frametitlealignment=\center
}


\newcommand{\note}[1]{
	\mdfsetup{
		middlelinewidth=2pt,
		backgroundcolor=yellow!10
	}
	\begin{mdframed}#1\end{mdframed}
	\mdfsetup{
		backgroundcolor=white
	}
}


\newcommand{\cmd}[1]{\lstinline|#1|}


\newcommand{\python}{Python}
\newcommand{\awk}{AWK}
\newcommand{\perl}{Perl}
\newcommand{\julia}{Julia}
\newcommand{\gawk}{\texttt{gawk}}
\newcommand{\mawk}{\texttt{mawk}}
\newcommand{\sed}{\texttt{sed}}
\newcommand{\grep}{\texttt{grep}}

\newcommand{\stdin}{\texttt{stdin}}
\newcommand{\stdout}{\texttt{stdout}}
\newcommand{\stderr}{\texttt{stderr}}

\newcommand{\mysqldump}{\texttt{mysqldump}}
\newcommand{\cat}{\texttt{cat}}
\newcommand{\pv}{\texttt{pv}}
\newcommand{\shell}{\texttt{shell}}

\newcommand{\Unix}{UNIX\textregistered}
\newcommand{\Intel}{Intel\textregistered}
\newcommand{\Core}{Core\texttrademark}



% Opening

\title{Python vs. AWK, Perl, Julia: \\ Processing large data files}
\author{Pierre S. Caboche}

%% Custom format to indicate revision date (if needed)
\date{
	July 2, 2022
	% \\	\medskip \footnotesize (Revised: \today)
}


\begin{document}
	
	\maketitle
	
	\begin{abstract}
		In this article we will learn how to perform a very useful task (processing a large text file, splitting data into ``buckets") using three different scripting languages: Python, AWK, and Perl.
		This will allow use to compare those four languages in terms of features, syntax, and performance. \\
		
		This will also give us the opportunity to learn more about these languages. Our examples and explanations should help you easily get started, even if you have no prior experience with Python, AWK, Perl, or Julia.
	\end{abstract}


\bigskip
\bigskip
\bigskip

%%% Intro


\renewcommand{\currentPart}{Preamble}


\section*{Target audience}

This article might be of interest for:
\begin{itemize}
	\item programmers in general
	\item people who need to quickly get started with \python, \awk, \perl, or \julia
	\item Data Engineers, who regularly need to write scripts for processing large files
	\begin{itemize}
		\item in our example, we will split data into ``buckets".
		\item specifically, we will take a MariaDB database dump file, and separate the data on a per-table basis
	\end{itemize}
\end{itemize}




\newpage

\section*{Introduction}

\python\ is a general-purpose language which has become extremely popular in domains such as Big Data, Data Engineering, Machine Learning, etc. \python\ also has a number of applications in other fields due to its flexibility and relative ease of use. \\

In this article, we will try to use \python\ to process a large text file (a very common problem in Big Data and other fields), evaluate Python's abilities to accomplish this task, and then compare it with solutions using other scripting languages (namely: \awk, \perl, and \julia). \\

The problem we are trying to solve is the following: we have been given a huge text file, and its data needs to be separated into several smaller, more manageable files (``buckets" of data) based on certain conditions. \\

In our example, the file will be a ``backup" (dump file) of a MySQL / MariaDB database, as produced by the \mysqldump\ utility, and we need to separate the data on a per-table basis (1 file per table). \\

I have chosen this example because:
\begin{itemize}
	\item it should be easy to follow (we will deliberately keep the examples simple)
	\item it touches different domains
	\begin{itemize}
		\item Data Engineering (for the main concept)
		\item MySQL / MariaDB database administration (for the example)
	\end{itemize}
\end{itemize}


\medskip

\subsection*{Scripting languages}

To process our files, we will be using scripting languages only (Python, AWK, Perl).
The advantage of scripting languages is that the scripts can be easily edited, without the need to recompile the code into an executable. \\

Our scripts will be compatible with the Linux Command Line Interface (CLI). 
On Windows, it is possible to run such scripts using Windows Subsystem for Linux (WSL) 2. \\




In this article, we see how to implement a very common problem (processing a text file), in the following scripting languages: 

\begin{itemize}
	\item \python
	\item AWK
	\item \perl
	\item \julia
\end{itemize}

\newpage

For each language, we will see how to perform some common operations. This will not only allow us to easily get started with these languages, but it will also allow us to compare how these operations are performed in each language: \\
\begin{itemize}
	\item General
	\begin{itemize}
		\item how to run the script
		\item variables (strings, integers)
		\item string operation (concatenation, formatting)
	\end{itemize}
	\item Files and I/O
	\begin{itemize}
		\item how to read a stream of data from the standard input ( \stdin\ )
		\item how to open and close files
		\item how to write to standard output ( \stdout\ ), standard error ( \stderr\ ), and files
	\end{itemize}
	\item Regular expressions
	\begin{itemize}
		\item how to test if the data matches a certain \emph{regular expression}
		\item how to capture groups in a \emph{regular expression} (named groups, vs. numbered groups)
	\end{itemize}
\end{itemize}


\medskip

\subsection*{The general problem}


The type of problem we're trying to solve is of the following form:

\begin{itemize}
	\item we have a text file containing a lot of data
	\item this data needs to be split into several smaller ``buckets" (one file per bucket)
	\item we are reading the input file line by line, and writing to different output files
	\item by analysing the content of a line, we decide when to switch between the different output files
\end{itemize}


\medskip

For example, we may have a file with the following template:

\begin{lstlisting}[language=Python]
# The following needs to go to file #1
Some data here...
(...)

# The following needs to go to file #2
More data there...
(...)

# The following needs to go to file #1
Even more data, which needs to go to the first bucket...
(...)	
\end{lstlisting}

\medskip


In this example, the ``bucket switch" is indicated by a line of the form:
\begin{lstlisting}[language=Python]
# The following needs to go to file ...
\end{lstlisting}
\dots and we determine which bucket to switch to based on the content of the line.



\newpage
\section*{General Information}

This document was first published at: \\
\mbox{} \hfill \url{https://pcaboche.github.io} 

\subsection*{Legal}
\input{"READ_ME_(LEGAL).txt"}

%%% TOC
\newpage
\renewcommand{\currentPart}{Table of Contents}

\addtocontents{toc}{\setcounter{tocdepth}{3}}
\setcounter{tocdepth}{3}
\tableofcontents

%%% Article body
\newpage
% Copyright 2022 Pierre S. Caboche. All rights reserved.

\renewcommand{\currentPart}{Splitting a MariaDB dump file, on a per-table basis}

\newpage
\part{MySQL / MariaDB : splitting a database dump file, on a per-table basis}

This specific problem will serve as our example throughout the article:

\begin{itemize}
	\item we've been given a big \cmd{.sql} file, which is the result of running \mysqldump to ``backup" a MySQL/MariaDB database
	\item the \cmd{.sql} file contains the data from all our tables
	\item we need to split that data into several \cmd{.sql} files, one file per table
	\item the string \cmd{-- Table structure for table `table_name`} marks the beginning of the definition for table \cmd{table_name}, followed by its data. \
	\item Whenever we encounter such a string, we need to change the output of our script (to a file named \cmd{table_name.sql})
\end{itemize}


In this problem, each table constitutes a ``bucket". \\

If, for example, we wanted to separate the table structure from its actual data, this would constitute 2 buckets per table (one for the structure, and one for the data). To make our script easier to understand, we will keep 1 bucket per table. \\

What's important to remember is:
\begin{itemize}
	\item a ``bucket" represents how the data is separated \emph{logically}
	\item a file is how the bucket data is stored \emph{physically}
	\item we decide to make just 1 bucket per table (and therefore, 1 output file per table)
\end{itemize}



\newpage

\section*{Note on the example}

Our database ``dump" file actually contains more than the tables' definition and data. For example, it may contain definitions for views, stored procedures, and other definitions. \\

A fully working example would require more complex logic, and may also depend on external factors which are out of my control (such as: how the "dump" file was generated in the first place, the version of \mysqldump, and others).

We have deliberately kept the examples in this article simple and easy to follow. \\

The scripts provided in this article are FOR ILLUSTRATION PURPOSE ONLY, USE AT YOUR OWN RISK.


\section*{What the scripts will do}

Our scripts will:
\begin{itemize}
	\item read data from the standard input ( \stdin\ )
	\item write to different files based on some conditions
\end{itemize}


The output file names will follow the same form: \newline
\lstinline|${number}_${table_name}.sql| \\

For example: \cmd{0041_some_table_name.sql} \\


We are numbering our files in order to reassemble them later (and in the correct order), and compare the input with the output of our scripts (this will be important later\dots) \\


The output files will be stored in an output directory. 
We will use a different output directory for each of our implementations (\cmd{python/}, \cmd{awk/}, \cmd{perl/}). This will be useful later, to compare the output results.


\section*{Tools}

Here is a typical example of how we will run a script:

\begin{lstlisting}[language=sh]
$ time pv input_file | ./script.py
\end{lstlisting}

\note{
	We use the linux \cmd{time} command to measure how long a script took to execute.
}


\note{
	We use \emph{pipes} ( \cmd{|} ) to pass data from one process to another. The output of one process ( \stdout ) is passed to the standard input ( \stdin\ ) of the next. \\
	
	Our scripts will read data from the standard input ( \stdin\ ). This approach is generally more flexible that reading data from files.
}

\note{
	We prefer to use the \pv\ (\emph{pipe viewer}) command instead of \cat\ (or the \cmd{<} file indirection).
	
	\pv\ will not only read the file, but also provide extra information, such as: the file size, \emph{current} read speed (in a unit such as MiB/s), and overall progress. This immediately tells us when the read/processing speed is abnormally slow.
}

Please note however, that when talking about the \emph{overall} read speed, we divide the file size by the real time (as reported by \cmd{time}), not what is displayed by \pv\ (which is the \emph{current} read speed at the end of the file processing). 


\section*{Execution environment} \label{execution-environment}


I first ran the experiment on a computer equipped with a fast SSD on NVMe PCIe gen 3. 
We'll call this computer \emph{``gen-8"}, for it has an \nth{8} generation \Intel\ \Core~i5 processor. \\

Later on I got access to a more recent computer (\nth{11} generation \Intel\ \Core~i5) with a faster SSD (PCIe gen 4). I re-ran the tests, and included them for comparison.
We'll call this computer \emph{``gen-11"}. \\



Below is some information about the hardware and software used\ldots

\begin{table}[h]
	\caption{Hardware comparison}
	\centering
	\begin{tabular}{| c | c | c |}
		\hline
			& Computer 1 & Computer 2 \\
		\hline
			& & \\
		Name & gen-8 & gen-11 \\
			& & \\
		CPU & \Intel\ \Core\ i5-8259U CPU & \Intel\ \Core\ i5-1135G7 \\
		    & 8 $\times$ 2.30GHz (max: 3.80 GHz) & 8 $\times$ 2.40GHz (max: 4.20 GHz) \\
		    & \footnotesize Released: Q2'2018 & \footnotesize Released: Q3'2020 \\
			& & \\
		RAM & 16 GB & 16 GB \\
			& (2 $\times$ 8GB, DDR4-2400 MHz) & (2 $\times$ 8GB, DDR4-3200 MHz) \\
			& & \\
		%& ADATA 512 GB& Samsung 980 Pro 1 TB\\
     	SSD & NVMe PCIe \emph{gen 3} & NVMe PCIe \emph{gen 4} \\
		    & Seq. read: $\sim$2.3 GB/s & Seq. read: $\sim$6.5 GB/s \\
     		& Seq. write: $\sim$1.7 GB/s & Seq. write: $\sim$4 GB/s \\
			& \footnotesize Partition format: XFS & \footnotesize Partition format: XFS \\
			& & \\
		\hline
	\end{tabular}
\end{table}

\begin{table}[h]
	\caption{Software versions}
	\centering
	\begin{tabular}{ l  l }
		& \\
		OS & Linux (Fedora 35) \\
		\python & 3.10.5 \\
		GNU Awk &  GNU Awk 5.1.0, API: 3.0 (GNU MPFR 4.1.0-p13, GNU MP 6.2.0) \\
		\perl & v5.34.1 \\
		\julia & 1.7.3 \\
\end{tabular}
\end{table}	



\medskip

I also tested the scripts on a Windows 10 laptop, using WSL 2 (Windows Subsystem for Linux, version 2). \\

The scripts run well under WSL 2, but that laptop had a slower SSD, which impacted the results (some scripts were bottlenecked by the read speed of its SSD, of $\sim$700 MB/s).




\newpage
% Copyright 2022 Pierre S. Caboche. All rights reserved.

\renewcommand{\currentPart}{Python}

\newpage
\part{Python} \label{python}

We will use the \python\ script as a reference. We will study its implementation details later, when we compare the Python implementation with the \awk\ and \perl\ scripts\dots \\


Below is our original implementation, in \python:

\begin{figure}[h]
	\caption{bucket.py}
	\lstinputlisting[language=python]{"files/bucket.py"}
\end{figure}  


\newpage


Make sure this script is executable by performing a:
\begin{lstlisting}[language=sh]
chmod u+x bucket.py
\end{lstlisting}



\section*{A few things to notice\dots}

Below are a few things I would like to point out regarding scripts in general, and this Python script in particular\dots


\subsection*{Specifying the interpreter}

Unlike Windows, Linux does not rely on a file extension to determine how to run a program. As long as a file is marked as ``executable", Linux will consider is as executable. \\

To determine which interpreter to use, the program loader looks at the first line of a script. \\

In a Linux script, the first characters of the first line are \cmd{\#!}, also called \emph{``shebang"} (because it comprises a ha\emph{sh}tag and an exclamation mark, also known as \emph{``bang!"} Hence the name \emph{``shebang"}) \\

Once a \emph{``shebang"} is detected, the program loader reads the rest of the first line to determine which interpreter to use (and with which parameters). \\

Note that we need to specify the full path of the interpreter executable. Usually, this will start with \cmd{/usr/bin} \footnote{\cmd{/usr/bin} contains binaries available to all users in multi-user mode, as opposed to \cmd{/bin} which is contains essential binaries required at boot up and single-user mode}. \\

\bigskip



In this article, we will use the following at the beginning of our scripts\dots \\

For \cmd{python}:
\begin{lstlisting}[language=sh]
#!/usr/bin/python3
\end{lstlisting}

For \gawk:
\begin{lstlisting}[language=sh]
#!/usr/bin/gawk -f
\end{lstlisting}

For \cmd{perl}:
\begin{lstlisting}[language=sh]
#!/usr/bin/perl
\end{lstlisting}

For \cmd{julia}:
\begin{lstlisting}[language=sh]
#!/usr/bin/julia
\end{lstlisting}


\newpage
\subsection*{Incrementing an integer}

\note{
	\python\ doesn't have the \cmd{++} operator for incrementation. \\
	
	In other languages, the misuse of the \cmd{++} operator might lead to unforeseen consequences, which \emph{may} be the reason why \cmd{++} has been left out of \python.
}


If you try using the \cmd{++} operator in \python\ (like you might be doing in other programming languages), then you'll be met with a syntax error, which can be confusing at first. \\

In any case, in \python, to increment an integer, you would need to do a \cmd{+= 1}:

\begin{lstlisting}[language=python]
# Python
file_num += 1
\end{lstlisting}

\medskip


In the other languages ( \awk\ and \perl\ ), we will use the \cmd{++} unary operator to increment integers. For example in \awk:

\begin{lstlisting}[language=awk]
# AWK
file_num++ 
\end{lstlisting}



\subsection*{Compiling the regular expression}

Near the beginning of our script, we have the following instructions:

\begin{lstlisting}[language=python]
pattern=r'^-- Table structure for table `?(?P<TABLE>\w+)`?'
parser = re.compile(pattern, re.IGNORECASE)
\end{lstlisting}

Here, we ``compile" the regular expression once, store the compiled expression in a variable (\cmd{parser}), and then use it again and again ( \cmd{parser.match(line)} ).

This is to avoid compiling the regular expression every time we need it.

\subsection*{Reading from \stdin\ }

We read the content of \stdin\ line by line, with the following loop:
\begin{lstlisting}[language=python]
	for line in sys.stdin :
	# loop instructions...
\end{lstlisting}

In this code, \cmd{sys.stdin} is an \emph{iterator} over \stdin, and our loop follows this pattern:
\begin{lstlisting}[language=python]
for <variable> in <iterator> :
# loop instructions...
\end{lstlisting}


\newpage

\section*{Result} 


The result of execution is as follows (\emph{note: some of the output was edited to fit in the page}):

\begin{lstlisting}[language=sh]
[user@gen-8 test]$ time pv mysql_dump.sql | ./bucket.py
7.42GiB 0:14:55 [8.49MiB/s] [====================================================>] 100%  

real    14m55.335s
user    14m29.250s
sys     0m28.531s
\end{lstlisting}

As you can see, our input file was 7.42 GiB in size, and Python took nearly 15 minutes to process it at an abysmal 8 MiB/s... ( = 7420 / 15 / 60 ) \\

When running the script for the first time, \pv\ showed a very low 10 MiB/s, and my first reaction was: \emph{``did I do something wrong?"}.
As we'll discover later, this was not the case\dots \\

In the meantime (and while the \python\ script was still running\dots), I wanted to try another implementation. So I rewrote the script in \awk\dots


\newpage
\input{"content/part-gawk"}

\newpage
\input{"content/part-perl"}

\newpage
\input{"content/part-julia"}

\newpage
\input{"content/part-other-technical-considerations"}

\newpage
\input{"content/part-python-issues"}

\newpage
\input{"content/part-verifying-results"}

\newpage
\input{"content/part-results-and-conclusion"}

\bigskip

%%% Bibliography

\newpage
\renewcommand{\formatPartTitle}{}

\bibliography{content/biblio-python-awk-perl}
\bibliographystyle{apalike}

\bigskip
\bigskip

\listoffigures
\listoftables


% Label on the last page. Allows to easily get the page number
\label{LastPage}

\end{document}
