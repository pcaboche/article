% Copyright 2022 Pierre S. Caboche. All rights reserved.

\part{\LaTeX\ vs. \LibreOffice, \Word\dots} \label{msword}

\renewcommand{\currentPart}{\LaTeX\ vs. \LibreOffice, \Word\dots}

I started to use \LaTeX\ because of how easy it is to add \rruby\ characters. 
After experimenting with \LibreOffice\ and \Word, I tried \LaTeX\ and was genuinely impressed by how much time it saved me for this task\dots \\

To properly grasp how effective \LaTeX\ is at adding \rruby\ characters, we need to understand how the same process is done in \LibreOffice\ and \Word.


\section*{\LibreOffice}

\subsection*{Configuration}

This is a quick guide to enabling \rruby\ character support in \LibreOffice.

\subsubsection*{Language settings}

\begin{itemize}
	\item Go to \emph{Tools > Options\dots}
	\item under \emph{Language settings > Languages}
	\begin{itemize}
		\item check the \emph{Asian} box
		\item select a default language (e.g. \emph{Japanese})
	\end{itemize}
	\item Click \emph{OK}
\end{itemize}

This should enable the support for features like \emph{Asian Phonetic Guide}

\subsubsection*{Edit the context menu}

\begin{itemize}
	\item Go to \emph{Tools > Customize\dots} \\
	This allow you to edit the Context Menu (accessible when you right-click on something)
	
	\item Under \emph{Target} (to the right)
	\begin{itemize}
		\item Select \emph{Text} \\
		This allows you to edit the Context Menu for when you right-click on some selected text
	\end{itemize}

	\item Under \emph{Search} (to the top-left)
	\begin{itemize}
		\item Search for \emph{``Asian Phonetic Guide"} \\
		\emph{Asian Phonetic Guide} should appear in the list of \emph{Available Commands}
		\item Move \emph{Asian Phonetic Guide} from \emph{Available Commands} (the menu on the left) to \emph{Assigned Commands} (the menu on the right)
	\end{itemize}
	\item Click \emph{OK}
\end{itemize}

From now on, \emph{Asian Phonetic Guide} should be available when you right click on some selected text.


\subsection*{Adding \ppinyin\ with \LibreOffice}

\begin{itemize}
	\item Select some text
	\item Right-click \\
	If you've followed the configuration above, then \emph{``Asian Phonetic Guide\dots"} should appear in the menu
	\item Select \emph{``Asian Phonetic Guide\dots"} \\
	A menu called \emph{``Asian Phonetic Guide"} should appear
\end{itemize}

The \emph{``Asian Phonetic Guide"} menu allows you to specify the \rruby\ characters you want to add to the text. 
You also need to specify the alignment, position, and style for the \rruby\ characters. \\


\subsection*{A very slow process\dots} \label{slow-process}

In \LibreOffice, you will need to repeat the above process for virtually \textbf{\emph{every}} character for which you want to add some \rruby\ annotation. \\
	
This is slow and tedious\dots \\
	
Not only that, but should you wish to change the size or alignment of the \rruby\ characters, then you will also have to repeat the process for  \textbf{\emph{every}} character in your document. \\



\section*{\Word}

The process is somewhat similar in \Microsoft\ \Word. \\

\Microsoft\ \Word\ suffers from the same quirk as \LibreOffice, where you need to specify the \rruby\ (and style, size, alignment) for nearly \textbf{\emph{every}} character in your document\dots \\

\Word\ has one advantage over \LibreOffice: for each selected character, \Word\ suggests a possible \rruby\dots
\ but \LaTeX\ does even better (at least in Mandarin) by automatically adding \ppinyin\ to your text, thanks to the \xxpinyin\ package (and \LaTeX\ does it for free!) \\

The fact that \LaTeX\ also produces beautiful documents is just the icing on the cake\dots


\newpage

\renewcommand{\currentPart}{Conclusion}

\part{\currentPart}

I started to use \LaTeX\ out of necessity, because I needed a way to write documents (for my personal use) which would contain a lot of Japanese \furigana\ (\rruby).
After some experimentation, I found that adding \rruby\ characters was considerably more efficient in \LaTeX\ than in either \LibreOffice\ or \Word\ (see \emph{Part~\ref{msword}}). \\

The process of using \LaTeX\ also turned out to be easier than I initially thought. \LaTeX\ produces very good-looking documents by default, which you can customise if you need to (like I did in this article and others\ldots) 

But the main reason for me to use \LaTeX\ was the possibility to define custom commands which you can then \emph{re-use} at will, to produce documents with a consistent layout (this, in my opinion, is one of \LaTeX's biggest strengths). \\

So this is, in a nutshell, how I started out with \LaTeX\ldots \\

And everything was fine, until I tried to render my \LaTeX\ files on a different Operating System\ldots\ \\

 
Indeed, the way I wrote my \LaTeX\ files prevented them from properly render on \Linux\ (which I'm now using more and more, including at home). 
Solving this issue was the subject of \emph{Parts~\ref{fonts} and~\ref{latex}} (which apply not only to Japanese and Chinese, but any language that use a non-Latin writing script). \\

At first I was focused on documents that feature texts in Japanese language, and how to add \furigana\ (Part~\ref{japanese}). Then I discovered the \texttt{xpinyin} package, which does much of the same but is tailored towards Standard Mandarin Chinese. I found it to be a related issue, and therefore decided to cover it as well (in Part~\ref{chinese}). \\

Hopefully this document will have helped you discover what \LaTeX\ can do, and how to handle languages like Japanese, Chinese, or Korean (and other non-Latin writing scripts).


\section*{What we learned}

This article presented the following subjects:
\begin{itemize}
	\setlength\itemsep{0em}
	\item fonts and non-latin writing systems
	\item Noto fonts
	\item \rruby\ characters, \furigana, \ppinyin
	\item some aspects specific to the Japanese and Chinese languages
	\item the \texttt{ruby} package
	\item the \texttt{xpinyin} package
	\item the use of custom macros (to add \furigana) 
\end{itemize}

\bigskip


