% Copyright 2022 Pierre S. Caboche. All rights reserved.

\part{Chinese texts in \LaTeX} \label{chinese}

\renewcommand{\currentPart}{Chinese texts in \LaTeX}


\CJKfontspec{Noto Sans CJK SC}

In the previous parts, we saw the following:

\begin{itemize}
	\item how to install \LaTeX, for Windows and Linux
	\item how to find portable fonts for different writing systems
	\item how to use the CJK fonts in \XeLaTeX
	\item how to add \emph{ruby} characters to text
\end{itemize}

In our example, we saw how to add \emph{furigana} to Japanese texts. This is the general technique, which can be used for adding \emph{ruby} characters to texts in any language or dialect. \\

However, if your goal is to add the \emph{pinyin} pronunciation to text in Mandarin Chinese (Simplified or Traditional), then there is a \LaTeX\ package to do that: \texttt{xpinyin}.



\section*{Automatically add \ppinyin}

As we saw in the previous part, a \kanji\ in Japanese can have multiple pronunciations.
In Chinese, however, things are different. With a few exceptions, most \kanji\ in Mandarin Chinese have only one possible pronunciation. \\

Thanks to this (almost) 1-to-1 match, it is possible to know (with a high degree of confident) how a given \kanji\ should be pronounced in Mandarin Chinese. The \texttt{xpinyin} package was developed to automatically add the \ppinyin. \\

As mentioned earlier, there are some exceptions\dots \\
For example, the character 乐 may be pronounced ``\pinyin{le4}" as in \xpinyin*{快乐} (happy), or ``\pinyin{yue4}" as in \xpinyin*{音\xpinyin{乐}{yue4}} (music). 
When that happens, it is possible to specify the pronunciation manually with the  \texttt{\textbackslash xpinyin} macro (examples below).


\section*{The \texttt{xpinyin} package}

To use the \texttt{xpinyin} package, add the following in your document preamble:
\begin{lstlisting}
\usepackage{xpinyin}
\end{lstlisting}

The \texttt{xpinyin} package allows to add \emph{pinyin} in your \LaTeX\ document. \\ 

Here is a quick introduction to some of its features:

\begin{description}
	\item[\texttt{\textbackslash pinyin}] \mbox{}\\
	It is used to output pinyin. For the convenience of input, ü can be replaced by v. \\
	Examples:
	\begin{tabular}{ l l }
		\lstinline|\pinyin{lv2zi}|     & $\rightarrow$ \pinyin{lv2zi}\\
		\lstinline|\pinyin{nv3hai2zi}| & $\rightarrow$ \pinyin{nv3hai2zi} \\	
	\end{tabular}

	\newpage
	Below is an example of each of the ``four tones" (+neutral tone) of Mandarin Chinese:
	
	\begin{tabular}{ l l l }
		  \lstinline|\pinyin{ma1}| $\rightarrow$ \pinyin{ma1} 
		& \lstinline|\pinyin{ma2}| $\rightarrow$ \pinyin{ma2} 
		& \lstinline|\pinyin{ma3}| $\rightarrow$ \pinyin{ma3} \\
		
		  \lstinline|\pinyin{ma4}| $\rightarrow$ \pinyin{ma4}
		& \lstinline|\pinyin{ma} |  $\rightarrow$ \pinyin{ma}  \\
	\end{tabular}
	\bigskip
	
	\item[\texttt{\textbackslash xpinyin}] \mbox{}\\
	\lstinline|\xpinyin| can be used to set pinyin. \\
	This is useful within \texttt{\textbackslash xpinyin*} macro or \texttt{pinyinscope} environment. \\
	Example: \\
	\begin{tabular}{ l l }
		\lstinline|\xpinyin{乐}{yue4}| & $\rightarrow$ \xpinyin{乐}{yue4}\\	
	\end{tabular}
	
	\item[\texttt{\textbackslash xpinyin*}] \mbox{}\\
	Automatic phonetic notation for Chinese characters
	Example: \\
	\begin{tabular}{ l l }
		\lstinline|\xpinyin*{甄士隐梦幻识通灵}| 
		& $\rightarrow$ \xpinyin*{甄士隐梦幻识通灵}
		\\
		
		\lstinline|\xpinyin*{\xpinyin{重}{zhong4}要}| 
		& $\rightarrow$ \xpinyin*{\xpinyin{重}{zhong4}要}
		\\	
	\end{tabular}
	
	\item[\texttt{pinyinscope}] \mbox{}\\
	Automatic phonetic notation for Chinese characters in the \texttt{pinyinscope} environment (useful for long texts). \\
	Syntax:
\begin{lstlisting}
\begin{pinyinscope}[⟨options⟩]
	...
\end{pinyinscope}
\end{lstlisting}
\end{description}




For the rest of the documentation (in Chinese only), see: \cite{xpinyin} at\\
\url{https://ctan.org/pkg/xpinyin}


\bigskip
\bigskip
\subsection*{Example}

Below is a complete example:\\

\newpage


\texttt{example-03-pinyin.tex}
\lstinputlisting[language=tex]{"files/example-03-pinyin.tex"}

\bigskip
\bigskip
And here is the result:

\begin{mdframed}[frametitle={\colorbox{white}{Text with \emph{pinyin}}}]
	\begin{itemize}  	
		\item \xpinyin*{妈麻马骂吗} $\leftarrow$ the ``four tones" (+neutral tone) of Mandarin Chinese 
		\item \xpinyin*{快乐的音乐} $\leftarrow$ the \emph{pinyin} on the last character is wrong
		\item \xpinyin*{快乐的音\xpinyin{乐}{yue4}} $\leftarrow$ this is correct (we specified the last \emph{pinyin} manually)
	\end{itemize}

	\bigskip

	\begin{pinyinscope}
You can also use the \texttt{pinyinscope} environment:\par
\bigskip
滚滚长江东逝水,浪花淘尽英雄。\par
是非成败转头空,青山依旧在,几度夕阳红。\par
白发渔樵江渚上,惯看秋月春风。\par
一壶浊酒喜相逢,古今多少事,都付笑谈中。\par
\bigskip
是非成败转头空,青山依旧在,惯看秋月春风。\par
一壶浊酒喜相逢,古今多少事,滚滚长江东逝水,浪花淘尽英雄。\par
几度夕阳红。白发渔樵江渚上,都付笑谈中。。。。。。\par
	\end{pinyinscope}	
\end{mdframed}
