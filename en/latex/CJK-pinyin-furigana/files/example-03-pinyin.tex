% This source code is under the BSD License
% Copyright 2022 Pierre S. Caboche

% Note: use XeLaTeX for rendering
%!TEX encoding = UTF-8

\documentclass{article}

\usepackage{xeCJK}
\usepackage{xpinyin}
\setCJKmainfont{Noto Sans CJK SC} % Simplified Chinese

\begin{document}
	\begin{itemize}
		\item \xpinyin*{妈麻马骂吗} $\leftarrow$ the ``four tones" (+neutral tone) of Mandarin Chinese
		\item \xpinyin*{快乐的音乐} $\leftarrow$ the \emph{pinyin} on the last character is wrong
		\item \xpinyin*{快乐的音\xpinyin{乐}{yue4}} $\leftarrow$ this is correct (we specified the last \emph{pinyin} manually)
	\end{itemize}

	\bigskip

	\begin{pinyinscope}
You can also use the \texttt{pinyinscope} environment:\par
\bigskip
滚滚长江东逝水,浪花淘尽英雄。\par
是非成败转头空,青山依旧在,几度夕阳红。\par
白发渔樵江渚上,惯看秋月春风。\par
一壶浊酒喜相逢,古今多少事,都付笑谈中。\par
\bigskip
是非成败转头空,青山依旧在,惯看秋月春风。\par
一壶浊酒喜相逢,古今多少事,滚滚长江东逝水,浪花淘尽英雄。\par
几度夕阳红。白发渔樵江渚上,都付笑谈中。。。。。。\par
	\end{pinyinscope}	
\end{document}
