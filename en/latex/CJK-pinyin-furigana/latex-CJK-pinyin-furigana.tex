% Note: use XeLaTeX for rendering
% !TEX encoding = UTF-8


\documentclass[twoside]{article}


%% Packages and their configuration
%%%%%% Packages and configuration

\usepackage{geometry}
\geometry{
	a4paper,
	total={170mm,257mm},
	left=35mm,
	right=35mm,
	top=30mm,
	bottom=30mm,
}


\usepackage{fontspec}
\setmainfont{DejaVu Serif} 
\setsansfont{DejaVu Sans} 


%% Use Sans-Serif by default
\renewcommand{\familydefault}{\sfdefault}



%% bibliography
\usepackage[authoryear]{natbib}


\usepackage{pgfplots}


\usepackage{titleps}
\usepackage{xstring}
\newcommand{\currentPart}{}

\newcommand{\formatPartTitleDefault}{
	\textbf{\IfStrEq{\thepart}{}{}{Part \thepart\ ---  }\currentPart}
}

\newcommand{\formatPartTitle}{\formatPartTitleDefault}

\newcommand{\copyrightNotice}{\color{gray}{\emph{\textcopyright\ 2022 Pierre S. Caboche. All rights reserved.}}}


\newpagestyle{main}{
	\sethead
		% even
		[\formatPartTitle]
		[]
		[]
		% odd
		{}
		{}
		{\formatPartTitle}
	
	\setfoot
		% even
		[\textbf{\thepage\ \color{lightgray}{|}}]
		[]
		[\copyrightNotice]
		% odd
		{\copyrightNotice}
		{}
		{\textbf{\color{lightgray}{|} \color{black}{\thepage} }} 
	
	\headrule
	%\footrule
}
\pagestyle{main}

\widenhead[25pt][25pt]{25pt}{25pt}




\usepackage{listings}
\lstset{
	basicstyle=\fontsize{10}{10}\selectfont\ttfamily
	,frame=lines
	,tabsize=2
	,keywordstyle=\bfseries%\itshape,
	,commentstyle=\itshape\color{teal}
	,stringstyle=\color{magenta}
	,breaklines=true,
	,postbreak=\mbox{\textcolor{red}{$\hookrightarrow$}\space}
}



\usepackage{hyperref}
\usepackage[super]{nth}

\usepackage{marginnote}
\setlength{\marginparwidth}{70pt}


\widowpenalties 1 10000
\raggedbottom




%% Custom macros


\newcommand{\insertBlankPage}[1][This page intentionally left blank.]{
\newpage
\thispagestyle{empty} % remove header and footer for current page
\vspace*{\fill}
\begin{center}
#1
\end{center}
\vspace*{\fill}
\newpage
}


\newmdenv[
	middlelinewidth=2pt,
	backgroundcolor=yellow!10
]{note}



\newcommand{\idxCmd}[1]{\index{\texttt{\textbackslash#1}}}
\newcommand{\idxEnv}[1]{\index{\texttt{#1 \footnotesize{(environment)}}}}
\newcommand{\idxPkg}[1]{\index{\texttt{#1 \footnotesize{(package)}}}}


\newcommand{\quoteCmd}[1]{\texttt{\textbackslash#1}\idxCmd{#1}}
\newcommand{\quoteEnv}[1]{\texttt{#1}\idxEnv{#1}}
\newcommand{\quotePkg}[1]{\texttt{#1}\idxPkg{#1}}

%%% Terms with custom rendering

\newcommand{\LoremIpsum}{\emph{Lorem Ipsum}\index{Lorem Ipsum}}

%% \TeXstudio
\newcommand{\TeXstudio}{\TeX studio\index{TeXstudio}}
\newcommand{\WYSIWYG}{WYSIWYG\index{WYSIWYG}}
\newcommand{\WYSIWYM}{WYSIWYM\index{WYSIWYM}}

\newcommand{\Iomega}{Iomega\textregistered\index{Iomega\textregistered}}
\newcommand{\Zip}{Zip\texttrademark\index{Zip\texttrademark}}

%% Note the use of non-breaking spaces (~) and long dash (---)
\newcommand*{\longref}[1]{\ref{#1}~---~\nameref{#1}}


\newcommand{\TeXworks}{\emph{TeXworks}\index{TeXworks}}

\newcommand{\Microsoft}{\emph{Microsoft\textregistered}\index{Microsoft\textregistered}}
\newcommand{\Word}{\emph{Word\textregistered}\index{Word\textregistered}}


\usepackage{hologo}

\newcommand{\XeTeX}{\hologo{XeTeX}\index{XeTeX}}
\newcommand{\XeLaTeX}{\hologo{XeLaTeX}\index{XeLaTeX}}
\newcommand{\pdfTeX}{\hologo{pdfTeX}\index{pdfTeX}}
\newcommand{\pdfLaTeX}{\hologo{pdfLaTeX}\index{pdfLaTeX}}
\newcommand{\LuaTeX}{\hologo{LuaTeX}\index{LuaTeX}}
\newcommand{\LuaLaTeX}{\hologo{LuaLaTeX}\index{LuaLaTeX}}
\newcommand{\ConTeXt}{\hologo{ConTeXt}\index{ConTeXt}}
\newcommand{\MiKTeX}{\hologo{MiKTeX}\index{MiKTeX}}




%%%% Other include files. 
%%%% Their source code will also be featured in this document
%%%% We use the 'lstlisting' package to display some the source code.

%% Our function for adding furigana
% This source code is under the BSD License
% Copyright 2022 Pierre S. Caboche

\usepackage{ruby,tikz}
\newcommand{\furi}[1]{\foreach \kanji/\furigana in {#1}{\ruby{\kanji}{\furigana\vphantom{あ}}}}


%% Configuration for "ruby" package (loaded by 'files/furi')
\renewcommand{\rubysize}{0.4}     % default:  0.4
\newcommand{\defaultrubysep}{0ex} % default: -0.5ex
\renewcommand{\rubysep}{\defaultrubysep}




%opening
\title{ \LaTeX: rendering texts in Japanese and Chinese \\
(on Linux, Windows, and other~OSes)
}
\author{Pierre S. Caboche $\left(\ \furi{南瓜石/kabocha piēru}\ \right)$
	\footnote{The name Caboche (meaning ``head" in French) sounds similar to the Japanese word for ``pumpkin" (``kabocha" - 南瓜), while the given name Pierre (ピエール, \emph{piēru}) means ``stone" in French, so I use the character for ``stone" (石\ - \emph{ishi}).}
}

%% Custom format to indicate revision date (when needed)
%\date{
%	July 11, 2022 \\
%	\medskip 
%	\footnotesize \emph{Revised: \today}
%}


\begin{document}

\maketitle



\begin{abstract}
	This article tries to solve some common problems in \LaTeX\ when dealing with texts in Chinese, Japanese, and other languages that use a non-Latin writing system. \\
	
	The first problem we intend to solve is related to \emph{portability}: making sure that a \LaTeX\ document containing texts in Japanese, Chinese (or other languages) can still be rendered in \Windows, \Linux, or other systems (i.e. all our dependencies will have to be freely available for different OSes). 
	
	This is a prerequisite to the second problem we want to solve: adding \emph{ruby} characters, (e.g. \emph{furigana}, \emph{pinyin}) to texts in Japanese, Chinese (and other languages). \\
	
	However, this article needed to go beyond solving those two issues; and for a better understanding we also had to provide some background information about related subjects, including: the type of fonts we'll use (and where to find them), \LaTeX, \emph{ruby} characters, and even a few words about the Japanese and Chinese language. \\
	
	By the end of this article, we will see what it involves to add \emph{ruby} characters in \LibreOffice\ and \Microsoft\ \Word, and see how they compare to \LaTeX\ for this type of task.
\end{abstract}

\newpage


%%% Intro

% Copyright 2022 Pierre S. Caboche. All rights reserved.

\renewcommand{\currentPart}{Preamble}

\section*{Background}
I started to use \LaTeX\ to write documents containing a lot of Japanese text and \furigana\footnote{one of my hobbies is to study the lyrics of the Japanese songs I like, then try to sing them at the \emph{karaoke}. \LaTeX\ allows me to quickly add \furigana\ to the lyrics, or any other Japanese text}. From my experience (and by using some of the techniques described in this article), adding \furigana\ was considerably faster to do in \LaTeX\ than in either \Word\ or \LibreOffice, as we'll discover towards the end of this  article\dots \\


However, when I switched to \Linux, I discovered that my \LaTeX\ documents didn't render at all. \\

When I tried to look for a solution online, I found a lot of documents whose advice were either:
\begin{itemize}
	\item \emph{outdated}, as they relied on the obsolete packages % \texttt{CJKutf8} package under \pdfLaTeX
	\item \emph{not portable}: they were written with \Windows\ in mind, and recommended the use of fonts that are not readily available on other systems (e.g. \Meiryo)
\end{itemize}

I eventually found a solution to those issues, I decided to share my findings in this article.




\section*{Goal}

Our goal in this article is to learn how to perform the following:

\begin{itemize}
	\item in \LaTeX, display texts in Chinese, Japanese, etc. \\
	\phantom{MM}\dots without relying on proprietary fonts, which might not be available on \Linux
	\item add \rruby\ characters, especially Japanese \furigana\ (e.g. \kabocha) and Chinese \ppinyin\ (e.g. \xpinyin*{南瓜})
\end{itemize}



\section*{Methodology}

To achieve our goals, we will do the following:

\begin{itemize}
	\item install the \emph{\Noto\ Fonts} for the relevant languages
	\item install \LaTeX
	\item render documents containing texts in Chinese, Japanese, Korean, etc.\\
		\phantom{MM}\dots in a way that works on Windows, Linux, and other systems
	\item add \furigana\ to text in Japanese with the \texttt{ruby} package, as well as a \emph{custom macro}
	\item add \ppinyin\ to text in Mandarin Chinese with the \xxpinyin package
	\item perform the same tasks in \LibreOffice\ (mini-guide included), and compare it with our solution in \LaTeX
\end{itemize}

The rest of this article goes into more details about \emph{what} those tools are, \emph{how} to use them, and \emph{why}.



\newpage

\section*{What are \emph{ruby} and \emph{furigana}?}

\emph{Ruby} characters are annotations usually placed on top of\footnote{or to the right, if the text is displayed vertically} Chinese, Japanese, or Korean characters\footnote{\rruby\ characters can technically be used in other languages too}, which are usually used to show the pronunciation of such characters\footnote{\rruby\ characters have other usages, but are mainly used to indicate pronunciation}. \\


\CJKfontspec{Noto Sans CJK SC}

When adding \rruby\ characters to texts in Standard Mandarin Chinese, \ppinyin\ (see below) are usually used as \rruby. \\


Below is an example of \ppinyin\ used as \rruby:

\begin{center}
	\begin{pinyinscope}
	雪花飄飄\ 北風蕭蕭\par
	天地\ 一片蒼茫\par
	\end{pinyinscope}
\end{center}

\bigskip

\CJKfontspec{Noto Sans JP}

In Japanese, \rruby\ characters are usually called \furigana.\\

Below is the word ``\furigana" (振り仮名), with \furigana\ added to it:
\begin{center}
	\begin{tabular}{c c}
		\furi{振/ふ,り/,仮/が,名/な}    \\
		\furi{振/fu,り/ri,仮/ga,名/na} \\
	\end{tabular}
\end{center}


\bigskip


\emph{Ruby} characters may also be referred to as \emph{rubi}, and may be plurialised as: \emph{ruby}, \emph{rubi}, or \rubies\ (I tend to use the phrase ``\emph{ruby} characters" to avoid the confusion between singular and plural).





\section*{What is \ppinyin?}

\CJKfontspec{Noto Sans CJK SC}

\ppinyin\ is the official romanization system for Standard Mandarin Chinese. \\

The name ``\ppinyin" comes from \emph{``Hànyǔ Pīnyīn"} (汉语拼音), literally: \emph{``to spell the sound of the Han language"}. \\

\ppinyin\ can be used on their own (e.g. ``\pinyin{han4yu3pin1yin1}") or as \rruby\ characters (e.g. \xpinyin*{汉语拼音}). \\

\CJKfontspec{Noto Sans JP}


\section*{About foreign loanwords}

Words of Chinese or Japanese origins are invariable in English. For example, the plural of \emph{anime} or \emph{manga} is ``\emph{anime}", ``\emph{manga}". \\

As such, the word ``\kanji" may refer to either one \kanji\ (i.e. Chinese character, also used in Japanese) or several \kanji. \\

Foreign loanwords (therefore, invariable in plural) used in this article include: \kanji, \hiragana, \katakana, \furigana, \ppinyin.




\section*{Conventions}


Some commands in this article need to be executed with \emph{administrator} privileges (on Linux, that means \emph{root}) to perform operations such as: installing new software on the system, modifying some system configuration, etc. \\

If you have \emph{administrator} privileges on your machine, please read on\ldots

\note{
In this article, we indicate that a Linux command needs to be executed with \emph{root} privileges by using: \\

\cmd{sudo} \\

However, please note that there are cases when the \cmd{sudo} command will not work; for example, if the current user does not appear in the list of \cmd{sudoers} (file \cmd{/etc/sudoers}). \\

If that is your case, please use whichever method you normally use to run a command as \emph{root} (e.g. \emph{doas}, \emph{su}), while keeping in mind that being logged in as \emph{root} is extremely bad practice.
}


If you do NOT have \emph{administrator} privileges on your machine, please contact your administrator.




\newpage
\section*{General Information}

This document was first published at: \\
\mbox{} \hfill \url{https://pcaboche.github.io} 

\subsection*{Legal}
\input{"READ_ME_(LEGAL).txt"}


%%% TOC
\newpage
\renewcommand{\currentPart}{Table of Contents}

\addtocontents{toc}{\setcounter{tocdepth}{3}}
\setcounter{tocdepth}{3}
\tableofcontents

%%% Article body
\newpage
% Copyright 2022 Pierre S. Caboche. All rights reserved.

\part{Fonts} \label{fonts}

\renewcommand{\currentPart}{Fonts}


In order to properly render \LaTeX\ documents containing CJK characters (Chinese, Japanese, Korean), you will need to use fonts capable of displaying such characters\footnote{the same is true for other writing systems, each requiring specialised fonts.}.

Developing fonts that support CJK characters is very costly, and therefore such fonts may not be readily available across all Operating Systems. \\

For example, when switching to Linux, I didn't have access to the \Meiryo\ font anymore because it is © \MicrosoftCorp. So I had to find some portable alternative\dots

The \Noto\ font family \citep{noto-fonts} were designed to solve this problem. \\

The \Noto\ fonts are free (under the ``SIL Open Font License") and were commissioned by Google. Their goal is to cover a wide range of languages and writing scripts (including Chinese, Japanese, and Korean). Not all CJK ideographs are covered (30,000 of the nearly 75,000 CJK unified ideographs), but the most commonly used characters seem to be represented. \citep{wiki-noto} \\

If you were to install all the \Noto\ fonts in their entirety (i.e. for every language and script available), this would end up occupying more than 1.1 GB of disk space (as of time of writing).
To reduce space, it is recommended to install only the fonts that you intend to use.


\subsection*{Etymology}

\note{
	The name ``\Noto" stands for ``\textbf{no} more \textbf{to}fu". When a character cannot be rendered by a computer program, some of these programs (e.g. web browsers) show a substitute character instead (usually in the form of a small rectangle). \citep{wiki-noto} 
}

Those characters are sometimes colloquially referred to as ``\tofu", due to their resemblance with a block of tofu. Also, such substitute characters were quite likely to appear when trying to render texts from languages in regions like China, Japan, Korea\dots\footnote{if the slang had developed in another part of the world, then ``tofu" characters might be have known by some other name (and probably some other food-related item too). It isn't hard to imagine that in some parallel universe, these might be referred to as ``paneer" characters instead\dots
}\\

The goal of ``\Noto" is to eliminate those ``\tofu" characters (by properly rendering texts that use different writing systems).
\\

\subsection*{\Portability}

By switching to the \Noto\ fonts, your documents will look different (when compared to using the proprietary fonts available by default on some Operating Systems) but you will gain in portability.

Noto fonts are available at: \url{https://www.google.com/get/noto/} \\

\newpage

Later we'll see how to easily install the \Noto\ fonts on Linux, but first we'll need to determine which fonts are available, and which packages we'll need.

\bigskip

\subsection*{Noto fonts packages}

Using the package manager for your \linux\ distribution (e.g. \cmd{dnf}, \cmd{apt}, \cmd{pacman}, etc.), we'll look for the packages containing the word \textbf{noto}.\\

The list of matching results is very long (covering a large list of languages and scripts), so we'll filter even further (looking for terms like ``cjk" or ``japanese"). \\

For example, in Fedora:
\begin{lstlisting}[language=sh]
$ dnf search noto | grep -i -E 'cjk|japanese'
\end{lstlisting}

\bigskip
Searching for just ``japanese" will give us the following:
\begin{lstlisting}[language=prolog,keywordstyle=\color{red},otherkeywords={Japanese}]
$ dnf search noto | grep -i 'japanese'
google-noto-sans-cjk-jp-fonts.noarch : Japanese Multilingual Sans OTF font files for google-noto-cjk-fonts
google-noto-sans-jp-fonts.noarch : Japanese Region-specific Sans OTF font files for google-noto-cjk-fonts
google-noto-sans-mono-cjk-jp-fonts.noarch : Japanese Multilingual Sans Mono OTF font files for google-noto-cjk-fonts
google-noto-serif-cjk-jp-fonts.noarch : Japanese Multilingual Serif OTF font files for google-noto-cjk-fonts
google-noto-serif-jp-fonts.noarch : Japanese Region-specific Serif OTF font files for google-noto-cjk-fonts
\end{lstlisting}

\bigskip
Here is some explanation regarding the package names\dots


\subsection*{Typefaces}

When talking about CJK fonts, the terms \emph{\Serif} and \emph{\Sansserif} have the following meaning:
\begin{description}
	\item[Serif] \mbox{}\\ (roman)\\ means that the font will show the brush strokes
	\item[Sans serif] \mbox{}\\ (sans, sans-serif, gothic)\\
	means that brush strokes are not present
\end{description}

\bigskip

To better illustrate the difference, we'll create a \LaTeX\ document containing a few examples (at the moment we'll focus mainly on the document output. In later chapters we'll see the details of how the document works): \\

\newpage

\texttt{example-01-typefaces.tex}
\lstinputlisting[language=tex]{"files/example-01-typefaces.tex"}

\bigskip

\texttt{sample-text.tex}
\lstinputlisting[language=tex]{"files/sample-text.tex"}

\bigskip


This shows the differences between those typefaces\ldots

\newpage

\begin{mdframed}[frametitle={\colorbox{white}{Differences between \emph{Serif}, \emph{Sans-Serif}, and \emph{Monospace} }}]
	
	\begin{itemize}
		\item
			\textrm{
				Serif \emph{(Roman)}: \index{serif}
				% This source code is under the BSD License
% Copyright 2022 Pierre S. Caboche

\begin{itemize}  
	\item regular
	\item \emph{emphasis} (usually in italics)
	\item \textit{italic}
	\item \textsc{small caps}
	\item 日本語 (\emph{kanji}, Regular)
	\item \textbf{日本語} (\emph{kanji}, Bold)
	\item \textit{日本語} (\emph{kanji} can't be italicized)
	\item \textsc{日本語} (\emph{kanji} don't have lowercase/uppercase\dots)
	\item にほんご (\emph{hiragana})
	\item ニホンゴ (\emph{katakana})
	\item ニホンゴ (half-width \emph{katakana})
\end{itemize}

			}
		\item 
			\textsf{
				Sans Serif \emph{(Gothic)}: \index{sans-serif} % This source code is under the BSD License
% Copyright 2022 Pierre S. Caboche

\begin{itemize}  
	\item regular
	\item \emph{emphasis} (usually in italics)
	\item \textit{italic}
	\item \textsc{small caps}
	\item 日本語 (\emph{kanji}, Regular)
	\item \textbf{日本語} (\emph{kanji}, Bold)
	\item \textit{日本語} (\emph{kanji} can't be italicized)
	\item \textsc{日本語} (\emph{kanji} don't have lowercase/uppercase\dots)
	\item にほんご (\emph{hiragana})
	\item ニホンゴ (\emph{katakana})
	\item ニホンゴ (half-width \emph{katakana})
\end{itemize}

			}		
		\item 
			\texttt{
				Typewriter \emph{(\monospace)}: 
				% This source code is under the BSD License
% Copyright 2022 Pierre S. Caboche

\begin{itemize}  
	\item regular
	\item \emph{emphasis} (usually in italics)
	\item \textit{italic}
	\item \textsc{small caps}
	\item 日本語 (\emph{kanji}, Regular)
	\item \textbf{日本語} (\emph{kanji}, Bold)
	\item \textit{日本語} (\emph{kanji} can't be italicized)
	\item \textsc{日本語} (\emph{kanji} don't have lowercase/uppercase\dots)
	\item にほんご (\emph{hiragana})
	\item ニホンゴ (\emph{katakana})
	\item ニホンゴ (half-width \emph{katakana})
\end{itemize}

			}
	\end{itemize}
\end{mdframed}

\renewcommand{\familydefault}{\sfdefault}


\newpage
From the previous example, we can see that:

\note{
	\begin{itemize}
		\item a font with \emph{Serif} will show the brush strokes, and therefore will give a better idea of how a \emph{kanji} is to be drawn
		\item a \emph{Sans-serif} font, on the other hand, tends to be easier to read
	\end{itemize}
}

\bigskip
Characters in Japanese, Chinese, and Korean are of fixed width (\monospaced). \\

Japanese also has \halfwidth\ \kana, i.e. \katakana\ characters which are half the width of regular \kana, and are used only in certain context where display is limited in size. \citep{wiki-halfwidth} \\


Characters in Japanese, Chinese, and Korean cannot be put in italics, and are not subject to ``casing" (i.e. there is no distinction between lowercase and uppercase).

\bigskip

\subsection*{Writing systems}

Earlier on, we saw that Fedora Linux provided packages with names like:
\begin{lstlisting}[language=prolog,keywordstyle=\itshape\color{blue},otherkeywords={-cjk,-jp-}]
google-noto-sans-cjk-jp-fonts.noarch
google-noto-sans-jp-fonts.noarch
google-noto-sans-mono-cjk-jp-fonts.noarch
google-noto-serif-cjk-jp-fonts.noarch
google-noto-serif-jp-fonts.noarch
...
\end{lstlisting}

\medskip

We already know that:
\begin{description}  
	\item[\emph{google-noto-} \dots\ \emph{-fonts}] \mbox{}\\
	represent the \emph{Google Noto} font families
	\item[\emph{sans-}, \emph{serif-}, \emph{mono-}] \mbox{}\\ are the different typefaces available
\end{description}

\bigskip

The remaining part of the package name corresponds to the font target (in terms of language, script, use, etc.) \\

\newpage

Below are some examples:
\begin{description}  
	\item[\emph{jp}] \mbox{}\\
	Japanese  
	\item[\emph{kr}] \mbox{}\\
	Korean
	\item[\emph{sc}] \mbox{}\\
	Simplified Chinese
	\item[\emph{tc}] \mbox{}\\
	Traditional Chinese
	\item[\emph{hk}] \mbox{}\\
	Traditional Chinese Region-specific
	\item[\emph{cjk-jp}, \emph{cjk-kr}, \emph{cjk-sc}, \emph{cjk-tc}, \emph{cjk-hk}] \mbox{}\\
	Multilingual (Chinese, Japanese, Korean) versions of the above
	\item[\emph{myanmar}] \mbox{}\\
	Myanmar
	\item[\emph{myanmar-ui}] \mbox{}\\
	Myanmar UI font (i.e. targeted towards apps and software user interfaces)
	\item[\emph{myanmar-vf}] \mbox{}\\
	Myanmar variable font
	\item[\emph{myanmar-vf-ui}] \mbox{}\\
	Myanmar UI variable font 
	\item[\dots]
\end{description}






\newpage

\section*{Installation}

After going through the list of available packages, we need to choose the ones we'll need and install them. 

\subsection*{Windows}

For Windows, you need to:
\begin{itemize}
	\item go to the Noto Fonts website ( \url{https://www.google.com/get/noto/} )
	\item select and download the fonts you need
	\item install the fonts on your system
\end{itemize}


\subsection*{Linux}

Installing new packages will require the \emph{super admin} privileges. \

Example, in Fedora:
\begin{lstlisting}[language=sh]
sudo dnf install \
	google-noto-sans-cjk-jp-fonts \
	google-noto-serif-cjk-jp-fonts
\end{lstlisting}

In this example, we installed both \emph{Serif} and \emph{Sans-serif} typefaces of the CJK Japanese fonts. \\


In Fedora, the \emph{Google \Noto} fonts will be installed in folders matching this pattern:
\cmd{/usr/share/fonts/google-noto*}

\bigskip

\section*{Other fonts}

In this article, we use \emph{\Noto\ Fonts} because they are free, portable, and cover a variety of writing systems. \\

Other fonts of your choosing can be used in your documents. \XeTeX\ should be able to use any fonts installed in your Operating System. \\


\newpage
% Copyright 2022 Pierre S. Caboche. All rights reserved.

\part{\LaTeX} \label{latex}

\renewcommand{\currentPart}{\LaTeX}

In the previous part, we talked about the fonts necessary to render CJK characters, and where to find them. \\

Next we'll need a working \LaTeX\ environment\dots

\section*{Differences between \TeX, \LaTeX, and others}

The original \TeX\ was created in the late 1970s by Donald Knuth, who needed a new typesetting program. 

\emph{At that time, Knuth was revising the second volume of his book \emph{``The Art of Computer Programming"},  got the galleys from his publisher, and was very disappointed in the result. The quality was so far below that of the first edition that he couldn't stand it. Around the same time, he saw a new book that had been produced digitally, and thought he could produce a digital typesetting system. So he started to learn about typography, type design, and the rules for typesetting math} \citep{tug}\footnote{I highly recommend you look at \cite{tug} if you want to learn more about the history of \TeX}, and thus started his work on \TeX.

\note{
The idea behind \TeX\ was \emph{``to allow anybody to produce high-quality books with minimal effort, and to provide a system that would give exactly the same results on all computers, at any point in time"} \citep{wiki-tex}
}

The commands in \TeX\ were basic, but allowed the creation of macros to extend the list of commands. \\

In the early 1980s, Leslie Lamport created \LaTeX, a typesetting program written in the \TeX\ macro language. \citep{wiki-latex} As such, \LaTeX\ provides a large set of macros for \TeX\ to interpret, and \TeX\ is in charge of formatting the output.


\LaTeX\ packages are centralised in a repository called ``The Comprehensive \TeX\ Archive Network" (CTAN), \emph{``the central place for all kinds of material around \TeX"} \citep{CTAN}.

\note{Broadly speaking, you can think of \LaTeX\ as: \emph{``\TeX, enhanced with a huge collection of macros: more than 6000 packages to date in \cite{CTAN}."}}


In 1989, Knuth declared that \TeX\ was feature-complete, and only bug fixes would be made \citep{tex-vs-latex}. Since then, new typesetting programs based on \TeX\ appeared: \pdfTeX, \XeTeX, \LuaTeX\dots\\

When those typesetting programs are used in conjunction with the \LaTeX\ macros, we talk of:  \pdfLaTeX, \XeLaTeX, \LuaLaTeX\dots \\

The advantage of \XeTeX\ (and therefore \XeLaTeX) is that:
\begin{itemize}
	\item \XeTeX\ supports UTF-8 by default
	\item \XeTeX\ can make use of the fonts that are installed on your computer (not just the standard \LaTeX\ fonts)
\end{itemize}

This is required to handle texts in Japanese, Chinese, or other languages. We'll use \XeLaTeX\ to generate our documents. \\



\section*{Getting \LaTeX}

The official \LaTeX\ Project website \citep{latex-project} provides some information about \LaTeX\ (including how to install \LaTeX). \\

Link: \url{https://www.latex-project.org} \\

That being said, here is some information to get you started\dots


\subsection*{\Windows}

On \Windows, I would recommend using \MiKTeX, which includes (among other things):

\begin{itemize}
	\item an \emph{``integrated package manager"} \citep{miktex-project} (which will help you download the missing \TeX\ packages, as you need them. This allows you to keep \emph{``just enough \TeX"} on your computer for your work)
	\item \TeXworks, a \emph{``simple \TeX\ front-end program (working environment)"} \citep{texworks}
\end{itemize}

When it comes to \TeX\ editors, I have a preference for \TeXstudio.
The goal of \TeXstudio\ is to \emph{``make writing \LaTeX\ as easy and comfortable as possible"} \citep{texstudio}. \\

On Windows, I would usually install the following:
\begin{itemize}
	\item \MiKTeX:   \url{https://miktex.org}
	\item \TeXstudio: \url{https://www.texstudio.org} 
\end{itemize}


\subsection*{\Linux}

On \Linux, I normally install \texttt{texlive}  
(as most Linux distribution provide it through their official repositories \citep{texlive}), as well as an editor for \TeX\ (usually \TeXstudio).

\begin{description}
	\item[texlive] \mbox{} \\ 
	\emph{Required}.
	\TeX\ formatting system.
	
	\item[texstudio] \mbox{} \\
	\emph{Recommended}.
	A feature-rich editor for \LaTeX\ documents. \\
\end{description}


To install these packages in Fedora (requires \emph{super user} privileges):
\begin{lstlisting}[language=sh]
$ sudo dnf install texlive texstudio
\end{lstlisting}

\bigskip

Additional \TeX\ packages need to be installed separately. \\

Such packages normally have a name that starts with "\texttt{texlive-}" \\
( e.g. \texttt{texlive-mdframed} )

\bigskip


\section*{Installing fonts}

This is a reminder (from the previous part) that we need to install some specialised fonts to be able to render CJK characters. \

In the previous part, we also explained how to choose which fonts to install (based on our needs), and how to install them. \\

Please see \emph{``\nameref{fonts}"} for more details. \\


\section*{Additional packages}

Some of our examples require additional packages. \\

My advice is to try and render the documents we provide as examples. If you come across any error, then install the missing packages. This will save you some disk space.

\subsection*{Windows}

If you're using \MiKTeX, then it will automatically handle package dependencies (\MiKTeX\ will ask you permission before downloading the missing packages for you). \\

This allows to keep \emph{``just enough TeX"} (i.e. only install the necessary packages).
The downside is, the \MiKTeX\ repositories can be significantly slower than those of major Linux distributions.


\subsection*{Linux}

Under Linux, you will need to install the necessary \LaTeX\ packages (installation requires \emph{super user} privileges):

\begin{description}
	\item[xeCJK] \mbox{} \\ 
	\emph{Required} \\
	Support for CJK documents in \XeLaTeX. \citep{xecjk} \\
	Package name (Fedora): \texttt{texlive-xecjk}

	\item[xpinyin] \mbox{} \\
	\emph{Optional} \\
	Automatically add pinyin to Chinese characters. \citep{xpinyin} \\
	Package name (Fedora): \texttt{texlive-xpinyin}
\end{description}

\xxpinyin\ is a really impressive package! It contains a large database of Chinese characters and their pronunciation. \\


\newpage

\section*{Documentation for \xeCJK\ and \xxpinyin}

To this day, the documentation for \xeCJK\ \citep{xecjk} and \xxpinyin\ \citep{xpinyin} is available in Chinese only. \\

While ``CJK" stands for \emph{``Chinese, Japanese, Korean"}, only Chinese speakers have access to the documentation (which is very technical).

The \xxpinyin\ documentation is a bit easier to follow, despite the language barrier (it is reasonable to assume that \xxpinyin\ might be used by learners of Mandarin Chinese, who wish to quickly add \emph{pinyin} to the texts they want to study). \\

Finding a document covering all the steps for rendering CJK characters in \LaTeX\ (including fonts, package installation and use) was hard. Hopefully this article will help you getting set up. \\

I am focusing primarily on Japanese and Chinese scripts. That being said, some of the information contained in this document may apply to other writing systems. \\


\newpage

\section*{Using CJK fonts in \LaTeX}

To manage the CJK fonts, we'll need to use the \cmd{xeCJK} package:
\begin{lstlisting}[language=tex]
\usepackage{xeCJK}
\end{lstlisting}

We can set the CJK fonts to be used for the \emph{whole} document. This must be done in the \emph{preamble}:
\begin{lstlisting}[language=tex]
\setCJKmainfont{Noto Serif JP}
\setCJKsansfont{Noto Sans JP}
\setCJKmonofont{Noto Sans Mono CJK JP}
\end{lstlisting}

This way, when \lstinline|\sffamily| is used (e.g. inside a \lstinline|\textsf{}| block), \LaTeX\ will automatically render CJK characters with the \emph{Noto Sans JP} font.\\



It is also possible to manually switch to a different CJK font:
\begin{lstlisting}[language=tex]
\CJKfontspec{Noto Sans JP}
\end{lstlisting}

\bigskip

\section*{Finding the font names}

In our document, we'll need to tell \LaTeX\ which fonts we want to use. This implies referring to the font by name. \\

To find the font name, we have several solutions:
\begin{enumerate}
	\item explore the directory where the \emph{Noto fonts} are installed \newline
	( e.g. \cmd{usr/share/fonts/google-noto-cjk} ), open the font with a program like \cmd{gnome-font-viewer}, and copy the font name
	\item open \emph{LibreOffice} and view the list of available fonts, sorted by name
\end{enumerate}

I know, it is rather ironic to use \LibreOffice\ in order to find the font name we wish to use in a \LaTeX\ document, but in many cases it might be more convenient this way.





\newpage
\input{"content/part-japanese"}

\newpage
\input{"content/part-chinese"}

\newpage
\input{"content/part-conclusion"}


%%% Bibliography
\pagebreak


\renewcommand{\formatPartTitle}{}

%\renewcommand{\currentPart}{References}
%\part{References}

% Bibliography title can be customised (default: "\section*{\currentPart}")
%\renewcommand{\bibsection}{\section*{\currentPart}}
%\renewcommand{\bibsection}{}

\bibliography{content/biblio-cjk}
\bibliographystyle{apalike}


\printindex

% Label on the last page. Allows to easily get the page number
\label{LastPage}

\end{document}
