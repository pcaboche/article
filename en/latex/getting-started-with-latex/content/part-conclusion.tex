%%%%
%% Copyright 2022 Pierre S. Caboche
%% All rights reserved
%%%%


\section{Conclusion}

This article summarises what I have learned about \LaTeX\ over the past few months, while working on some articles. \\

It all started with a few notes, some tips and tricks about \LaTeX\ that I wanted to remember, and then grew into something much bigger: a guide for people who have never used \LaTeX\ before to be able to write documents or work on their thesis. \\

Interestingly, the writing of this document closely followed the steps described in it\dots\ It starts with opening a template (either a relatively blank template provided by a \TeX\ editor like \TeXstudio, or a standard template that is adopted by the members of the same school, university, organisation\dots) and saving this template in a dedicated folder, where you will store all the files needed for generating the document. \\

Then you start writing down the subjects you want to talk about, and organising them in parts, sections, sub-sections\dots\ So you add a Table-of-Contents to have an overview of the overall structure of your documents. \\

Now that you have some basic plan in place, you start filling out the different parts with ideas. Sometimes the idea is clear enough and a new sentence appears in your document, other times the idea is a bit fuzzy, so you write it down in the form of a comment. \\

Comments are an extremely valuable tool in \LaTeX. They really help in the writing process! \\

\emph{Need to write down an idea that you might want to develop later?} Write down as a comment! \emph{Found an interesting article that you need to explore later?} Write down the reference as a comment! \emph{Not completely satisfied with the phrasing of a certain paragraph?} Comment it out, keep several versions in comments, compare those versions, and eventually you will find the right way to express your idea. \emph{Need to write a note to the people you're collaborating with on the document?} Write a comment!

Comments are fantastic in \LaTeX, and you can put them anywhere. Other text editors only allow you to add comments in a small box in the margin; this is very limiting. It doesn't come close to the power of \LaTeX\ comments and Ctrl+T! (toggle comments on or off, on the selected lines in \TeXstudio) \\

Now that ideas start flowing, and the document starts growing, you organise the content in multiple files of manageable size, which you store in separate folders depending on the file's purpose. \\

And now you start to spot recurring terms, or words that need to appear in the same, consistent format. So you start writing custom \emph{macros} for that, and you also consider adding an index to your document.

Then it's not only recurring terms, but every repeating patterns that you turn into macros, which you can call at will. And it makes your life easier. \\

\newpage

Now that you have a satisfying amount of content, well organised, with lots of cross-references, you want to experiment with the style of the document (unless someone provided you with a template that you need to follow, in that case your document already meets a certain standard of beauty). So you start modifying things like fonts, margin sizes, and probably add a header and footer to the document. 

In any case, it is better to focus on the content of the document first, and modify the style much later. If you want to change the style too early, and in the absence of content, you will have to use some placeholder text (e.g. \LoremIpsum), images, etc. which do not reflect the true content of your document.

So you modify your template much later, when you have at least a dozen pages ready.
\\

And this is how this document came to be! \\

I told you in the abstract that this document was describing how it was written. Although I was initially talking about the different techniques used to generate a document in \LaTeX, in return this also applies to the writing process itself. \\

In any case, I hope you found this article useful, that it helped you on your journey with \LaTeX, or maybe convinced you to try out \LaTeX.