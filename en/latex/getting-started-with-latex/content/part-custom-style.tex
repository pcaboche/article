%%%%
%% Copyright 2022 Pierre S. Caboche
%% All rights reserved
%%%%

\part{Customising the style}

So far we've focused mainly the features to help you write the \emph{content} of the document (e.g. organising the \LaTeX\ files, writing paragraphs, adding notes, a Table of Contents, managing an index, cross-references, a bibliography\dots) and writing macros to help you with repetitive tasks. \\

All these are very important to write ideas that flow naturally while being inter-connected with each other. So we've focalised on writing those ideas, and trusted \LaTeX\ to produce good-looking documents for us, using its default style. \\

In this part, we'll finally learn how to add all the bells and whistles, and tweak the appearance of our document to our liking\dots



\section{Making the document two-sided} \label{two-sided}

Making a document two-sided is useful if that document is destined to be printed, and its pages are meant bound together. \\

This means that the style needs to change between even and odd pages: the inner margin (closer to the binding) needs to be wider, margin notes (section~\ref{margin-notes}) need to be switched between left and right (to remain in the outer margin), etc. \\

To make a document two-sided, you need to specify the option \texttt{twoside} in \quoteCmd{documentclass}:
\begin{lstlisting}[language=tex]
\documentclass[twoside]{article}
\end{lstlisting}


\section{Changing the margins}

Changing the margins is done with the \quotePkg{geometry} package:

\lstinputlisting[language=tex]{include/featured/change-margins.tex} 

With the \quoteCmd{geometry} command, we specify things like the paper size (\texttt{a4paper}), as well as the \texttt{top} and \texttt{bottom} margins (which are self-explanatory).

Similarly, we could also specify the \texttt{left} and \texttt{right} margins, but instead we set the \texttt{inner} and \texttt{outer} margins. \\

In the case of a two-sided document (section \ref{two-sided}) used for printing and binding, \texttt{inner} refers to the margin closer to the binding, while \texttt{outer} refers to the margin that is away from it. If the document is meant to be printed and bound, then the \texttt{inner} margin should be slightly bigger than the \texttt{outer} one (to leave enough room for the binding). \\

In the case of a one-sided document (i.e. not meant for printing and binding), \texttt{inner} and \texttt{outer} simply refer to \texttt{left} and \texttt{right} margins respectively. \\

As we have modified the margin sizes, then we must also adjust the \\
\texttt{marginparwidth} size, otherwise the ``\nameref{margin-notes}" (section~\ref{margin-notes}) might not fit in the margin anymore.

If you're not using margin notes or margin paragraphs, there is no need to specify \texttt{marginparwidth}. Otherwise, \texttt{marginparwidth} should be smaller than the value for the \texttt{outer} margin. \\



\section{Changing the fonts} \label{changing-the-fonts}


Changing the fonts for our document is done with the \quotePkg{fontspec} package:

\lstinputlisting[language=tex]{include/featured/change-fonts.tex}

\medskip

Here, we select one font per typeface:
\begin{description}
	\item[\textrm{Roman}]  \mbox{}\\ 
		\emph{a.k.a \textrm{Serif}}  \\
		A font that contains \emph{serif}, the small lines or strokes at the end of larger strokes or characters.
		You can set the \emph{Serif} font using \\ \quoteCmd{setromanfont}
		
	\item[\textsf{Sans}] \mbox{}\\ 
		\emph{a.k.a \textsf{Sans-Serif, Gothic}}  \\
		A font that contains \emph{no} serif.
		You can set the \emph{Sans-Serif} font using \\ 
		\quoteCmd{setsansfont}
	
	\item[\texttt{Monospaced}]  \mbox{}\\ 
		\emph{a.k.a \texttt{Typewriter}} \\
		A font in which all characters are of equal horizontal space. 
		This is important for showing computer code, input files, CSV files, etc. (any application where we need to easily count characters, and see how each character would align in a grid). 
		You can set the \emph{Monospaced} font using \\ 
		\quoteCmd{setmonofont}
		
\end{description}

Finally, we set the main font for the document with the  \quoteCmd{setmainfont} command. \\

Fonts with \emph{serif} look aesthetically pleasing on printed documents, while \emph{sans serif} fonts tend to be easier to read on screens\footnote{most modern printers have a print resolution in excess of 600 Dots-Per-Inch; for comparison, an iPhone 13 has a screen density of 460 Pixel-Per-Inch}.
\\


This is why I prefer to use \emph{sans serif} fonts for documents like this one, meant to be read on-line rather than being printed. \\


In most \LaTeX\ document classes, the default style usually uses a \emph{serif} font as the main font, specifically because \emph{serif} fonts look better in print. \\

For this reason, calling the \quoteCmd{setromanfont} command will also reset the main font for the document.

That's why it is advisable to call \quoteCmd{setmainfont} only \emph{after} \quoteCmd{setromanfont}. \\

In our example, we set both the \emph{serif} and \emph{sans-serif} fonts, but keep the default \emph{monospaced} provided by \LaTeX. \\
 



\section{Customising headers and footers} \label{headers-footers}

To define our headers and footers, we will use the \quotePkg{titleps} package, which I find much easier to use than \quotePkg{fancyhdr}. \\

For this section, we will take a look at how headers and footers were defined for this article, and analyse the example:

\lstinputlisting[language=tex]{include/featured/pkg-titleps.tex}

\bigskip

The configuration of headers and footers is done in the preamble of the document. The first thing we do is use the \quotePkg{titleps} package. \\

Then we use the \quotePkg{xcolor} package because we want to use the colour \texttt{lightgray} in our footer (note: we will need the \texttt{[table]} option in section \emph{\longref{tabular-features}}) . \\

Then we define the \emph{page style} ``main" with the \quoteCmd{newpagestyle} command. \\

In \quoteCmd{newpagestyle}, we define our headers and setters, using \quoteCmd{sethead} and \quoteCmd{setfoot}. These commands have the following signature:

\begin{description}
	\setlength{\itemsep}{-0.5em}
	
	\item[\quoteCmd{sethead}\texttt{[\emph{evenLeft}][\emph{evenCenter}][\emph{evenRight}]\{\emph{left}\}\{\emph{center}\}\{\emph{right}\}}]
	\item[\quoteCmd{setfoot}\texttt{[\emph{evenLeft}][\emph{evenCenter}][\emph{evenRight}]\{\emph{left}\}\{\emph{center}\}\{\emph{right}\}}] \mbox{} \\ 
	\mbox{} \\ 
	These 2 commands each have:
	\begin{itemize}
		\setlength{\itemsep}{-0.2em}
		\item 3 mandatory parameters: \texttt{\emph{left}}, \texttt{\emph{center}}, \texttt{\emph{right}}
		\item 3 optional parameters: \texttt{\emph{evenLeft}}, \texttt{\emph{evenCenter}}, \texttt{\emph{evenRight}}
	\end{itemize}
\end{description}

Headers and footers are both divided in 3 parts: \texttt{\emph{left}}, \texttt{\emph{center}}, and \texttt{\emph{right}}.
The mandatory parameters apply to both \emph{one-sided} and \emph{two-sided} documents (section \ref{two-sided}):
\begin{itemize}
	\item if the document is \emph{one-sided}, then \texttt{\emph{left}}, \texttt{\emph{center}}, and \texttt{\emph{right}} define the text to be displayed on the \texttt{\emph{left}}/\texttt{\emph{center}}/\texttt{\emph{right}} part of the header/footer \emph{for all pages}
	
	\item if the document is \emph{two-sided}, then:
	\begin{itemize}
		\item \texttt{\emph{left}}, \texttt{\emph{center}}, and \texttt{\emph{right}} only apply to \emph{odd} pages
		\item \texttt{\emph{evenLeft}}, \texttt{\emph{evenCenter}}, and \texttt{\emph{evenRight}} only apply to \emph{even} pages
	\end{itemize}
\end{itemize}



\bigskip

In our headers/footers, we use the following commands: \\

\begin{tabular}{l l}
	\quoteCmd{thesection}   & prints the section number \\
	\quoteCmd{sectiontitle} & prints the section title \\ 
	\quoteCmd{thepage}      & prints the page number \\
\end{tabular}

The commands \quoteCmd{headrule} and \quoteCmd{footrule} add a rule, respectively: below the headline (\quoteCmd{headrule}), and above the footline (\quoteCmd{footrule}).

\bigskip

Now that the ``main" \emph{page style} defined, we set the \emph{page style} to ``main" using the \quoteCmd{pagestyle} command. \\

Finally, we use the \quoteCmd{widenhead} to make the headlines/footlines wider. Here is the syntax:
\begin{description}
	\setlength{\itemsep}{-0.5em}
	\item[\quoteCmd{widenhead}\texttt{[\emph{evenLeft}][\emph{evenRight}]\{\emph{left}\}\{\emph{right}\}}]
\end{description}

\bigskip

We could have simplified our header/footer definitions by using \quoteCmd{sethead*}, \quoteCmd{setfoot*}, and \quoteCmd{widenhead*} instead of \quoteCmd{sethead}, \quoteCmd{setfoot}, and \quoteCmd{widenhead}.

The \emph{starred} version of these commands do not take the optional parameters; instead, for \emph{even} pages, they use the same values as \texttt{\emph{left}}, \texttt{\emph{center}}, and \texttt{\emph{right}}, but flipped:

\begin{description}
	\setlength{\itemsep}{-0.5em}
	
	\item[\quoteCmd{sethead*}\texttt{\{\emph{left/evenRight}\}\{\emph{center}\}\{\emph{right/evenLeft}\}}]
	\item[\quoteCmd{setfoot*}\texttt{\{\emph{left/evenRight}\}\{\emph{center}\}\{\emph{right/evenLeft}\}}]
	\item[\quoteCmd{widenhead*}\texttt{\{\emph{left/evenRight}\}\{\emph{right/evenLeft}\}}]
\end{description}


I used the \emph{non-starred} version of these commands because they better illustrate how to define headers and footers (and the distinction better \emph{even} and \emph{odd} pages in the case of \emph{two-sided} documents).
