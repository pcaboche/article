%%%%
%% Copyright 2022 Pierre S. Caboche
%% All rights reserved
%%%%

\part{Tables}

In this part, we are going to give a quick introduction to ``tables" in \LaTeX. \\

There are many packages which allow to makes tables, each with different sets of options and features. This is a vast subject. \\

Please refer to the packages' respective documentation for more information.


\section{\texttt{tabular}}

The \quoteEnv{tabular} environment is used to create tables and other kinds of grids. \\

Here is an example:
\begin{table}[h]
	\caption{Tic, Tac,Toe}
	\medskip
	\centering
\begin{tabular}{c | c | c}
X & O & X \\
\hline
O & X & X \\
\hline
X & O & O \\
\end{tabular}

\end{table}

Now let's take a look at the code for that grid:
\lstinputlisting[language=tex]{include/featured/example-tabular-tic-tac-toe.tex}

\bigskip

In the \quoteEnv{tabular} environment, we specify the alignment and separators for the cells (\texttt{\{c | c | c\}}).

Here are \emph{some} values for the alignment (list non-exhaustive): \\
\begin{center}
	\begin{tabular}{l l}
		\texttt{l} & align left \\
		\texttt{c} & center \\
		\texttt{r} & align right \\
		\texttt{p\{\textit{width}\}} & makes the cells a \quoteCmd{parbox} of a certain \texttt{\textit{width}} \\
		\texttt{|} & a vertical line, to separate columns  \\
		\texttt{||} & a double vertical line, to separate columns \\
	\end{tabular}
\end{center}

\bigskip
An horizontal line is drawn with the \quoteCmd{hline} command. \\

Finally, we need to specify the content of the grid: \\
Each cell is separated by an \texttt{\&}, and each row is terminated by a \lstinline|\\|. \\

These are the basics for using a \quoteEnv{tabular}.

\bigskip
\bigskip



\subsection{Tabular features} \label{tabular-features}

Now we'll look at a more complex example that shows some of the features of the \quoteEnv{tabular} environment\dots\ 
Besides \quotePkg{tabular}, this example also requires the \quotePkg{xcolor} package. \\
\begin{table}[h]
	\caption{A more complex tabular}
	\medskip
	\centering
\begin{tabular}{| l | c  c | l  c  c  c  c || c  c  c  c  c  r |}
\hline
A1 & 2 & 3 & 4 & 5 & 6 & 7  & 8 & 9 & 10 & 11 & 12 & 13 & 14 \\
\hline
B & & & & & & & & & & & & & \\
\cline{3-6} \cline{8-9} \cline{11-13}
C & 
& \multicolumn{4}{|c|}{4 cols} 
&
& \multicolumn{2}{|c|}{2 cols} 
&
& \multicolumn{3}{|c|}{3 cols} &
\\
\cline{2-14}
D 
& \multirow{9}{*}{\rotatebox{90}{9 rows}} 
& \multirow{4}{*}{\rotatebox{90}{4 rows}} 
& & & & & & & & & & &
\\
E & & & & 
\multicolumn{3}{c}{\multirow{2}{*}{\cellcolor{blue!25}text}}
& & &
\multicolumn{3}{c}{\multirow{2}{*}{\cellcolor{blue!25}text}}
& & 
\\
F & & & & & & & & & \multicolumn{3}{c}{\cellcolor{blue!25}} & &
\\
G & & 
\\
\cline{3-3}
H & & \multirow{5}{*}{\rotatebox{90}{5 rows}} 
& & & & & & & \cellcolor{blue!25} & & & &
\\
I  & & & & \multicolumn{3}{c}{\cellcolor{blue!25}} 
& & & & & & &
\\
J  & & & & \multicolumn{3}{c}{\multirow{-2}{*}{\cellcolor{blue!25}text}}
& & & & & & &
\\
K & & & & & & & & & & & & &
\\
L & & & & & & & & & & & & & x
\\
\hline
\end{tabular}
	\label{table:complex}
\end{table}


In table \ref{table:complex}, we assigned a letter to each row, and we number each column. This makes it easier to understand certain commands (i.e. \quoteCmd{cline}, which takes column numbers as a parameter). \\

Now let's take a look at the code:
\lstinputlisting[language=tex]{include/featured/example-tabular-complex.tex}

\bigskip


	
In the \quoteEnv{tabular} environment, we define our column list and their alignment.  \\

Column separators are specified with a \texttt{|} in the list of columns. Notice how on row G the vertical lines are broken (including the right border). This is because we didn't specify enough \texttt{\&}. If the cell is not specified (even as an empty cell), then the column separators won't appear. \\

The \quoteCmd{hline} command draws a line that spreads the whole width of the table. \\

For finer control, we use the \quoteCmd{cline} command. For example, \lstinline[language=tex]|\cline{3-6}| draws a line from column 3 to column 6 (both included). \\


We notice a number of cells that occupy more than one column.
This is done using the \quoteCmd{multicolumn} command. \\

For example in C3-6, we do:
\begin{lstlisting}[language=tex]
\multicolumn{4}{|c|}{4 cols}
\end{lstlisting}
This creates a cell that spans 4 columns, with its content centered and cell separators on both sides (\texttt{|c|}).

\begin{note}
	Note that a \quoteCmd{multicolumn} may also ``span" only 1 column:
	\begin{center}
		\texttt{\textbackslash multicolumn\{1\}\{c\}\{text\}}
	\end{center}
	This is mainly used to change the alignment or the borders for that particular cell (e.g. make the cell content \emph{centered} in an otherwise \emph{left}-aligned column), in table headers for instance.
\end{note}

Now that we spanned a cell over multiple columns, we're going to do the same over multiple rows. For that, we will need the \quotePkg{multirow} package:
\lstinputlisting[language=tex]{include/featured/pkg-multirow.tex}

This package provides us with the \quoteCmd{multirow} command \\

We will also need to rotate the content of the cell to make it vertical. This is done using the \quoteCmd{rotatebox}, which is part of the \quotePkg{graphics} package (note: this might already be loaded if you are using a package like \quotePkg{pgfplots}, see section~\ref{plots}).
\lstinputlisting[language=tex]{include/featured/pkg-graphics.tex}

In D3, we define a cell that spans 4 rows, and rotate the text 90\textdegree\ counterclockwise:

\begin{lstlisting}[language=tex]
\multirow{4}{*}{\rotatebox{90}{text}}
\end{lstlisting}

Next in E5 (and other places), we define a cell that spans multiple columns \emph{and} multiple rows:
\begin{lstlisting}[language=tex]
\multicolumn{3}{c}{\multirow{2}{*}{text}}
\end{lstlisting}

As we'll see, \quoteCmd{multirow} is harder to work with than \quoteCmd{multicolumn}\dots

\bigskip

Next, we change the colour of a cell by using the \quoteCmd{cellcolor} command.
To use \quoteCmd{cellcolor}, we will need to call the \quotePkg{xcolor} package with the \texttt{table} option:
\begin{lstlisting}[language=tex]
%%% In the preamble
\usepackage[table]{xcolor}
\end{lstlisting}

We paint cell H10 with the colour \texttt{blue!25} (i.e. ``blue, 25\% saturation") by doing this:
\begin{lstlisting}[language=tex]
\cellcolor{blue!25}
\end{lstlisting}

Then we try to colour cells that span multiple columns/rows, with various degree of success\dots \\

\quoteCmd{cellcolor} works very well with \quoteCmd{multicolumn}, but not \quoteCmd{multirow}:
\begin{itemize}
	\item as you can see in cell E5-F7, only the first row (E5-7) gets painted
	\item in E10-F12, we try to paint the second row (F10-12), but then the text gets ``eaten" away
	\item in I5-J7, we painted the first row (I5-7), then made a  \quoteCmd{multirow} on the second line (J5-7) with a \emph{negative} number of rows (so the text gets printed \emph{over} the previous row). It's a really ugly workaround, though\dots
\end{itemize}

\bigskip

So as you can see, working with \quoteCmd{multirow} (and some other advanced features) is not always easy. \\


\begin{note}
As with many things in \LaTeX, it's often best to keep your tables simple, and focus on what's important: deliver clear and impactful ideas without relying on complicated features.
\end{note}


\bigskip


\subsection{Paragraphs a inside table}

In the next sample, we will create a table with cells that can accept paragraphs as their content:

\begin{table}[h]
	\caption{Paragraphs inside a tabular}
	\label{table:with-paragraphs}
	\centering
	\begin{tabular}[t]{| l | c | p{3cm} || p{5cm} |}
\hline
A1 
& \multicolumn{1}{|c|}{2}
& \multicolumn{1}{c||}{3}
& \multicolumn{1}{c|}{4}
\\	

\hline
B 
& \multirow{12}{*}{\rotatebox{90}{Multiple rows\dots}}
&
&
This cell can contain a whole paragraph! \newline
\newline
If there is a need to add a newline, use 
the \texttt{\textbackslash newline} command.
\\

C 
& 
& 
\cellcolor{blue!25} 
As you can see, we can colour the cell very easily!
&
\\
\hline
\end{tabular}

\end{table}


Here is the code:
\lstinputlisting[language=tex]{include/featured/example-tabular-paragraphs.tex}

\bigskip

As you can see in table \ref{table:with-paragraphs}:

\begin{itemize}
	\item in the column definition, we use \texttt{p\{\textit{width}\}} (e.g. \texttt{p\{3cm\}}). \\
	This forces you to specify the column's width, but allows you to put long texts in your table. 
	
	\item to add a new line, use the \quoteCmd{newline} command (\lstinline|\\| will not work)
	
	\item we didn't need a \quoteCmd{multirow} for cell C3, so colouring it was easy (unlike in table \ref{table:complex}, where we had to use some workarounds).
	
	\item for cells B2-C2, where we tried to display some text rotated at 90 degrees, We used a \quoteCmd{multirow}. \\
	Noticed however that the number of rows it spans had to be tweaked manually (it's not 2, it's 12). Colouring it is also complicated (as always with \quoteCmd{multirow})
\end{itemize}


\section{\texttt{longtable}}

A \quoteEnv{longtable} is used for tables which might stretch over several pages. \\

To use a \quoteEnv{longtable}, you first need to load the \quotePkg{longtable} environment:
\lstinputlisting[language=tex]{include/featured/pkg-longtable.tex}

\bigskip

In a \quoteEnv{longtable} environment you may define the following:

\begin{itemize}
	\item the \emph{preamble}, which comprises:
	\begin{itemize}
		\item \emph{caption} and \emph{label} \\
		These are used for cross-references (section \ref{cross-references}), and to show the long table in the list of tables (section \ref{list-of-tables})
		
		\item the \emph{first head} \\
		This is the header which appears at the beginning of the table. \\
		Ends with: \quoteCmd{endfirsthead}
		
		\item the \emph{head} \\
		This is the header which appears at the top of the table for every page after the first.
		If the \emph{first head} is not defined, then the \emph{head} will also appear at the beginning of the table. \\
		Ends with: \quoteCmd{endhead}
	
		\item the \emph{foot} \\
		This is the footer that appears at the bottom of the table for every page after the first.
		If the \emph{last foot} is not defined, then the \emph{foot} will also appear at the end of the table. \\
		Ends with: \quoteCmd{endfoot}
		
		\item the \emph{last foot} \\
		This is the footer that appears at the end of the table. \\
		Ends with: \quoteCmd{endlastfoot}
		
	\end{itemize}
	\item the \emph{content} \\
	The table's \emph{content} of the table is found after the \emph{preamble} \\
\end{itemize}


In terms of usage\dots
\begin{itemize}	
	\item \emph{first head} \\
	Can be used to show the column names
	
	\item \emph{head} \\
	Can be used to indicate \emph{``\dots continued from previous page."} \\
	Can also be used to show the column names (on top of every page)

	\item \emph{foot} \\
	Can be used to indicate \emph{``Continued on next page\dots"}

	\item \emph{last foot} \\
	Closes the table (e.g. print a \quoteCmd{hline} if the table has borders) \\
\end{itemize}



The code below shows how to define a \quoteEnv{longtable}:
\begin{lstlisting}[language=tex]
\begin{longtable}{|l|l|l|}
	\caption{A long table.} 
	\label{table:longtable-example} \\
	
	%%% first head (at the beginning of the table)
	\hline
	\multicolumn{1}{|c|}{Column 1} &
	\multicolumn{1}{c|}{Column 2} &
	\multicolumn{1}{c|}{Column 3}
	\\
	\hline
	\endfirsthead
	
	%%% head (repeated on every page)
	\hline
	\multicolumn{3}{|l|}{\dots continued from previous page.} 
	\\
	\multicolumn{1}{|c}{Column 1} &
	\multicolumn{1}{c}{Column 2} &
	\multicolumn{1}{c|}{Column 3}
	\\
	\hline
	\endhead
	
	%%% foot (repeated on every page)
	\hline
	\multicolumn{3}{|r|}{Continued on next page\dots} \\
	\hline
	\endfoot
	
	%%% last foot (at the end of the table)
	\hline
	\endlastfoot
	
	%%% content...
	
	Data 1 & Data 2 & Data 3 \\
	Data 4 & Data 5 & Data 6 \\
	...
	
\end{longtable}

\end{lstlisting}




\section{Other table packages}

There are other packages allowing to create tables, each with different options:
\begin{center}
	\begin{tabular}{l l}
		\quoteEnv{nicematrix} & \url{https://www.ctan.org/pkg/nicematrix} \\
		\quoteEnv{array} & \url{https://www.ctan.org/pkg/array} \\
		\quoteEnv{tabularx} & \url{https://www.ctan.org/pkg/tabularx} \\
		\multicolumn{1}{c}{\dots} & \multicolumn{1}{c}{\dots} \\
	\end{tabular}
\end{center}



\section{\texttt{table}} \label{table-env}

The \quoteEnv{table} environment is used to reference a table (section  \ref{cross-references}), add a caption, and add the table to the list of tables (section \ref{list-of-tables}). \\

Here is an example:
\lstinputlisting[language=tex]{include/featured/example-table.tex}

And here is the result:
\begin{table}[h]          % we want to anchor the table "here" ([h])
	\caption{Tic-Tac-Toe again} % the table caption
	\label{table:some-table}    % used for cross-references
	\centering                  % used to center the table on the page
	
	\begin{tabular}{c | c | c}
		X & O & X \\
		\hline
		O & X & X \\
		\hline
		X & O & O \\
	\end{tabular}
\end{table}




\section{List of tables} \label{list-of-tables}

We can easily add a list of tables (preferably at the end of the document) with the \quoteCmd{listoftables} command:
\begin{lstlisting}[language=tex]
\listoftables
\end{lstlisting}
