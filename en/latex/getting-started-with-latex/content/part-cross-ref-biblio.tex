%%%%
%% Copyright 2022 Pierre S. Caboche
%% All rights reserved
%%%%

\part{Cross-references, bibliography}


\section{Cross-references} \label{cross-references}

Adding a cross-reference in \LaTeX\ is easy! \\

First, you need to put a label next to the object you want to reference (e.g. section, table, figure\ldots) This is done with the \lstinline|\label| command:

\begin{lstlisting}[language=tex]
\section{Cross-references} \label{cross-references}
\end{lstlisting}

\medskip

Of course, labels must be \emph{unique}! \\

Don't worry though, \TeXstudio\ will highlight any label that has been defined more than once in a document. \\

Labels are \emph{case-sensitive}, but they can be \emph{any} string of characters (including numbers, white spaces, punctuation, multibyte unicode characters\ldots) This is different from identifiers used in bibliography (section \ref{bibliography-file})

For the sake of consistency, it can be a good idea to follow the same naming convention for all types of \texttt{id}, including labels (i.e. only alphanumeric characters, dashes, underscores). \\

Now that we've defined a label, it's easy to reference it.\\

The following shows some examples of how to use commands like \quoteCmd{ref}, \quoteCmd{nameref}, \quoteCmd{pageref} :
\begin{lstlisting}[language=tex]
Section \ref{cross-references} 
is called \emph{``\nameref{cross-references}"} 
and can be found on page \pageref{cross-references}.
\end{lstlisting}


$\ldots$and here is the result:
\begin{mdframed}
Section \ref{cross-references} 
is called \emph{``\nameref{cross-references}"} 
and can be found on page \pageref{cross-references}.
\end{mdframed}


\subsection{Things to avoid with references}

In your document, avoid expressions such as \emph{``in the next section'}, \emph{``in the previous section'}\dots\ Instead, refer the that section (or other object) by reference. \\

As explained in ``\longref{getting-organised}", we put the content of our document in separate files, which we then include using the \quoteCmd{input} command. \\

By doing so, it becomes incredibly easy in \LaTeX\ to reorganise your document by changing the order in which we call \quoteCmd{input} on our different \texttt{.tex} files.
This means that the ``next" or ``previous" section is now completely different. \\

You will not have this problem by using a \quoteCmd{label} and associated reference commands.

\bigskip


\subsection{Tip: Finding references with \TeX studio}

\TeXstudio\ offers some useful features to work with references:

\begin{itemize}
	\item if you right-click on the reference of a \quoteCmd{ref}, \quoteCmd{nameref}, \quoteCmd{pageref}\dots you have the option to ``Go to Definition"
	
	\item likewise, if you right-click on the reference of a \quoteCmd{label}, you have the option to ``Find usages"
	
	\item when tying ``\quoteCmd{ref}\texttt{\{}" (or ``\quoteCmd{nameref}\texttt{\{}",  ``\quoteCmd{pageref}\texttt{\{}"\dots), \TeXstudio\ will list the different \texttt{labels} available
	
	\item \TeXstudio\ will let you know if a \quoteCmd{label} has been declared more than once, or if a reference does not exist (by underlying the label id with a red or green squiggly line).
	
\end{itemize}



\section{Tip: Counting pages}

It is possible (with some caveats, as discussed below), to determine the total number of pages in \LaTeX: put a \quoteCmd{label} on the last page, and reference that label with \quoteCmd{pageref}. \\

Here are the details:

\begin{enumerate}
	\item just before the
	\lstinline|\end{document}|
	at the end of our document, put an empty element (e.g. a \quoteCmd{hphantom}\footnote{see: ``\longref{phantom}"}, or an empty \quoteCmd{mbox}), followed by a \quoteCmd{label} \footnote{
the reason why we put an empty element is, sometimes the index at the end of the document is more than one page long. We want an \emph{invisible} object to serve as an anchor on the \emph{last} page for our label.
}:
\begin{lstlisting}[language=tex]
\mbox{}\label{LastPage}
\end{document}
\end{lstlisting}

Note that if you insert a blank page (section \ref{blank-page}) at the end, and you want to include blank pages in the total page number (probably a bad idea, but it's possible\dots), you would need to make the label as part of the last page:
\begin{lstlisting}[language=tex]
\insertBlankPage[This page intentionally left blank.\label{LastPage}]
\end{document}
\end{lstlisting}
	
	\item now refer to this label with \quoteCmd{pageref} :
\begin{lstlisting}[language=tex]
This document contains \pageref{LastPage} pages.
\end{lstlisting}


\end{enumerate}

And here is the result: ``This document contains \pageref{LastPage} pages." \\


Note: this can only work if your page number starts at 1, and you don't change it with something like a \quoteCmd{setcounter}:
\begin{lstlisting}[language=tex]
\setcounter{page}{42}
\end{lstlisting} 


\newpage

\section{Custom counters} \label{custom-counters}

\LaTeX\ uses counters to keep a reference on all sorts of objects: parts, sections, subsections (section  \ref{parts-sections}), tables (section \ref{table-env}), figures (section \ref{figures}), etc. \\


In \LaTeX\, it is also possible to create your own custom counter. \\

For example, if we were to write a book to teach a foreign language, we would probably incorporate some conversation examples. Then we might want to assign a number to this conversation, so that we can easily refer to it later (e.g. \emph{``In conversation 2, Mr Smith used the \emph{subjunctive mood} to indicate it is important that Mr Novak \emph{be} present at the meeting on Monday."}) \\

For that purpose, we can define a new counter named ``\texttt{conversation}", using the 
\quoteCmd{newcounter} command:
\begin{lstlisting}[language=tex]
\newcounter{conversation}
\end{lstlisting}


If we want to reset the counter for every section (i.e. every time the ``\texttt{section}" counter is increased), then we will declare the ``\texttt{conversation}" counter like this:
\begin{lstlisting}[language=tex]
\newcounter{conversation}[section]
\end{lstlisting}


\newcounter{conversation}
\refstepcounter{conversation}

To increase the counter, we use the \quoteCmd{refstepcounter} command:
\begin{lstlisting}[language=tex]
\refstepcounter{conversation}
\end{lstlisting}

Finally we can output the current counter value with \quoteCmd{the\emph{countername}}. In our example, \texttt{\emph{countername}} will be equal to \texttt{conversation}:

\begin{lstlisting}[language=tex]
Conversation: \theconversation \label{conv:shopping}
\end{lstlisting}

\bigskip

Note that in this code, we used the \quoteCmd{label} command to defined a label \\
(\texttt{conv:shopping}). This label will take the value of the latest counter that was used (i.e. \texttt{\textbackslash theconversation}, the current value of counter \texttt{conversation}). \\

Now we can use this as a reference, using the \quoteCmd{ref} command:
\begin{lstlisting}[language=tex]
In conversation \ref{conv:shopping}, Mr Smith used the subjunctive mood 
to indicate ...
\end{lstlisting}

\bigskip

For more information about counters, see: \\
\url{https://www.overleaf.com/learn/latex/Counters} 

\bigskip



\section{Quotation marks}

While we are on the subject of quoting and referencing, here is how to make quotation marks in \LaTeX:

\begin{table}[h]
	\caption{Quotation marks}
	\centering
	\medskip

	\begin{tabular}{ c c c }
		\lstinline| `Single quotes' | & $\rightarrow$ & `Single quotes' \\
		\lstinline|``Double quotes" | & $\rightarrow$ & ``Double quotes" \\
		\lstinline|``Double quotes''| & $\rightarrow$ & ``Double quotes'' \\
	\end{tabular}
\end{table}


I often use double quotes when referring to sections by name.\\ 
For example: \emph{``\nameref{cross-references}"}.

\section{Quotes}

The following document shows how to add quotes using packages like 
\quotePkg{epigraph},
\quotePkg{fancychapters},
\quotePkg{quotchap},
\quotePkg{dirtytalk}, or
\quotePkg{csquotes}: \citep{typesetting-quotations}\\

\hfill
\url{https://www.overleaf.com/learn/latex/Typesetting_quotations}





\section{Adding a bibliography} \label{add-biblio}

In \emph{``\nameref{cross-references}"}, we've seen how to add references that point to different sections \emph{within} our document. Now we must also learn how to quote source \emph{outside} our document (e.g. books, articles, journal\ldots). This is the role of the bibliography. \\

A few things to know about bibliography styles in \LaTeX:
\begin{enumerate}
	\item there exists different styles of bibliography
	\item there also exists several bibliography packages in \LaTeX\ (\quotePkg{bibtex}, \quotePkg{biblatex}, \quotePkg{natbib})
	\item each library provides different bibliography styles readily available
	\begin{itemize}
		\item therefore, your choice of library may be dictated by the types of bibliography styles they provide
	\end{itemize}
	\item those different libraries are not compatible with each other
	\begin{itemize}
		\item if you were to switch libraries, you might have to delete a few generated files, like the \texttt{.bbl} and \texttt{.aux} one, and then recompile the document
	\end{itemize}
\end{enumerate}

\bigskip



\subsection{Bibliography file} \label{bibliography-file}

Whichever bibliography library you choose, you will need to specify at least one bibliography file (\texttt{.bib}) to store all your external references. \\

Here is a short example:
\begin{figure}[h]
	\caption{bibliography.bib}
	\centering
	\lstinputlisting[language=tex]{content/bibliography.bib}
\end{figure}

As you can see, a reference has a type (\texttt{@book}), an \texttt{id}, and a list of ``fields" (\texttt{key = value,}).

The \texttt{id} is important, as it allows to identify the reference to cite. Unlike with cross-references (section \ref{cross-references}), a bibliography \texttt{id} cannot contain things like spaces, punctuation, etc. (only alphanumeric characters, dashes -) \\

In this example, the fields \texttt{language}, \texttt{date}, and \texttt{isbn} do note seem to be used by \quotePkg{natbib}. However, \quotePkg{natbib} allows to use \texttt{note} to add some information to the reference, so you can use that as a workaround.

\quotePkg{biblatex} doesn't use \texttt{note}, but makes use of the other fields. \\ 


\begin{note}
A bibliography file may contain many references, but only the references that have been cited at least once will appear in the bibliography section of the document.
\end{note}

This means it is possible to share the same bibliography file with several \LaTeX\ documents (think of it as a centralised ``database" for all your external references).


\subsection{Bibliography, APA-like style, with \texttt{natbib}} \label{apa-biblio}


There exists different types of bibliography, but I have a preference for the Harvard APA style. 

I find this style clearer. 
Besides, this style seems to be used by a number of learning institutes (e.g. 
the University of Portsmouth\footnote{see: \url{https://www.city.ac.uk/__data/assets/pdf_file/0017/77030/portsmouth_harvard_guide.pdf}}
, the University of Adelaide\footnote{see: \url{http://maths.adelaide.edu.au/anthony.roberts/LaTeX/ltxxref.php}}
\ldots)

In this article, we will try to get as close as possible to the Harvard APA style of bibliography, using \LaTeX. \\


For this, we will use the \quotePkg{natbib} package.


\begin{note}
To use \quotePkg{natbib}, make sure the bibliography tool is set to  \texttt{BibTeX}. \\
	
In \TeXstudio, go to ``Options" >> ``Configure TeXstudio"; then in the ``Build" menu, change ``Default Bibliography Tool" to ``BibTeX".
\end{note}


Call this in the preamble:
\begin{lstlisting}[language=tex]
\usepackage{natbib}
\end{lstlisting}

\bigskip


Then, in the body of the document, call the following where you want the bibliography to be printed:

\begin{lstlisting}[language=tex]
\bibliographystyle{apalike}
\bibliography{content/bibliography}
\end{lstlisting}



Note:

\begin{note}
When calling \quoteCmd{bibliography}, \emph{DO NOT} specify the \texttt{.bib} extension in the path. \\

Also note that, similarly to the \quoteCmd{input} command, the path is relative to the \emph{root} document, not the current source file.
\end{note}

\medskip

Only the references that have been cited at least once in the document will appear in the bibliography. So the next step will be to add some citation. \\


There are many ways to cite a reference:

\newcommand{\bibid}{\texttt{\{id\}}}

\begin{table}[h] 
	\caption{Citing}
	\centering
	\begin{tabular}{l l l l}
		\\
		& \multicolumn{1}{c}{\texttt{id:}}  & \multicolumn{1}{c}{\texttt{c}} & \multicolumn{1}{c}{\texttt{c-prog}}  \\
		& \quoteCmd{citeauthor}\bibid\     & \citeauthor{c}     & \citeauthor{c-prog} \\
		& \quoteCmd{citeyear}\bibid\       & \citeyear{c}       & \citeyear{c-prog} \\
		& \quoteCmd{cite}\bibid\           & \cite{c}           & \cite{c-prog} \\		
		
		\texttt{(natbib)} & \quoteCmd{citefullauthor}\bibid\ & \citefullauthor{c} & \citefullauthor{c-prog} \\
		\texttt{(natbib)} & \quoteCmd{citeyearpar}\bibid\    & \citeyearpar{c}    & \citeyearpar{c-prog} \\
		\texttt{(natbib)} & \quoteCmd{citep}\bibid\          & \citep{c}          & \citep{c-prog} \\
		\texttt{(natbib)} & \quoteCmd{citealt}\bibid\        & \citealt{c}        & \citealt{c-prog} \\
		\texttt{(natbib)} & \quoteCmd{citealp}\bibid\        & \citealp{c}        & \citealp{c-prog} \\
		\texttt{(natbib)} & \quoteCmd{citet}\bibid\          & \citet{c}          & \citet{c-prog} \\
%		\texttt{(natbib)} & \quoteCmd{citenum}\bibid\        & \citenum{c}        & \citenum{c-prog} \\
	\end{tabular}
	\label{table:citing}
\end{table} 


As shown in table \ref{table:citing}, some commands are only defined in the \quotePkg{natbib} package. Trying to use them with \quotePkg{biblatex} will throw an error. \\



\subsection{Customising the Bibliography title (\texttt{natbib})} \label{customise-biblio-title}

The \quotePkg{natbib} package allows you to customise the bibliography title. \\

This is done by redefining the \quoteCmd{bibsection} command:

\lstinputlisting[language=tex]{include/featured/biblio-custom.tex}

\begin{lstlisting}[language=tex]
The Bibliography title can be customised
\renewcommand{\bibsection}{\section*{External references}}
\end{lstlisting}

\bigskip

This is a change from its default implementation, which would look like this:
\begin{lstlisting}[language=tex]
\renewcommand{\bibsection}{\section*{\currentPart}}
\end{lstlisting}

\newpage

\subsection{Bibliography, using \texttt{biblatex}}

The \texttt{biblatex} library provides the following styles out of the box: \texttt{alphabetic}, \texttt{authortitle}, \texttt{authoryear}, \texttt{draft}, \texttt{numeric}, \texttt{reading}, and \texttt{verbose}. \\

Here is an important note if you want to use \texttt{biblatex} with \TeXstudio:


\begin{note}
To use \quotePkg{biblatex} in \TeXstudio, you will need to configure the bibliography tool to \texttt{biber}. \\

Go to ``Options" >> ``Configure TeXstudio"; then in the ``Build" menu, change ``Default Bibliography Tool" to ``Biber".
\end{note}

\bigskip

In the preamble:
\begin{lstlisting}[language=tex]
\usepackage[utf8]{inputenc}
\usepackage[english]{babel}
\usepackage[style=authoryear]{biblatex}

%% Specify the path to the bibliography file
\addbibresource{content/bibliography.bib}
\end{lstlisting}

Here, we specified the \texttt{style} to be \texttt{authoryear} but other styles exist, as shown in \cite{bibliography-styles} :

\hfill \url{https://www.overleaf.com/learn/latex/Biblatex_bibliography_styles} \\

Then in the document, print the bibliography with \quoteCmd{printbibliography}:

\begin{lstlisting}[language=tex]
\printbibliography
\end{lstlisting}
