
%%%%
%% Copyright 2022 Pierre S. Caboche
%% All rights reserved
%%%%

\part{Your first document} \label{your-first-document}


In this article, we will start with an empty document (created from a built-in template), and then slowly add features to it\dots \\

At first, we will focus on the features necessary to add \emph{content} to our document (organising the document, add notes, table of contents, references, cross-references, bibliography, etc), in other words: everything that is important to present inter-connected ideas in a coherent manner. \\

We'll learn how to customise the style later (and if you've been given a template to follow, then you'll not even need to modify the style\dots)


\section{Starting with a template}

In \TeXstudio, go to ``File >> New From Template\dots", select ``Article" in the list of ``Builtin" template, and finally click ``OK". \\

What we get is the following file:
\lstinputlisting[language=tex]{files/empty-article.tex}

Now in \TeXstudio, click on the ``Build \& View" button. This should generate the document without error\dots

\begin{note}
Congratulations! If the document displayed, then you have a working \LaTeX\ environment!\\

With the installation done, we can now concentrate on learning \LaTeX\ itself. (note: you might still need to install additional packages when required)
\end{note}

\bigskip

Let's take a look at that file\dots \\

A \LaTeX\ file has two main parts:
\begin{itemize}
	\item the \emph{preamble} (everything before \lstinline[language=tex]|\begin{document}| ), which the template calls \emph{``opening"}
		
	\item the document's \emph{body} (\lstinline[language=tex]|\begin{document}| and all of its content)
\end{itemize}

\medskip

The \emph{preamble} is where we invoke all the packages necessary to generate our documents, as well as configure those packages. Some \LaTeX\ commands can only be called in the \emph{preamble} (they will throw an error if called in the \emph{body}); this is especially the case with configuration commands). \\

The \emph{body} is where our document's content goes. \\

\medskip

The \quoteCmd{documentclass} command specifies which document class (template) our document follows. We chose the class \texttt{article}, but there are others built-in in \LaTeX\ (\texttt{letter}, \texttt{beamer}, \texttt{book}, \texttt{memoir}\dots). 

The square brackets \texttt{[]} is where we specify options to our document class, for example \texttt{twoside} %to indicate that each page is to be printed on two sides, and then bound together 
(section \ref{two-sided}). \\

In the \emph{preamble} of our template, you can notice the following commands: \quoteCmd{title} and \quoteCmd{author}. As the name implies, their goal is to declare the document's title and author (section  \ref{set-title-author}).

Then in the \emph{body}, the command \quoteCmd{maketitle} generate the document's title (including author, as well as date). It is possible to  customise the date (section \ref{customise-date}). \\

We'll examine the rest of the template in the following sections:
\begin{itemize}
	\item \lstinline[language=tex]|\begin{abstract}|: section \ref{write-abstract}
	\item \lstinline[language=tex]|\section{}|: section \ref{parts-sections}
\end{itemize}

\bigskip

\subsection{Set the title and author} \label{set-title-author}

The first thing we'll do to that template document, is specify the title and author, which is really straightforward:
\begin{lstlisting}[language=tex]
\title{Article title}
\author{Your Name here}
\end{lstlisting}

\idxCmd{title}
\idxCmd{author}

Next we will modify the document's date\dots 

\subsection{Customise the date}  \label{customise-date}

By default, \quoteCmd{maketitle} will print out the date the document has been last generated (the result of calling \quoteCmd{today}), but we might want to change this for different reasons\dots \\

One reason could be that we want to specify a publication date, and not change afterwards.

Another reason would be to add a revision date, on top of the publication date that is set in stone. We are going to do the latter. \\

Here is how we do it, using the \quoteCmd{date} command:
\lstinputlisting[language=tex]{include/featured/pkg-custom-date.tex}




\newpage

\subsection{Write an abstract} \label{write-abstract}

In our ``Article" document template, you will notice the following block:

\begin{lstlisting}[language=tex]
\begin{abstract}
	
\end{abstract}
\end{lstlisting}

\idxEnv{abstract}

As the name implies, this is where you need to write the abstract for your article. \index{abstract} \\

\bigskip

An abstract is the summary of a research paper (whether published or not). The goals of an abstract is to:
\begin{itemize}
	\item introduce the content of the paper
	\item present some of the ideas developed in the paper
	\item help remember the key points
\end{itemize}

\medskip

An abstract is usually short (one paragraph, 6-7 sentences, 150-200 words). \\


\medskip


If you don't know what to put in an abstract, then a good tip I was given years ago was to
write one or two sentences for each of the following subjects:

\begin{center}
\textbf{B}ackground, \textbf{P}urpose, \textbf{M}ethodology, \textbf{R}esults, \textbf{C}onclusion 
\end{center}

Usually: 1 sentence per subject, and an extra sentence each for Methodology and Conclusion. \\


So that's what the ``abstract" is\dots\ I know, this has nothing to do with \LaTeX, but people might be wondering about the purpose of this block. \\
\medskip

Note that I don't always follow those rules for my articles. Specifically, my abstracts tend to be longer and describe the paper in more details.

\bigskip


\section{Organising your files} \label{getting-organised}


\subsection{File hierarchy}

Unlike other text editors (e.g. LibreOffice), a \LaTeX\ document can be made up of multiple files.

On top of that, \LaTeX\ will create a number of other intermediate files (with extensions such as: \texttt{.aux}, \texttt{.bbl}, \texttt{.blg}, \texttt{.idx}, \texttt{.ilg}, \texttt{.ind}, \texttt{.lof}, \texttt{.log}, \texttt{.log}, \texttt{.out}, \texttt{.toc}, and even a \texttt{.gz} archive), as well as the final \texttt{.pdf} file. \\

All these files will be found in the same folder (in addition to \emph{our} own files, which will define the content of our document). So we need to organise this folder. \\

Our goal is to have only one \texttt{.tex} file at the root of our folder hierarchy, and all other files grouped in different folders. \\

Here is the main hierarchy that I used for this document\footnote{folders may contain sub-folders}:
\begin{description}
	\item[\texttt{include/}] \mbox{} \\
	folder contains our external dependencies, configurations, and custom macros.
	In particular:
	\begin{description}
		\item[\texttt{packages.tex}] \mbox{} \\
		file containing all our dependencies and their configuration
		\item[\texttt{macros.tex}] \mbox{} \\
		file containing our custom macros.
	\end{description}

	\item[\texttt{content/}] \mbox{} \\
	folder containing all the \texttt{.tex} files that form the content of our article,
	divided in parts / chapters / sections, etc.
	
	We also use this folder to store bibliography files (\texttt{.bib}), see Section \ref{add-biblio} \\
	
	I also strongly suggest that you organise your files in a way that makes it easy to see if they belong to a part / chapter / section or other. One way to achieve this is by adopting some naming conventions for your files. \\
	
	For example:
	\begin{description}
	\item[] \mbox{} \\
		\texttt{part-(...).tex\\
		chapter-(...).tex\\
		section-(...).tex}
	\end{description}


	\item[\texttt{files/}] \mbox{} \\
	folder containing other files necessary for generating the document (e.g. images, source code files\dots)
\end{description}

\begin{note}
You are free to use whichever file structure you want. However, in this article we will be using the structure described above, and will refer to each folder or file by their name in that hierarchy.
\end{note}


\bigskip


For your \LaTeX\ documents, I suggest that you organise your files sooner rather than later: \\

Starting from a template, you may start adding the different parts, sections, sub-sections 
(section \ref{parts-sections}) to lay out the structure of your document. 
Then you may add a Table of Content (section \ref{add-toc}) to better see that structure.

Finally, in the different sections you may add some notes in the form of comments (see ``\longref{comments}") to know which ideas will go into which section. \\

At that point, you may still have most of your ideas in just one \LaTeX\ file. \\

However, when you start writing the actual content of your document (not just adding sections names and comments), that's when you need to make different files (in the \texttt{content/} folder).


\newpage

\subsection{Using \texttt{\textbackslash input}}

The \quoteCmd{input} command is used to load the content of a file, \emph{verbatim}, into the current document, where the \quoteCmd{input} command has been called.

\begin{lstlisting}[language=tex]
%% Importing our custom macros
% Copyright 2022 Pierre S. Caboche. All rights reserved.

\usepackage{mdframed}
\mdfsetup{
	%innertopmargin=10pt,
	frametitleaboveskip=-\ht\strutbox,
	frametitlealignment=\center
}


\newcommand{\note}[1]{
	\mdfsetup{
		middlelinewidth=2pt,
		backgroundcolor=yellow!10
	}
	\begin{mdframed}#1\end{mdframed}
	\mdfsetup{
		backgroundcolor=white
	}
}


\newcommand{\cmd}[1]{\lstinline|#1|}


\newcommand{\python}{Python}
\newcommand{\awk}{AWK}
\newcommand{\perl}{Perl}
\newcommand{\julia}{Julia}
\newcommand{\gawk}{\texttt{gawk}}
\newcommand{\mawk}{\texttt{mawk}}
\newcommand{\sed}{\texttt{sed}}
\newcommand{\grep}{\texttt{grep}}

\newcommand{\stdin}{\texttt{stdin}}
\newcommand{\stdout}{\texttt{stdout}}
\newcommand{\stderr}{\texttt{stderr}}

\newcommand{\mysqldump}{\texttt{mysqldump}}
\newcommand{\cat}{\texttt{cat}}
\newcommand{\pv}{\texttt{pv}}
\newcommand{\shell}{\texttt{shell}}

\newcommand{\Unix}{UNIX\textregistered}
\newcommand{\Intel}{Intel\textregistered}
\newcommand{\Core}{Core\texttrademark}



  ...

%% Adding some content to our document
%%%%
%% Copyright 2022 Pierre S. Caboche
%% All rights reserved
%%%%

\part{Text styles, indexes}

In this part, we'll talk about custom text styles and adding indexes.  
This will be useful when we learn how to write our first macros, in section ``\longref{first-custom-macro}".

\section{\emph{Emphasis}, italics, bold\dots} \label{text-style}

A recurring question from people new to \LaTeX\ is: \emph{``how do I put text in \textbf{bold} face?"} when the underlying question really is \emph{``how can I \emph{emphasise} some important idea?"} \\

In \LaTeX, there is a command specifically used for \emph{emphasis}: \quoteCmd{emph}

The difference is, by default in \LaTeX, \emph{emphasis} is not made using \textbf{bold face}, but using \textit{italics} instead.

Also, \emph{if you were to \emph{emphasise} some text inside of another \emph{emphasis}, then the inner \emph{emphasis} will appear \emph{upright} (and not in italics), as if the two \emph{emphases} had cancelled each other\dots} \\

So we advise you to \quoteCmd{emph} for emphasis in \LaTeX. \\

However, if you really \emph{must} specifically put your text in bold, italics, underline, etc., then here are some commands for that: \\

\begin{tabular}{l l l l}
	\quoteCmd{emph}\texttt{\{text\}}      & \emph{text}      & \emph{emphasis}          \\
	\quoteCmd{textit}\texttt{\{text\}}    & \textit{text}    & \textit{italics}         \\
	\quoteCmd{textbf}\texttt{\{text\}}    & \textbf{text}    & \textbf{bold face}       \\
	\quoteCmd{texttt}\texttt{\{text\}}    & \texttt{text}    & \texttt{monospaced font} & \footnotesize (as configured in section \ref{changing-the-fonts}) \\
	\quoteCmd{underline}\texttt{\{text\}} & \underline{text} & \underline{underlined}   \\
\end{tabular}


\section{Adding an index} \label{add-index}

First, we need to use the \quotePkg{imakeidx} package and call the \quoteCmd{makeindex} command:

\lstinputlisting[language=tex]{include/featured/pkg-imakeidx.tex} 

Then we need to use the \quoteCmd{index} command, to add an index on the relevant word:

\begin{lstlisting}[language=tex]
Some text here, mentioning a keyword to be indexed. \index{keyword}
\end{lstlisting}


Finally, we need to print the index with \quoteCmd{printindex} :

\begin{lstlisting}[language=tex]
\printindex
\end{lstlisting}

The difficult part is to specify an index whenever certain words are mentioned.
For this purpose, we can use macros to make our task easier\ldots \\

To see what an index looks like, go to page \pageref{our-index} (for this is where you will find the index for this document).


\end{lstlisting}

\bigskip

This way, the content of our document can be broken down into smaller, more manageable files, but they will appear as one big file to the \LaTeX\ compiler. As a result of this behaviour\dots

\begin{note}
The path specified as parameter to the \quoteCmd{input} command 
is always relative our main \texttt{.tex} document at the \emph{base} of our folder hierarchy.
\end{note}

For example, if from a \texttt{.tex} document in our \texttt{content/} folder we include another \texttt{.tex} file, its path must be relative to the \emph{base} of our folder hierarchy, where our main \texttt{.tex} document is located.

\begin{note}
This is also true for other \LaTeX\ command that accept a file path as a parameter, like \quoteCmd{includegraphics} (page \pageref{includegraphics}), or \quoteCmd{lstinputlisting} (page \pageref{lstinputlisting}).
\end{note}

\bigskip

The \quoteCmd{input} command is not to be confused with the \quoteCmd{usepackage} command\dots\
(see next section) \\


\subsection{Calling \texttt{\textbackslash usepackage}}
The \quoteCmd{usepackage} command is used to load one or more \LaTeX\ packages, sometimes with some parameters. \\

For example:
\begin{lstlisting}[language=tex]
%%% In the preamble
	
%% Loading one package
\usepackage{xstring}

%% Loading several packages
\usepackage{titleps, xcolor}

%% Loading one package, with some option
\usepackage[super]{nth}
\usepackage[authoryear]{natbib}
\end{lstlisting}

Note that \quoteCmd{usepackage} can only be called in the preamble (not in the body of the document).

\newpage
\section{Parts, Sections, etc.} \label{parts-sections}

A document of class \texttt{article} (like this one), is divided into: parts, sections, subsections. 
The document class \texttt{book} also adds the ``chapter" subdivision: \\

\quoteCmd{part}

\quoteCmd{chapter} (\texttt{book} only)

\quoteCmd{section}

\quoteCmd{subsection} \\


For example:
\begin{lstlisting}
\part{Your first document}
   % ... introduce the part ...
   
\section{Starting with a template}
   % ... content of the section ...

\subsection{Set the title and author}
   % ... content of the subsection ...
\end{lstlisting} 

\bigskip

The previous commands will define the different subdivisions that need to be \emph{numbered}.
Such subdivisions will not only be assigned a number, but they will also appear in the Table of Contents if you add one (see Section \ref{add-toc}). \\

Also worth noting: the \quoteCmd{part} command does not reset the numbering of the chapters / sections it contains (i.e. you could have Sections 1, 2, 3 in part I; Sections 4, 5, 6, 7 in part II, etc.)

\bigskip

If you don't want to number a subdivision, use the \emph{starred} version of the previous commands, i.e.:

\quoteCmd{part*}

\quoteCmd{chapter*} (\texttt{book} only)

\quoteCmd{section*}

\quoteCmd{subsection*} \\


\begin{note}
Non-numbered subdivisions will not appear in the Table of Contents.
\end{note}


\newpage

\section{Adding a Table of Contents (TOC)} \label{add-toc}

Adding a table of contents is very easy. \\

Just call \quoteCmd{tableofcontents} where you want your Table of Contents to be (only one per document) :
\begin{lstlisting}[language=tex]
\tableofcontents
\end{lstlisting}



\subsection{Customising the Table of Contents} \label{customising-toc}

It is also possible to customise the Table of Contents. \\

Here is what I used for this article:
\lstinputlisting[language=tex]{include/featured/customise-toc.tex}

\bigskip

\begin{note} 
Note: if you modify those parameters, you might need to generate your document several times for the changes to take effect.
\end{note}

\bigskip

More customisation options can be found at \citep{customize-toc} : \\

\hfill
\url{https://texblog.org/2011/09/09/10-ways-to-customize-tocloflot/}


\bigskip	


%%%%
%% Copyright 2022 Pierre S. Caboche
%% All rights reserved
%%%%


\newpage

\section{Paragraphs, footnotes}

In this section, we will learn how to make paragraphs, sub-paragraphs, etc. \\

The best way to explain how it works is to show the \LaTeX\ code, and show you how it renders. \\



This is the \LaTeX\ code:

\lstinputlisting[language=tex]{content/sample-paragraphs.tex}

\bigskip



\dots and this is the output. Please note that the code explains how paragraphs work. I needed some sample text, but instead of using a generic \LoremIpsum, I figured that this section could be self-describing\dots

\bigskip

\begin{mdframed}
	This is the first line of our first paragraph (after all, we have to start somewhere\ldots)
This is on a new line (in the \LaTeX\ code), but it is still part of the same paragraph (in the generated document); so it is possible to add comments without disrupting the flow of the paragraph. For example\dots\ 
% This is a comment...
% Everything after the % until the end of the line is a comment, and will be ignored
this was added after the comment. As you can see, this is still part of the same paragraph; the comment didn't appear, and the flow of text was not disrupted.


By leaving (at least) one empty line between this and the previous block of text, we start a new sub-paragraph. Sub-paragraphs are very useful if you want to introduce a new topic in continuation with the previous point, but not a completely new idea that would warrant a new paragraph.

If you want to create a new paragraph, you need to terminate the current paragraph with a pair of backslashes ( \texttt{\textbackslash\textbackslash} ) and then leave at least one empty line between this and the next block of text.
\\


This is the start of a new paragraph. Now let's see what happens when we add a pair of backslashes ( \texttt{\textbackslash\textbackslash} ) but do not leave an empty line with the next block of text\dots 
\\
This is a new sub-paragraph; but unlike the others, this one is not indented (please, bear in mind: in some cases, that can make it difficult to see that this is a sub-paragraph at all).
\\

So far we've seen that a new paragraph can be used to introduce a new idea, while a new sub-paragraph can be used to introduce a new point pertaining to the same idea. The layout of the text (indentation, line breaks) will help us differentiate the different ideas/points.
\\

\medskip

And if you need a bigger break between paragraphs, you can use \texttt{\textbackslash medskip} or \texttt{\textbackslash bigskip} (medium or big skip, respectively). Use sparingly, though.

\begin{center}
	\textreferencemark\textreferencemark\ 
	Now, a word about footnotes\dots\ 
	\textreferencemark\textreferencemark
\end{center}

Sometimes when talking about a particular subject, you might feel the need to add some related information. Your first instinct would be to put such information in parentheses (but the problem with parentheses is that they appear in the middle of a text, and break the flow of the discussion. So you usually want to keep them both short and relevant. This is a counter-example, to illustrate what happens when the content in parentheses is too long: we tend to lose focus and forget what we were talking about). \\

One way to avoid being side-tracked by tangent ideas is to use \emph{footnotes}. As the name implies, footnotes are notes that appear at the foot of the page (or, in this case, at the end of a \texttt{mdframed} box) and referred to by number, like this\footnote{this is an example of a footnote. A footnote will not disrupt the flow of the discussion, so they can be used to add some comment or extra information that would be too long to put in parentheses
}.\\
\end{mdframed}


\bigskip
\bigskip


\subsection{Footnotes} \label{footnote}

In the previous code sample, we saw an example of a \texttt{footnote}. \\

In \LaTeX, it's very easy to add a footnote with the \lstinline[]|\footnote| command:
\begin{lstlisting}[language=tex]
This is some text\footnote{and this is the footnote}.
\end{lstlisting}
\medskip

Footnotes will appear at the foot of the page (see: example\footnote{this is an example of a footnote. As you can see, it is able to hold a lot more information than a margin note
}), 
or at the bottom of components like a \texttt{mdframed} box. \\

As mentioned earlier, footnotes can be useful when you want to add some information but it would be too long to put in parentheses without disturbing the flow of ideas.\\

\bigskip

So the following can be used to add additional information:
\begin{itemize}
	\item parentheses: used for short explanations, closely related to the subject
	\item footnotes: used for additional information that is either too long to be in parentheses, or not closely related to the subject
	\item margin notes (section \ref{margin-notes}): to bring the reader's attention on a particular subject
\end{itemize}

\newpage


\subsection{Margin notes and paragraphs} \label{margin-notes}

\LaTeX\ also allows to put notes in the margins, thanks to \emph{margin paragraph} and \emph{margin notes}.
\\

\idxCmd{marginpar}
\marginpar{This is a margin paragraph} To 
add a margin paragraph, use the \lstinline[language=tex]|\marginpar| 
command:
\lstinputlisting[language=tex]{content/sample-marginpar.tex}


\medskip

To add a \emph{margin note}, you will need \quotePkg{marginnote} package, 
which provides the \quotePkg{marginnote} command:
\lstinputlisting[language=tex]{include/featured/pkg-marginnote.tex}
\medskip

\idxCmd{marginnote}
\marginnote{This is a margin note} A 
margin note can be added by using the \lstinline[language=tex]|\marginnote| 
command:
\lstinputlisting[language=tex]{content/sample-marginnote.tex}


\bigskip

Notice how in both cases, we put the \quotePkg{marginpar} or \quotePkg{marginnote} command immediately after the first word in the paragraph. This is to make sure the margin paragraph/note will be aligned vertically with the first word of the paragraph. \\

Also keep in mind that in a two-sided document (section~\ref{two-sided}), a margin paragraph will be on the \emph{outer} margin (i.e. left margin for even-numbered pages, right margin for odd-numbered pages). If the document is one-sided, margin paragraphs will appear in the right margin. \\


A \marginpar{marginpar vs. \\ footnote}
margin paragraph (or note) stands out from the rest of the text. 
They can be used to bring the reader's attention on a particular paragraph. However, \emph{it is better to keep them short}. \\

In contrast, a footnote (\emph{section \ref{footnote}}) can be much longer, and is used to provide additional information without breaking the flow of the discussion.
\newpage



\subsection{Putting text in a frame} \label{mdframed}

Previously, we've seen how margin notes and margin paragraphs could be used to bring attention to certain parts of your document. \\

Now we will learn how to make some ideas stand out, by putting them in a big box, thanks to the \quotePkg{mdframed} package:
\lstinputlisting[language=tex]{include/featured/pkg-mdframed.tex}

\bigskip

The \quotePkg{mdframed} package defines, amongst other things, the \quoteEnv{mdframed} environment. Here is a very simple example :
\lstinputlisting[language=tex]{include/featured/example-frame-1.tex}

You can notice how an environment is delimited by \texttt{\textbackslash begin\{\emph{envname}\}} and \\
\texttt{\textbackslash end\{\emph{envname}\}}. \\

And here is the result of the code above:
\begin{mdframed}
Help! I've been framed!
\end{mdframed}



\bigskip



This is a very simple frame, but we can modify its appearance with the \quoteCmd{mdfsetup} command.

One thing to know about the \quoteCmd{mdfsetup} command, is that is modifies the properties of \emph{every} \quoteEnv{mdframed} that comes after it. This is a problem if you want to modify the attributes of only \emph{one} frame. \\


To get around this issue, we will create an \emph{environment}. \\
We will name it ``\texttt{vegas}". Why? because \emph{``what happens in \emph{\texttt{vegas}} stays in \emph{\texttt{vegas}}"} (and what happens in an environment stays in that environment"). \\

So here is our ``\texttt{vegas}" environment:
\lstinputlisting[language=tex]{include/featured/example-frame-vegas-env.tex}
\newenvironment{vegas}[1][!!! Welcome to Vegas !!!]{
	\mdfsetup{
		frametitle={
			\tikz
			\node[rectangle,fill=yellow!60,draw=red, ultra thick]
			{#1};
		},
		frametitleaboveskip=-\ht\strutbox,
		frametitlealignment=\centering,
		backgroundcolor=yellow!30,		
		linewidth=4pt,
		linecolor=red!70,
		roundcorner=8pt,		
		shadow=true,
		shadowcolor=orange!40,
	}
	\begin{mdframed}}
	{\end{mdframed}}


For the purpose of this example, we declared this environment with one optional parameter: the title of the frame, which defaults to ``!!! Welcome to Vegas !!!". \\

Now it's time to use this environment:
\lstinputlisting[language=tex]{include/featured/example-frame-vegas.tex}

And here is what it looks like: \\
\begin{vegas}
What happens in this environment stays in this environment\ldots
\end{vegas}


\bigskip

Wow, so many colours! \\


\bigskip

Now, let's create another frame:
\lstinputlisting[language=tex]{include/featured/example-frame-2.tex}

And here is the result:
\begin{mdframed}[
	frametitle={The title of the frame},
	frametitleaboveskip=-\ht\strutbox,
]
Help! I've been framed!
\end{mdframed}



As you can see, our very flamboyant ``\texttt{vegas}" environment did not have any influence over to the new frame. The \quoteCmd{mdfsetup} command only affected the environment where it was used (i.e. ``\texttt{vegas}"), and had no effect outside of it. \\

\bigskip


Again, we can define a new environment, call \quoteCmd{mdfsetup} inside of it, and the effect of \quoteCmd{mdfsetup} will not be seen outside of the environment:
\lstinputlisting[language=tex]{include/featured/example-frame-blue-env.tex}
\newenvironment{blueframe}[1]{
\mdfsetup{
	frametitle={
		\tikz
		\node[rectangle,fill=blue!20]
		{#1};
	},
	linecolor=blue!20,
	linewidth=2pt
}
\begin{mdframed}}
{\end{mdframed}}



\lstinputlisting[language=tex]{include/featured/example-frame-blue.tex}


Result:
\begin{blueframe}{Put title here}
Help! I've been framed!
\end{blueframe}



\bigskip

In our two previous environments (``\texttt{vegas}" and ``\texttt{blueframe}"), we passed the frame title as a parameter (whether optional or mandatory). However, if we don't need to add a title, there is a simpler way to define frame environment, thanks to the \quoteCmd{newmdenv} command. \\


Below is the definition for ``\texttt{note}", an environment which I used to add some important notes throughout this document, in the form of a frame with a slightly yellow background:
\lstinputlisting[language=tex]{include/featured/note-env.tex}

Here is how to call it:
\lstinputlisting[language=tex]{include/featured/example-frame-note.tex}
Here is the result:
\begin{note}
This is an example of note.
\end{note}



\bigskip

As you can see, all the options that we would normally pass to the \quoteCmd{mdfsetup} command are passed in the optional parameter (in square brackets \texttt{[]}) of \quoteCmd{newmdenv}. \\
The name of the environment is passed as the mandatory argument (in curly brackets \texttt{\{\}}) of \quoteCmd{newmdenv}.

\bigskip

Another way to modify the appearance of a \quoteEnv{mdframed} is to define a new style with the \quoteCmd{mdfdefinestyle} command, then use that style by specifying the \texttt{style} attribute in the \quoteEnv{mdframed} options:

\lstinputlisting[language=tex]{include/featured/example-frame-pinky-env.tex}
And the result:
\mdfdefinestyle{pinky}{
	linecolor=pink,
	linewidth=2pt,
	backgroundcolor=pink!30
}

\begin{mdframed}[style=pinky]
They're Pinky and the Brain, \\
Yes, Pinky and the Brain, \\
One is a genius, the other's insane\ldots
\end{mdframed}



\bigskip


This concludes our presentation of \quotePkg{mdframed}. \\

If you want to know more about this package (with many other examples), please refer to its official documentation: 
\begin{center}
\url{https://www.ctan.org/pkg/mdframed}
\end{center}


\newpage

\subsection{Other types of boxes}

There exists other kinds of boxes, to draw boxes around text:
\begin{itemize}
	\setlength{\itemsep}{-0.5em}
	
	\item \quoteCmd{parbox}
	\item \quoteCmd{mbox}
	\item \quoteCmd{makebox}
	\item \quoteCmd{fbox}
	\item \quoteCmd{framebox}
	\item \quotePkg{fancybox} (package)
\end{itemize}

\bigskip

For more information, please see:
\begin{center}
	\url{https://en.m.wikibooks.org/wiki/LaTeX/Boxes}
\end{center}

\bigskip



\section{Comments} \label{comments}


In \LaTeX, a comment starts with: \% \\


Everything after the ``\%" till the end of the line will be ignored. If you want to actually print a ``\%" in \LaTeX, you'll need to escape it with a backslash: \textbackslash \% \\

Unlike in other computer languages, there are no multiline comments in \LaTeX, only single-line comments.
\\


Fortunately, an editor like TeXStudio can make the process of adding/removing comments a lot easier, thanks to the \texttt{Ctrl+T} shortcut (for \emph{``Toggle Comment"}).

\begin{mdframed}
\medskip
	To toggle comments in TeXStudio:
	\begin{itemize}
		\item Select the lines you want to add comments to (or remove comments from)
		\item Press \texttt{Ctrl+T}
		
		This will either add or remove the comments from those lines
	\end{itemize}
\medskip
\end{mdframed}



Comments are extremely useful in \LaTeX! In fact\dots 

\begin{note}
When you think about it, comments are one of the most powerful features in \LaTeX. \\

Indeed, comments allow you to keep multiple drafts of the same paragraphs, which you can then enable/disable with a single \texttt{Ctrl+T} in \TeXstudio. This way, you never have to throw away any idea! (just comment them out, so they don't show up in the document).
\end{note}



Comments have a number of applications:

\begin{description}
	\item[\emph{Describing}] \mbox{}\\
	The most obvious use of a comment is to describe what some \LaTeX\ code is doing (e.g. describe the packages that you are using and what you are using them for)
	
	\item[\emph{Taking notes}] \mbox{}\\
	For example, if you already have your document's plan laid out (i.e. you've already defined the structure in terms of: parts, sections, sub-sections\ldots), you can leave comments to list the main ideas to put in those sections later (as some kind of \emph{TODO} list)
	
	Furthermore, if several people are working on a document you can leave notes to other contributors in the form of comments (and if you work alone, you can always leave notes to yourself)
	
	\item[\emph{Drafting ideas}] \mbox{}\\
	This is probably the most important use of comments (this represents about 99\% of the cases when I use comments in \LaTeX)
	
	Sometimes you're writing things but are not completely satisfied with the wording. In \LaTeX, it's very easy to put a sentence in comments, rewrite it, and repeat the process until you're satisfied (while not having to discard what you wrote, and not having those earlier versions appear in your document).
\end{description}


In many of those cases, you don't want your comments to be removed by accident (i.e. any case other than \emph{Drafting ideas}). To avoid that, simple use more than one ` \lstinline|%| ' for your comment. \\

This way, one \texttt{Ctrl+T} will not be enough to remove the comment\ldots
\begin{lstlisting}[language=tex]
%%% Packages and configuration
\end{lstlisting}


\bigskip


\section{Collaboration, version management}

The main focus of \LaTeX\ is document generation. As such, it does not provide dedicated collaboration and version management tools. \\

However, \LaTeX\ is a markup language. This means that, collaboration and management of \LaTeX\ files can be handled with the same tools as any other source files; so you can use your favourite version management tools with \LaTeX\ (e.g. GIT, SVN\dots) \\

In section ``\longref{comments}", we've seen how \LaTeX\ comments can be used for note-taking, drafting entire paragraphs, etc. A version-control management system can help you keep a history of such items.


\newpage

\part{Managing breaks}

In part \ref{your-first-document}, we talked about how to how to write paragraphs, how to organise your documents in sections, etc. \\

Sometimes, however, a paragraph or section break does not land in a place that is convenient for us, as it disrupts the flow of the article, For example, a section may start at the bottom of a page, a figure may end up on a different page from the text that references it (I always find this confusing to read when that happens), etc. 

To avoid that, you might want to make some manual adjustments to the document layout. This is the subject of this part\dots

\section{Non-breaking spaces}

In \LaTeX, you can create a non-breaking space by using a tilda: $\sim$ \\

Non-breaking spaces are used to keep certain terms together, without automatic line breaks being inserted in the middle. \\

This is useful, for example, to keep numbers and their units together:

\begin{lstlisting}[language=tex]
	The maximum theoretical speed of SATA-3 is 6~Gb/s, or 750~MB/s.
\end{lstlisting}

\medskip

Whenever possible, \LaTeX\ will keep those terms together. \\

In the previous example, the ``6" should not be separated from its ``Gb/s" unit, and the ``750" should not be separated from the ``MB/s".

\bigskip


\section{Phantom text} \label{phantom}

Since we are on the subject of adding breaks and spaces, here are the commands to add some empty space based on the width/height of a given text:

\begin{description}
	\item{\quoteCmd{phantom}\texttt{\{text\}}} \mbox{} \\ 
		Creates some space, of the same width and height as the \texttt{text} passed as parameter
		(a combination of \quoteCmd{hphantom} and \quoteCmd{vphantom})
		
	\item{\quoteCmd{hphantom}\texttt{\{text\}}} \mbox{} \\ 
		Creates some horizontal space, of the same width as the \texttt{text} passed as parameter
	\item{\quoteCmd{vphantom}\texttt{\{text\}}} \mbox{} \\ 
		Creates some vertical space, of the same height as the \texttt{text} passed as parameter
\end{description}

\newpage

Below is an example of how to put an \quoteCmd{underline} under some \quoteCmd{phantom} text:
\begin{lstlisting}[language=tex]
I really think that \underline{\phantom{\LaTeX}} is great!
\end{lstlisting}

And here is the result:
\begin{mdframed}
I really think that \underline{\phantom{\LaTeX}} is great!
\end{mdframed}

\bigskip

\section{Skips and breaks}


\subsection{Adding vertical spaces}

Sometimes you might need to add some extra vertical space between paragraphs (or other elements). These are called ``skip". \\

Skips come in different sizes: \\

\begin{tabular}{l l}
	\quoteCmd{smallskip} & a small skip \\
	\quoteCmd{medskip}   & a medium skip \\
	\quoteCmd{bigskip}   & a big skip \\
\end{tabular}

\bigskip


\subsection{Page breaks}

You can start a new page with the \quoteCmd{newpage} command. \\

You can also start a new page with \quoteCmd{pagebreak}, in which case the content of the old page will be spread out vertically. \\

Having several \quoteCmd{newpage} (or \quoteCmd{pagebreak}) will not create empty pages. To insert empty page(s), see section \ref{blank-page}\dots

\bigskip


\subsection{Inserting a blank page} \label{blank-page}


While writing about \quoteCmd{newpage} and \quoteCmd{pagebreak}, I wondered \emph{``how to insert a blank page in a \LaTeX\ document?"} \\

After a minute of searching, I found the following solution, which I then wrapped in a custom macro with an optional parameter (section \ref{macro-optional-parameter}):


\lstinputlisting[language=tex]{include/featured/cmd-insertBlankPage.tex}


The macro will insert a blank page, with a custom message (passed as an optional parameter) written in the center of the page. If the message is not specified, it will default to ``This page intentionally left blank." \\

This macro uses the \quoteCmd{thispagestyle} command, which sets the style for the current page only. By setting it to \texttt{empty}, we remove the header and footer temporarily. Therefore only the message will appear on the center of the page. \\

Here are some examples of how to call our custom macro:
\begin{lstlisting}[language=tex]
% To display a blank page with a custom message
\insertBlankPage[Some text]

% To display a blank page 
% with the default "This page intentionally blank."
\insertBlankPage

% In this case, the message will actually be blank
\insertBlankPage[]
\end{lstlisting}


\bigskip

We will be calling the following:
\begin{lstlisting}
\insertBlankPage
\end{lstlisting}

See next page for the result\dots
\insertBlankPage






