%%%%
%% Copyright 2022 Pierre S. Caboche
%% All rights reserved
%%%%

\part{Text styles, indexes}

In this part, we'll talk about custom text styles and adding indexes.  
This will be useful when we learn how to write our first macros, in section ``\longref{first-custom-macro}".

\section{\emph{Emphasis}, italics, bold\dots} \label{text-style}

A recurring question from people new to \LaTeX\ is: \emph{``how do I put text in \textbf{bold} face?"} when the underlying question really is \emph{``how can I \emph{emphasise} some important idea?"} \\

In \LaTeX, there is a command specifically used for \emph{emphasis}: \quoteCmd{emph}

The difference is, by default in \LaTeX, \emph{emphasis} is not made using \textbf{bold face}, but using \textit{italics} instead.

Also, \emph{if you were to \emph{emphasise} some text inside of another \emph{emphasis}, then the inner \emph{emphasis} will appear \emph{upright} (and not in italics), as if the two \emph{emphases} had cancelled each other\dots} \\

So we advise you to \quoteCmd{emph} for emphasis in \LaTeX. \\

However, if you really \emph{must} specifically put your text in bold, italics, underline, etc., then here are some commands for that: \\

\begin{tabular}{l l l l}
	\quoteCmd{emph}\texttt{\{text\}}      & \emph{text}      & \emph{emphasis}          \\
	\quoteCmd{textit}\texttt{\{text\}}    & \textit{text}    & \textit{italics}         \\
	\quoteCmd{textbf}\texttt{\{text\}}    & \textbf{text}    & \textbf{bold face}       \\
	\quoteCmd{texttt}\texttt{\{text\}}    & \texttt{text}    & \texttt{monospaced font} & \footnotesize (as configured in section \ref{changing-the-fonts}) \\
	\quoteCmd{underline}\texttt{\{text\}} & \underline{text} & \underline{underlined}   \\
\end{tabular}


\section{Adding an index} \label{add-index}

First, we need to use the \quotePkg{imakeidx} package and call the \quoteCmd{makeindex} command:

\lstinputlisting[language=tex]{include/featured/pkg-imakeidx.tex} 

Then we need to use the \quoteCmd{index} command, to add an index on the relevant word:

\begin{lstlisting}[language=tex]
Some text here, mentioning a keyword to be indexed. \index{keyword}
\end{lstlisting}


Finally, we need to print the index with \quoteCmd{printindex} :

\begin{lstlisting}[language=tex]
\printindex
\end{lstlisting}

The difficult part is to specify an index whenever certain words are mentioned.
For this purpose, we can use macros to make our task easier\ldots \\

To see what an index looks like, go to page \pageref{our-index} (for this is where you will find the index for this document).

