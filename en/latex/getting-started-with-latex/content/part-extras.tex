%%%%
%% Copyright 2022 Pierre S. Caboche
%% All rights reserved
%%%%

\part{Extras}

In this part, we list some techniques, as well as some very useful packages (which I actually used in this document) that didn't get to have their own chapter\dots


\section{Other useful packages}

\subsection{Adding web links with \texttt{hyperref}}

The \quotePkg{hyperref} package defines the \quoteCmd{url} command, which allows to add links to web pages. \\

Here is how to import the package:
\lstinputlisting[language=tex]{include/featured/pkg-hyperref.tex}

\bigskip

The \quoteCmd{url} command is very straightforward:
\lstinputlisting[language=tex]{include/featured/example-url.tex}

Result: \\
\input{include/featured/example-url.tex}


\bigskip


\subsection{Ordinal numbers with \texttt{nth}}

The \quotePkg{nth} package is very useful to print ordinal numbers in English (\nth{1}, \nth{2}, \nth{3}, \nth{4}\dots)\\

Here is how to import the package, with the ordinal mark set to be \emph{superscripted}:
\lstinputlisting[language=tex]{include/featured/pkg-nth.tex}

Here are some examples:
\lstinputlisting[language=tex]{include/featured/example-nth.tex}

And here is the result: \\
\nth{1}, \nth{2}, \nth{3}, \nth{4}, \nth{21}, \nth{22}, \nth{23}, 
\nth{101}, \nth{1001}, \nth{1002}, \nth{1000001}
\\


Without \quoteCmd{nth}, you would have to use the \quoteCmd{textsuperscript} command. \\



\newpage

With languages other than English, you would have to use the \quoteCmd{textsuperscript} command (when applicable). For example in French: \\
1\textsuperscript{er} -- premier (masculine), 
1\textsuperscript{ère} -- première (feminine),
2\textsuperscript{ème} -- deuxième, 
3\textsuperscript{ème} -- troisième\dots

\lstinputlisting[language=tex]{include/featured/example-nth-french.tex}

As you can see, this is more complicated in French. \\

Not only do you have to cater for masculine/feminine/plural, but there are other rules. For example, ``\nth{2}" normally translates to << deuxième >> if there are more than two elements, but translates to << second(e) >> if there are more than two elements. \\

This goes to show that the \quotePkg{nth} package may not have an equivalent in other languages.
