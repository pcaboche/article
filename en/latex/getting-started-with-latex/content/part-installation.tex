%%%%
%% Copyright 2022 Pierre S. Caboche
%% All rights reserved
%%%%

\part{Installation}


\section*{Setting up a \LaTeX\ environment}

The official \LaTeX\ Project website %\citep{latex-project} 
provides some information about \LaTeX\ (including how to install \LaTeX). \\

Link: \url{https://www.latex-project.org} \\

That being said, here is some information to get you started\dots


\subsection*{Windows}

On Windows, I would recommend using \MiKTeX, which includes (among other things):

\begin{itemize}
	\item an \emph{``integrated package manager"}, which will help you download the missing \TeX\ packages, as you need them. This allows you to keep \emph{``just enough \TeX"} on your computer for your work \citep{miktex-project}
	\item \TeXworks, a \emph{``simple \TeX\ front-end program (working environment)"} \citep{texworks}
\end{itemize}

When it comes to \TeX\ editors, I have a preference for \TeXstudio, which we've introduced in \emph{Part \longref{texstudio}} \\



So on Windows, I would usually install the following:
\begin{itemize}
	\item \MiKTeX:   \url{https://miktex.org}
	\item \TeXstudio: \url{https://www.texstudio.org} 
\end{itemize}


\subsection*{Linux}

Under Linux, I tend to install \texttt{texlive}  
(as most Linux distribution provide it through their official repositories, %\citep{texlive})
 as well as an editor for \TeX\ (usually \TeXstudio).

\begin{description}
	\item[texlive] \mbox{} \\ 
	\emph{Required} \\
	\TeX\ formatting system
	
	\item[texstudio] \mbox{} \\
	\emph{Optional, but highly recommended} \\
	A feature-rich editor for \LaTeX\ documents \\
\end{description}


To install these packages in Fedora (requires \emph{super user} privileges):
\begin{lstlisting}[language=sh]
	$ sudo dnf install texlive texstudio
\end{lstlisting}

\bigskip

Additional \TeX\ packages need to be installed separately. \\

Such packages normally have a name that starts with "\texttt{texlive-}" \\
( e.g. \texttt{texlive-mdframed} )

\bigskip
