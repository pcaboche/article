% Copyright 2022 Pierre S. Caboche. All rights reserved.


\renewcommand{\currentPart}{Other Solr features}
\part{\currentPart}   \label{part - Other Solr features}


\section{``More Like This"} \label{mlt}


\subsection{What is ``More Like This"?} \label{what-is-MLT}

\MLT\ is a feature which, given a document, will:
\begin{enumerate}
	\item extract a list of ``interesting terms" (based on TF-IDF: \emph{term frequency–inverse document frequency}, see section~\ref{tf–idf}) from a certain field	
	\item find other documents with similar terms in them
\end{enumerate}




\bigskip

\subsection{Enabling ``More Like This"} \label{enabling-mlt}

In Solr, the \MLT\ feature is not accessible by default. You need to add a request handler before you can start using it. This can be done by either editing the \texttt{solrconfig.xml}, or by via the Config API. \\

We will activate it with the Config API, similar to this:
\begin{lstlisting}[language=sh]
curl -X POST -H 'Content-type:application/json' -d '{   
	"add-requesthandler": {                            
		"name": "/mlt",
		"class": "solr.MoreLikeThisHandler",
		"defaults": {"mlt.fl": "lyrics_txt_ja"}
	}
}' "http://${HOST}:${PORT}/solr/${CORE}/config"
\end{lstlisting}

\bigskip



This is done in our script \texttt{mysolr.sh}, which is was featured on page \pageref{mysolr.sh}. \\

All you need to do to enable \MLT\ is call:
\begin{lstlisting}[language=sh]
./mysolr.sh enable_mlt
\end{lstlisting}



\subsection{Accessing ``More Like This"}

Unlike with regular query (section \emph{\longref{using-solr}}), the \MLT\ feature does not have a dedicated page with a User Interface that you can access through the browser. Instead, we need to query \MLT\ via HTTP requests, at:
\begin{center}
	\texttt{http://localhost:8983/solr/songs\_jp/mlt?...}
\end{center}

\bigskip

The \MLT\ feature requires some query parameter (usually \texttt{q}) that will return one document (it is possible to have a query that returns more than one document, but the list of ``interesting terms" will be shorter and therefore, the suggested documents will be less relevant). \\

For this reason, we recommend using a query based on the \texttt{id} field. 

For example:

\hfill	\texttt{q=id:"THE BLUE HEARTS -- 歩く花"} \hfill $\leftarrow$ (this need to be URL-encoded)

\bigskip
\bigskip

After that, we need to specify the parameters related to \MLT, the documentation of which can be found here:

\begin{center}
	\url{https://solr.apache.org/guide/solr/latest/query-guide/morelikethis.html}
\end{center}
\bigskip


Below are the parameters that we will focus on:

\begin{itemize}
	\item \emph{mlt.fl}
	
	The field (or list of fields) to use for similarity.
	For better performance, it is advisable that these fields use 
	\texttt{termsVector}s that are \emph{stored}.
	
	This parameter is \emph{required}, but in the configuration (section \ref{enabling-mlt}) we already specified to use the field \texttt{lyrics\_txt\_ja}, so we won't have to specify this parameter every time in the URL.
	
	\item \emph{mintf} \hfill (default: \texttt{2})
	
	The minimum frequency below which terms will be ignored in the source document.
	
	The default is 2 (i.e. repeating terms). 
	
	However, our texts are short, and there usually aren't a lot of repeating terms outside of the chorus.	So we will set this to 1.
	
	\item \emph{mlt.interestingTerms} \hfill (default: \texttt{none})
	
	Shows the top terms used by the \MLT\ query (based on TF-IDF: \emph{term frequency–inverse document frequency}).
	
	Possible values: \texttt{none}, \texttt{list}, \texttt{details}.
	
	
	\item \emph{mlt.boost} \hfill (default: \texttt{false})
	
	Specifies if the query will be boosted by the interesting term relevance.
\end{itemize}

\bigskip


\subsection{Example} \label{mlt-solr-example}

Below is a script that will illustrate how to query the \MLT\ feature: \\

\texttt{get\_mlt.sh}
\lstinputlisting[language=sh]{files/bsd-licensed/solr/get_mlt.sh}
\bigskip

Let's call it on a document of \texttt{id}: ``GReeeeN -- 風" (a song called ``風" -- \emph{kaze} -- ``Wind", by band ``GReeeeN"):

\begin{lstlisting}[language=sh]
./get_mlt.sh 'GReeeeN -- 風'
\end{lstlisting}


\bigskip



The response from Solr will be in JSON format. \\



\newpage

Results: \\

``interesting terms":
\begin{longtable}{|l l l|}
	\hline
	& & \\
	\endhead
	
	\hline
	\endfoot

	そんな & \emph{sonna} & such, that sort of, that kind of \\
未来 & \emph{mirai} & the future (usually distant) \\
どの & \emph{dono} & which, what \\
想い & \emph{omoi} & thought \\
遠い & \emph{toi} & far, distant, far away \\
探す & \emph{sagasu} & to search for, to look for, to hunt for, to seek \\
選べる & \emph{erabu} & to choose, to select \\
どんな & \emph{donna} & what kind of, what sort of \\
清い & \emph{kiyoi} & clear, pure, noble \\
流れる & \emph{nagareru} & to stream, to flow (liquid, time, etc.) \\
行く & \emph{iku} & to go \\
描く & \emph{egaku} & to draw \\
飛ぶ & \emph{tobu} & to fly, to soar \\
新しい & \emph{atarashii} & new \\
ゆく & \emph{iku} & 行く:  to go \\
舞う & \emph{mau} &  to dance \\
まだ & \emph{mada} & still, as yet, only \\
乗せる & \emph{noseru} &  to place on (something) \\
風 & \emph{kaze} & wind \\
花びら & \emph{hanabira} & (flower) petal \\
場所 & \emph{basho} & place, location, spot, position \\
ヒラヒラ & & \\
吹く & \emph{fuku} & to blow (of the wind) \\
私 & \emph{watashi} & I, me (Pronoun, slightly formal or feminine) \\

	& & \\
\end{longtable}


"docs":
\begin{longtable}{|l l|}
	\hline
	\multicolumn{1}{|c}{Band} & \multicolumn{1}{c|}{Title} \\
	\hline
	& \\
	\endhead
	
	\hline
	\endfoot
	
	松任谷由実 & Flying Messenger \\
AKB48 & 風は吹いている \\
fripSide & rain of blossoms \\
ゆず & 歩行者優先 \\
BUMP OF CHICKEN & (please) forgive \\
いきものがかり & スパイス・マジック \\
ゆず & 風に吹かれた \\
fripSide & an evening calm \\
Kiroro & 落書き \\
fripSide & colorless fate -version 2018- \\

	& \\
\end{longtable}

The list of similar songs that Solr found is comprised of: 5 different songs by AKB48, 2 songs by GReeeeN, and 2 songs by Tokunaga Hideaki (徳永英明).

\newpage

\subsection{Usage}

In section \ref{what-is-MLT} we showed what the \MLT\ feature can be used for. \\

The example allowed us to better visualise \MLT\ in action:


\begin{itemize}
	\item the \emph{``interesting terms"} gives us a list of terms that are characteristic of the song, and make it distinctive
	
	\item the \emph{``docs"} list can help us discover the songs whose lyrics have some commonalities with the song being studied. This can help us choose which songs to explore next.
\end{itemize}