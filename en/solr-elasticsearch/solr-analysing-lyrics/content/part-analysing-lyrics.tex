% Copyright 2022 Pierre S. Caboche. All rights reserved.

\renewcommand{\currentPart}{Analysing lyrics}
\part{Analysing lyrics}   \label{analysing-lyrics}




\section{The dataset}

To create the dataset, I searched online for the lyrics from a variety of bands. \\

I tried to include popular bands, as well as a few personal favourites. I also tried to include bands from a variety of genres (Pop, Rock, Metal, Punk, \emph{Enka}\dots). \\

Having such a diversity would hopefully yield interesting results, as the bubbly vocabulary used by a Pop band might differ (at least in theory) from the sombre undertones of a metal band, or the melancholic verses of \emph{enka} songs. \\

Below are some of the bands I included for the study\dots \\


\subsection{Punk-Rock}

\subsubsection{THE BLUE HEARTS} \label{{THE BLUE HEARTS}}

THE BLUE HEARTS is one of my favourite bands, and what inspired me to write this paper (I was looking for the lyrics to some of their songs, and was interested in the writing style of the different band members). \\

\bigskip

THE BLUE HEARTS was a Japanese punk rock band active from 1985 to 1995. The band is very famous in Japan, and their songs have been featured in a lot of media (movies, series, video games, anime\dots) \\

THE BLUE HEARTS was mainly comprised of the following members:
\begin{itemize}
	\item Kōmoto Hiroto --- 甲本 ヒロト --- vocals, harmonica
	\item Mashima Masatoshi --- 真島 昌利 --- guitar, backing vocals
	\item Kawaguchi Junnosuke --- 河口 純之助 --- bass, backing vocals
	\item Kajiwara Tetsuya --- 梶原 徹也 --- drums
	\item Shirai Mikio --- 白井 幹夫 --- keyboards (support member)
\end{itemize}

\bigskip

The lyrics for THE BLUE HEARTS are credited to three of the band's members (Kōmoto, Mashima, and Kawaguchi). \\

Below is a selection of their songs, by author:

\begin{itemize}
	\item Kōmoto Hiroto --- 甲本 ヒロト
	\begin{itemize}
		\item \href{https://www.youtube.com/watch?v=RYC71PAuIKE}{リンダ リンダ} [Linda Linda]
		\item \href{https://www.youtube.com/watch?v=5yslQYshIAI}{ラブレター} [Love Letter]
		\item \href{https://www.youtube.com/watch?v=-p0Yqcx53O8}{情熱の薔薇} [Rose of Passion]
		\item \href{https://www.youtube.com/watch?v=t0rffmN7c6w}{歩く花} [The Walking Flower]
		\item \href{https://www.youtube.com/watch?v=BT-8_TFxohI}{キスしてほしい} [I want you to kiss me]
		\item \href{https://www.youtube.com/watch?v=xnQYtogdxyg}{夕暮れ} [Evening]
		\item 月の爆撃機 [Moon Bomber]
		\item 人にやさしく [Be kind to People]
		\item 1985
		\item 星をください [Give me a Star]
		\item M・O・N・K・E・Y
		\item 少年の詩 [Teenage Boy Poem]
		\item パーティー [PARTY]
	\end{itemize}
	
	\item Mashima Masatoshi --- 真島 昌利	
	\begin{itemize}
		\item \href{https://www.youtube.com/watch?v=BSkLe-EOq5U}{TRAIN-TRAIN}
		\item \href{https://www.youtube.com/watch?v=PQXMU1A8CjI}{青空} [Blue Sky]
		\item \href{https://www.youtube.com/watch?v=9yY5lwBT4mk}{TOO MUCH PAIN}
		\item \href{https://www.youtube.com/watch?v=egNok6oeMA0}{1000のバイオリン} [1000 violins]
		\item 夢 [Dream]
		\item 君のため [For you]
		\item 終わらない歌 [A never-ending Song]
		\item 僕はここに立っているよ [I'm standing here]	
	\end{itemize}
	
	\item Kawaguchi Junnosuke --- 河口 純之助	
	\begin{itemize}
		\item \href{https://www.youtube.com/watch?v=J4LKAFe82_w}{シンデレラ(灰の中から)} [Cinderella (from the ashes)]
		\item 心の救急車 [Ambulance of the Heart]
		\item 真夜中のテレフォン [Midnight Telephone]
	\end{itemize}
\end{itemize}

\bigskip


When the band broke up in 1995, Kōmoto Hiroto and Mashima Masatoshi went on to form a new band called ``The High-Lows", and then another band called ``The Cro-Magnons" after that (see~\ref{the-high-lows} and~\ref{the-cro-magnons}). \\


\bigskip

This will give us the opportunity to perform different types of queries:
\begin{itemize}
	\item most used words in songs by THE BLUE HEARTS (section~\ref{most-used-words-BLUE-HEARTS})
	
	\item most used words by the different songwriters throughout their career (not just as part of THE BLUE HEARTS), in section~\ref{most-used-words-by-songwriter}
\end{itemize}


\bigskip

After the band broke up, Kōmoto Hiroto and Mashima Masatoshi chose not to sing any of the songs from THE BLUE HEARTS, with a few occasional exceptions. \\

\bigskip

\subsubsection{The High-Lows (ザ・ハイロウズ)} \label{the-high-lows}

A Japanese punk rock band that was formed in 1995 by former members of THE BLUE HEARTS, Kōmoto Hiroto and Mashima Masatoshi (section~\ref{{THE BLUE HEARTS}}). \\

The band broke up in 2005. \\

Kōmoto and Mashima went on to form ``The Cro-Magnons".


\bigskip

\subsubsection{The Cro-Magnons (ザ・クロマニヨンズ)} \label{the-cro-magnons}

A Japanese punk rock band that was formed in 2006 by former members of THE BLUE HEARTS and ``The High-Lows", Kōmoto Hiroto and Mashima Masatoshi (section~\ref{{THE BLUE HEARTS}}). \\


\bigskip


\subsection{Pop}

\subsubsection{AKB48}

AKB48 (pronounced A.K.B. Forty-Eight) is an idol girl group named after the Akihabara (Akiba for short) area in Tokyo. \\

The band is immensely successful, and has sprouted multiple sister bands, both domestically (NMB48, HKT48, NGT48\dots) and abroad (JKT48, BNK48, MNL48, SNH48\dots) \\

The overall musical style of AKB48 is very girly, and sometimes flirtatious. \\

That being said, some of their songs really stand out:

\begin{itemize}
	\item \href{https://www.youtube.com/watch?v=fgESmgZ4ld8}{風は吹いている} [The Wind is Blowing] 
	\item \href{https://www.youtube.com/watch?v=JUbU6VLV6yI}{365日の紙飛行機} [365 Days Paper Plane]
	\item \href{https://www.youtube.com/watch?v=hUOc1yc8cH0}{桜の栞} [Cherry Blossom Bookmark]
	\item \href{https://www.youtube.com/watch?v=n24zp_wPC7E}{前しか向かねえ} [Only move forward]	
\end{itemize}





\bigskip
\subsection{Enka}

Enka (演歌) is a music genre typical from Japan. \\

Musically speaking, \emph{enka} resembles traditional Japanese music. In terms of lyrics, \emph{enka} songs include recurring themes such as: love, loss, loneliness, hardships, etc. to the point that it can be described as some sort of \emph{``Japanese blues"}. \\

\emph{Enka} songs are also often associated with a place and a season. \\

I've included songs from the following \emph{enka} singers in the dataset:

\subsubsection{Ishikawa Sayuri (石川さゆり)}

Born in 1958, Ishikawa Sayuri is a \emph{enka} singer who made her professional debut in 1973. She is one of the most-recognized and successful \emph{enka} singers in history. \\

Ishikawa's most popular songs include: 
\href{https://www.youtube.com/watch?v=OJxm9Lt-w6Y}{\furi{津軽/つがる,海峡/かいきょう, ・ /,冬/ふゆ,景色/けしき}} [Tsugaru Straight - Winter Scenery] (1977), and
\href{https://www.youtube.com/watch?v=yvc0LadtZUk}{\furi{天城/あまぎ,越/ご,え/}} [Walk Over the Amagi Pass] (1986).


\subsubsection{Miyako Harumi (都はるみ)}

Born in 1948, Miyako Harumi (née Kitamura Harumi) is a singer who made her debut in 1964 and is still active today, making television frequent appearances. \\

Although usually considered as an \emph{enka} singer, Miyako Harumi never thought of herself as such, arguing that the term \emph{``enka"} was not used in that context at the time of her début. \\

Popular song by Miyako Harumi include: \href{https://www.youtube.com/watch?v=QKmma_bRdQE}{\furi{北/きた,の/,宿/やど,から/}} [From an Inn in the North] (1975), and \href{https://www.youtube.com/watch?v=pltL1kuIU4}{\furi{大阪/おおさか,しぐれ/}} [Autumn Rain in Osaka] (1980). \\


\bigskip



\subsection{Metal}


\subsubsection{Ningen Isu (人間椅子)}

Ningen Isu (人間椅子 --- \emph{"The Human Chair"}) is a heavy metal band formed in 1987, and active ever since. The band is named after a short story by Edogawa Rampo, published in 1925. \\

The band is known for using some difficult and old words (from the Edo to Showa period, i.e. 1603 onwards), and the themes of their songs include: hell, the universe, Buddhism, etc. \\

For all these reasons, analysing the lyrics of their songs should be interesting (section \ref{most-used-words-Ningen-Isu}). \\

Personal favourites by Ningen Isu include:

\begin{itemize}
	\item \href{https://www.youtube.com/watch?v=CbI79e5iZKs}{\furi{無情/むじょう}のスキャット} [Heartless Scat]
	
	\item \href{https://www.youtube.com/watch?v=tKSjWKDSBmo}{杜子春} [Toshishun]
	
	\item \href{https://www.youtube.com/watch?v=CLoUY1kA4ZY}{なまはげ} [Namahage]
\end{itemize}

\bigskip

\subsubsection{BABYMETAL}

BABYMETAL (ベビーメタル) is a band often credited with the creation and success of a type of metal called ``kawaii metal" (\emph{cute metal}), a genre that combines heavy metal music and J-pop melodies. \\

The band formed in 2010. In October of 2021, the band released a cryptic video hinting a hiatus (or maybe the end of the band). \\

Some of their best songs include:

\begin{itemize}
	\item \href{https://www.youtube.com/watch?v=pDgqo6fcliY}{NO RAIN, NO RAINBOW}

	\item  \href{https://www.youtube.com/watch?v=nDqaTXqCN-Q}{イジメ、ダメ、ゼッタイ} [Bully, Never, Ever]
	
	\item \href{https://www.youtube.com/watch?v=WIKqgE4BwAY}{ギミチョコ!! --- Gimme chocolate!!} 
		
	\item \href{https://www.youtube.com/watch?v=oO7Y8NsnkRg}{PA PA YA!! (feat. F.HERO)}
	
	\item \href{https://www.youtube.com/watch?v=cK3NMZAUKGw}{メギツネ - MEGITSUNE}
	
	\item 紅月 --- アカツキ [Red moon]
\end{itemize}

\bigskip




\subsection{Other}

The dataset contains many more bands, and it would take too long to describe them all (the purpose of this section is to give you an idea of the type of band you will find in the dataset, and the reasons why some of the bands were added). \\

The list of bands in the dataset (and the number of songs for each band) can be found on page~\pageref*{query-list-of-bands} in section~\emph{\longref{query-list-of-bands}} \\






\newpage
\section{Exploring the dataset} \label{exploring-dataset}

\subsection{Queries and results}

Solr returns results in JSON format. \\

In this article, we show only \emph{part} of the results returned by Solr, in a more readable form (e.g. in textual form, as a table). \\

We also add some extra information (such as pronunciations, and common English translations) for convenience to the reader. \\

\bigskip

Below are the descriptions of the column names:

\begin{longtable}{l p{11.5cm}}


\emph{"numFound"} :& The total number of documents (i.e. songs) in the result-set. \\
& \\
	
\emph{Band} :& The band (バンド), or artist name.\\
& \\
&
This is the key returned when performing some \faceting\ on field: \texttt{band\_str}.
\\

& \\


\emph{Songwriter} :& The songwriter. (\furi{作詞家/さくしか})\\
& \\
&
This is the key returned when performing some \faceting\ on field: \texttt{lyrics\_by\_str}. 
(section \ref{songwriters-TBH}) 
\\



& \\


\emph{Lexeme}: &
Lexeme (\furi{語彙素/ごいそ})
\\


& 
A Japanese word, as it would appear in its dictionary form (i.e. without \emph{inflection})
\\
& \\

\emph{何曲}: &  \furi{何/なん,曲/きょく} -- number of songs\\


& \emph{nankyoku} -- \emph{``How many songs?", ``What songs?"} \\

& \\
&
This column represents the number of songs for each item in the result-set.  \newline
This is the value returned by Solr when performing some \faceting\ (on the fields like: \texttt{band\_str}, \texttt{lyrics\_by\_str}, \texttt{lyrics\_txt\_ja}), as we will see in the rest of this section. \\

%(何曲 -- \emph{nankyoku} -- \emph{``How many songs?", ``What songs?"})

& \\

\emph{Pronunc.}: 
&
\emph{(not part of the result-set returned by Solr; this column is been added for the convenience to the reader)}
\\
& \\
& Shows the most common pronunciations for a particular \emph{lexeme} (some words may have more than one pronunciation), in \emph{romaji}. \\

& \\

\emph{Meanings}:
&
\emph{(not part of the result-set returned by Solr; this column is been added for the convenience to the reader)}
\\
& \\
& Shows the most common meanings or English translations for a particular \emph{lexeme}
\\


\end{longtable}

\newpage

\subsection{General queries}  \label{general-queries}




\subsubsection{List of bands (whole dataset)} \label{query-list-of-bands}

Question: \\
\emph{``Which bands (or solo artists) are in the dataset? (and how many songs do they have?)"} \\


Query:

\begin{tabular}{|l|l|}
	\hline
	& \\
	rows: & 0 \\
	facet: & true \\
	facet.field: & band\_str \\
	& \\
	\hline
\end{tabular}



\bigskip
Result: \\

"numFound":
\input{content/results/numFound.tex}

\begin{longtable}{|l l|r|}
	\hline
	\multicolumn{2}{|c|}{Band (バンド)} & 
	\multicolumn{1}{|c|}{何曲}
	\\
	\hline
	& & \\
	\endhead
	
	\hline
	\endfoot
	
	松任谷由実 & \emph{Matsutōya Yumi} & 417 \\
AKB48 & & 376 \\
B'z & & 354 \\
SMAP & & 336 \\
徳永英明 & \emph{Tokunaga Hideaki} & 304 \\
ゆず & \emph{Yuzu} & 292 \\
JAM Project & & 264 \\
サザンオールスターズ & \emph{Southern All Stars} & 257 \\
aiko & & 256 \\
石川さゆり & \emph{Ishikawa Sayuri} & 249 \\
Mr.Children & & 244 \\
DREAMS COME TRUE & & 243 \\
ザ・クロマニヨンズ & \emph{The Cro-Magnons} & 216 \\
ORANGE RANGE & & 190 \\
BEGIN & & 189 \\
RADWIMPS & & 177 \\
GReeeeN & & 163 \\
いきものがかり & \emph{Ikimonogakari} & 152 \\
LiSA & & 150 \\
L'Arc~en~Ciel & & 149 \\
DA PUMP & & 145 \\
都はるみ & \emph{Miyako Harumi} & 140 \\
人間椅子 & \emph{Ningen Isu} & 137 \\
ザ・ハイロウズ & \emph{The High-Lows} & 135 \\
ゴールデンボンバー & \emph{Golden Bomber} & 133 \\
BUMP OF CHICKEN & & 132 \\
Aimer & & 131 \\
fripSide & & 131 \\
Superfly & & 130 \\
ONE OK ROCK & & 129 \\
GACKT & & 125 \\
MAN WITH A MISSION & & 110 \\
back number & & 109 \\
Kiroro & & 102 \\
THE BLUE HEARTS & & 101 \\
米津玄師 & \emph{Younezu Kenshi} & 92 \\
MONGOL800 & & 76 \\
河島英五 & \emph{Kawashima Eigo} & 62 \\
カネコアヤノ & \emph{Kaneko Ayano} & 61 \\
BABYMETAL & & 40 \\
Xmas Eileen & & 26 \\
YOASOBI & & 23 \\
Ado & & 15 \\

	& & \\
\end{longtable}

\bigskip



\subsubsection{Most used words (whole dataset)} \label{most-used-words-dataset}

Question: \\
\emph{``Which words appear the most in our dataset?"} \\

Query:

\begin{tabular}{| l | l |}
	\hline
	& \\
	start, rows: & 0, 0 \\
	facet: & true \\
	facet.field: & lyrics\_txt\_ja \\
	%	facet.mincount: & 1 \\
	& \\
	\hline
\end{tabular}




\bigskip
Result: \\




\begin{myLongTable}{Most used words in songs stored in our dataset}
	人 & \emph{hito} & person & 2800 \\
今 & \emph{ima} & now & 2729 \\
君 & \emph{kimi} & you, buddy, pal & 2574 \\
何 & \emph{nani} &  what & 2540 \\
心 & \emph{kokoro} & heart, mind, spirit& 2296 \\
誰 & \emph{dare} & who & 2238 \\
僕 & \emph{boku} &  I, me (Pronoun, Male term) & 2185 \\
夢 & \emph{yume} & dream & 2150 \\
そう & \emph{sou} & looking like & 2143 \\
中 & \emph{naka} & inside & 1998 \\
日 & \emph{hi, nichi} & day & 1949 \\
いい & \emph{ii} & good, excellent, fine, nice & 1915 \\
もう & \emph{mou} & already, yet, by now& 1890 \\
時 & \emph{toki} & time, hour, moment & 1874 \\
見る & \emph{miru} & to see & 1873 \\
あなた & \emph{anata} & you & 1850 \\
あの & \emph{ano} & that, those, the & 1747 \\
手 & \emph{te} & hand, arm & 1706 \\
空 & \emph{sora} & sky & 1615 \\
行く & \emph{iku} & to go & 1571 \\
胸 & \emph{mune} & chest, breast, heart & 1569 \\
知る & \emph{shiru} &  to be aware of, to know, to be conscious of & 1497 \\
風 & \emph{kaze} & wind & 1490 \\
愛 & \emph{ai} & love & 1473 \\
涙 & \emph{namida} & tears & 1457 \\
まま & \emph{mama} & as it is, as one likes & 1434 \\
目 & \emph{me} & eye & 1426 \\
くれる & \emph{kureru} & to give, to let (one) have & 1415 \\
夜 & \emph{yoru} & evening, night & 1407 \\
忘れる & \emph{wasureru} & to forget & 1397 \\
明日 & \emph{ashita, asu} & tomorrow, (only "asu") the near future & 1395 \\
生きる & \emph{ikiru} & to live, to exist & 1389 \\
一 & \emph{ichi} & one & 1384 \\
どこ & \emph{doko} & where, what place & 1366 \\
言う & \emph{iu} & to say & 1364 \\
笑う & \emph{warau} & to laugh & 1350 \\
く & \emph{ku} & ward, borough, city (in Tokyo) & 1346 \\
声 & \emph{koe} & voice & 1335 \\
言葉 & \emph{kotoba} & word, phrase& 1325 \\
今日 & \emph{kyou} & today & 1324 \\
ゆく & \emph{iku} & 行く:  to go & 1282 \\
くる & \emph{kuru} & 来:  to come & 1270 \\
世界 & \emph{sekai} & the world, society, the universe & 1257 \\
いつ & \emph{itsu} & when, at what time & 1253 \\
いつも & \emph{itsumo} & always, all the time, at all times & 1245 \\
来る & \emph{kuru} & to come & 1232 \\
いく & \emph{iku} & 行く:  to go & 1226 \\
きっと & \emph{kitto} & surely, undoubtedly, almost certainly & 1216 \\
泣く & \emph{naku} & to cry, to weep, to sob & 1176 \\
見える & \emph{mieru} & to be seen, to be in sight, to appear & 1164 \\
you & \emph{-} & - & 1152 \\
自分 & \emph{jibun} & I, me, myself, yourself, oneself (Pronoun) & 1136 \\
私 & \emph{watashi} & I, me (Pronoun, slightly formal or feminine) & 1130 \\
i & \emph{-} & - & 1113 \\
前 & \emph{mae} & in front (of), before & 1097 \\
消える & \emph{kieru} & to go out, to vanish, to disappear & 1096 \\
しまう & \emph{shimau} & to finish, to stop, to put an end to & 1080 \\
いつか & \emph{itsuka} & sometime, someday, one day & 1078 \\
歩く & \emph{aruku} & to walk & 1065 \\
そんな & \emph{sonna} & such, that sort of, that kind of & 1064 \\
恋 & \emph{koi} & (romantic) love & 1060 \\
変わる & \emph{kawaru} & to change, to be transformed, to be altered & 1055 \\
ずっと & \emph{zutto} & continuously, throughout & 1040 \\
事 & \emph{koto} & thing, matter & 1039 \\
未来 & \emph{mirai} & the future (usually distant) & 1009 \\
まだ & \emph{mada} & still, as yet, only & 997 \\
信じる & \emph{shinjiru} & to believe, to place trust in, to have faith in & 994 \\
思う & \emph{omou} & to think, to consider, to believe, to reckon& 972 \\
街 & \emph{gai} & \dots\ street, \dots\ quarter, \dots\ district & 969 \\
気 & \emph{ki} & spirit, mind, heart & 954 \\
二 & \emph{ni} & two & 944 \\
待つ & \emph{matsu} & to wait & 932 \\
みる & \emph{miru} & 見る:  to see & 913 \\
少し & \emph{sukoshi} & a little, a few & 910 \\
好き & \emph{suki} & liked, well-liked, in love (with) & 891 \\
光 & \emph{hikari} & light & 886 \\
道 & \emph{machi} & road, path, street, lane & 886 \\
日々 & \emph{hibi} & the everyday & 861 \\
強い & \emph{tsuyoi} & strong, potent, competent, tough, powerful & 860 \\
僕ら & \emph{bokura} & we (Pronoun, Male term) & 849 \\
想い & \emph{omoi} & thought & 844 \\
the & \emph{-} & - & 836 \\
遠い & \emph{toi} & far, distant, far away & 820 \\
度 & \emph{do} & (counter for occurrences) & 818 \\
わかる & \emph{wakaru} & 分かる:  to understand, to comprehend, to grasp & 804 \\
優しい & \emph{yasashii} & tender, kind, gentle, affectionate & 788 \\
気持ち & \emph{kimochi} & feeling, sensation, mood, state of mind & 786 \\
笑顔 & \emph{egaho} & smiling face, smile & 784 \\
終わる & \emph{owaru} & to end, to come to an end, to close, to finish & 766 \\
続ける & \emph{tsuzukeru} & to continue, to keep up, to keep on & 760 \\
時間 & \emph{jikan} & time & 759 \\
my & \emph{-} & - & 749 \\
顔 & \emph{kao} & face, visage & 735 \\
雨 & \emph{ame} & rain& 728 \\
同じ & \emph{onaji} & same, identical, equal, similar, alike & 725 \\
to & \emph{-} & - & 723 \\
探す & \emph{sagasu} & to search for, to look for, to hunt for, to seek & 711 \\
星 & \emph{hoshi} & star & 710 \\
出る & \emph{deru} & 出る:  to leave, to exit, to go out & 700 \\

	& & & \\
\end{myLongTable}

\bigskip
\bigskip





\subsubsection{List of songwriters (band: ``THE BLUE HEARTS")}  \label{songwriters-TBH}

Question: \\
\emph{``How many different songwriters are there in the band ``THE BLUE HEARTS"? "} \\


Query:

\begin{tabular}{| l |  l |}
	\hline
	& \\
	fq: & band\_str:"THE BLUE HEARTS" \\
	start, rows: & 0, 0 \\
	facet: & true \\
	facet.field: & lyrics\_by\_str \\
	facet.mincount: & 1 \\
	& \\
	\hline
\end{tabular}


\bigskip
Result: \\

"numFound":101

\begin{longtable}{|l l|l|}
	\hline
	\multicolumn{2}{|c|}{Songwriter (作詞家)} & 
	\multicolumn{1}{|c|}{何曲}
	\\
	\hline
	& & \\
	\endhead
	
	\hline
	\endfoot
	
	真島昌利 & Mashima Masatoshi & 50 \\
	甲本ヒロト & Kōmoto Hiroto & 45 \\
	河口純之助 & Kawaguchi Junnosuke & 5 \\
	甲本ヒロト・真島昌利 & Kōmoto Hiroto ・ & 1 \\
	& Mashima Masatoshi& \\
	& & \\
\end{longtable}

\bigskip






\newpage

\subsection{Most used words, by band} \label{most-used-words-by-band}

In this section, we are going to find the most used for the following artists:

\begin{itemize}
	\item THE BLUE HEARTS
	\item AKB48
	\item SMAP
	\item Ningen isu
	\item Kaneko Ayano
\end{itemize}

\bigskip


\subsubsection{Most used words (band: ``THE BLUE HEARTS")} \label{most-used-words-BLUE-HEARTS}

Question: \\
\emph{``Which words appear most often in songs by the band ``THE BLUE HEARTS"? "} \\

Query:

\begin{tabular}{| l |  l |}
	\hline
	& \\
	fq: & band\_str:"THE BLUE HEARTS" \\
	start, rows: & 0, 0 \\
	facet: & true \\
	facet.field: & lyrics\_txt\_ja \\
	%	facet.mincount: & 1 \\
	& \\
	\hline
\end{tabular}


\bigskip
Result: \\

"numFound":101

\begin{myLongTable}{Most used words in songs by the band ``THE BLUE HEARTS"}
	僕 & \emph{boku} &  I, me (Pronoun, Male term) & 42 \\
誰 & \emph{dare} & who & 36 \\
何 & \emph{nani} &  what & 33 \\
事 & \emph{koto} & thing, matter & 29 \\
人 & \emph{hito} & person & 29 \\
行く & \emph{iku} & to go & 25 \\
夜 & \emph{yoru} & evening, night & 24 \\
見る & \emph{miru} & to see & 24 \\
風 & \emph{kaze} & wind & 24 \\
今 & \emph{ima} & now & 22 \\
いい & \emph{ii} & good, excellent, fine, nice & 21 \\
中 & \emph{naka} & inside & 21 \\
やる & \emph{yaru} & to do, to undertake, to perform & 19 \\
笑う & \emph{warau} & to laugh & 19 \\
いく & \emph{iku} & 行く:  to go & 17 \\
どこ & \emph{doko} & where, what place & 17 \\
君 & \emph{kimi} & you, buddy, pal & 17 \\
時 & \emph{toki} & time, hour, moment & 17 \\
死ぬ & \emph{shinu} & to die & 17 \\
くる & \emph{kuru} & 来:  to come & 16 \\
そう & \emph{sou} & looking like & 16 \\
今日 & \emph{kyou} & today & 16 \\
いつ & \emph{itsu} & when, at what time & 15 \\
夢 & \emph{yume} & dream & 15 \\
一 & \emph{ichi} & one & 14 \\
言う & \emph{iu} & to say & 14 \\
そんな & \emph{sonna} & such, that sort of, that kind of & 13 \\
声 & \emph{koe} & voice & 13 \\
手 & \emph{te} & hand, arm & 13 \\
星 & \emph{hoshi} & star & 13 \\
空 & \emph{sora} & sky & 13 \\
雨 & \emph{ame} & rain& 13 \\
もう & \emph{mou} & already, yet, by now& 12 \\
心 & \emph{kokoro} & heart, mind, spirit& 12 \\
聞く & \emph{kiku} & to hear& 12 \\
言葉 & \emph{kotoba} & word, phrase& 12 \\
あげる & \emph{ageru} & 上げる:  to raise, to elevate & 11 \\
あなた & \emph{anata} & you & 11 \\
しまう & \emph{shimau} & to finish, to stop, to put an end to & 11 \\
上 & \emph{ue} & above, up, over & 11 \\
世界 & \emph{sekai} & the world, society, the universe & 11 \\
前 & \emph{mae} & in front (of), before & 11 \\
忘れる & \emph{wasureru} & to forget & 11 \\
明日 & \emph{ashita, asu} & tomorrow, (only "asu") the near future & 11 \\
気 & \emph{ki} & spirit, mind, heart & 11 \\
涙 & \emph{namida} & tears & 11 \\
生きる & \emph{ikiru} & to live, to exist & 11 \\
街 & \emph{gai} & \dots\ street, \dots\ quarter, \dots\ district & 11 \\
見える & \emph{mieru} & to be seen, to be in sight, to appear & 11 \\
遠い & \emph{toi} & far, distant, far away & 11 \\
みんな & \emph{minna} & everyone, everybody, all & 10 \\
俺 & \emph{ore} & I, me (Pronoun, Male term, rough or arrogant) & 10 \\
吹く & \emph{fuku} & to blow (of the wind) & 10 \\
日 & \emph{hi, nichi} & day & 10 \\
気持ち & \emph{kimochi} & feeling, sensation, mood, state of mind & 10 \\
泣く & \emph{naku} & to cry, to weep, to sob & 10 \\
目 & \emph{me} & eye & 10 \\
あの & \emph{ano} & that, those, the & 9 \\
つける & \emph{tsukeru} &  & 9 \\
どう & \emph{dou} & how, in what way, how about & 9 \\
まま & \emph{mama} & as it is, as one likes & 9 \\
みたい & \emph{mitai} & -like, sort of, similar to, resembling & 9 \\
本当 & \emph{hontou} & truth, reality, actuality, fact & 9 \\
生まれる & \emph{umareru} & to be born & 9 \\
道 & \emph{machi} & road, path, street, lane & 9 \\
く & \emph{ku} & ward, borough, city (in Tokyo) & 8 \\
くれる & \emph{kureru} & to give, to let (one) have & 8 \\
すぎる & \emph{sugiru} & to pass through, to pass by & 8 \\
つく & \emph{tsuku} & 着く:  to arrive at, to reach & 8 \\
でる & \emph{deru} & 出る:  to leave, to exit, to go out & 8 \\
なれる & \emph{nareru} & 慣れる:  to get used to & 8 \\
ゆく & \emph{iku} & 行く:  to go & 8 \\
夏 & \emph{natsu} & summer & 8 \\
悪い & \emph{warui} & bad, poor, undesirable & 8 \\
来る & \emph{kuru} & to come & 8 \\
欲しい & \emph{hoshii} & wanted, wished for, in need of, desired & 8 \\
歌う & \emph{utau} & to sing & 8 \\
者 & \emph{mono, sha} & person & 8 \\
達 & \emph{tachi} & (pluralizing suffix) & 8 \\
降る & \emph{furu} & to fall (of rain, snow, ash, etc.), to come down & 8 \\
飛ばす & \emph{tobasu} & to let fly, to make fly, to skip over, to leave out & 8 \\
ああ & \emph{aa} & ah!, oh!, alas! & 7 \\
いける & \emph{ikeru} &  & 7 \\
かける & \emph{kakeru} & & 7 \\
ずっと & \emph{zutto} & continuously, throughout & 7 \\
たくさん & \emph{takusan} & a lot, lots, plenty, many, a large number, much & 7 \\
みる & \emph{miru} & 見る:  to see & 7 \\
下 & \emph{shita} & below, down, under, & 7 \\
世の中 & \emph{yononaka} & society, the world, the times & 7 \\
今夜 & \emph{konya} & this evening, tonight & 7 \\
奴 & \emph{yatsu} & fellow, guy, chap (Derogatory) & 7 \\
少し & \emph{sukoshi} & a little, a few & 7 \\
思い出 & \emph{omoide} & memories, recollections, reminiscence & 7 \\
方 & \emph{hou} & direction, way, side & 7 \\
月 & \emph{tsuki} & moon & 7 \\
楽しい & \emph{tanoshii} & enjoyable, fun, pleasant & 7 \\
歴史 & \emph{rekishi} & history & 7 \\
続く & \emph{tsuzuku} & to continue, to last, to go on & 7 \\
言える & \emph{ieru} & to be possible to say, to be able to say & 7 \\

	& & & \\
\end{myLongTable}


\bigskip
\bigskip




\subsubsection{Most used words (band: ``AKB48")}

Question: \\
\emph{``Which words appear most often in songs by the band ``AKB48"? "} \\


\begin{tabular}{| l |  l |}
	\hline
	& \\
	fq: & band\_str:"AKB48" \\
	start, rows: & 0, 0 \\
	facet: & true \\
	facet.field: & lyrics\_by\_str \\
	%	facet.mincount: & 1 \\
	& \\
	\hline
\end{tabular}



\bigskip
Result: \\

"numFound":376

\begin{myLongTable}{Most used words in songs by the band ``AKB48"}
	人 & \emph{hito} & person & 216 \\
誰 & \emph{dare} & who & 204 \\
何 & \emph{nani} &  what & 191 \\
私 & \emph{watashi} & I, me (Pronoun, slightly formal or feminine) & 177 \\
今 & \emph{ima} & now & 167 \\
夢 & \emph{yume} & dream & 160 \\
愛 & \emph{ai} & love & 146 \\
どこ & \emph{doko} & where, what place & 145 \\
来る & \emph{kuru} & to come & 140 \\
心 & \emph{kokoro} & heart, mind, spirit& 137 \\
そう & \emph{sou} & looking like & 136 \\
いい & \emph{ii} & good, excellent, fine, nice & 133 \\
一 & \emph{ichi} & one & 128 \\
見る & \emph{miru} & to see & 128 \\
中 & \emph{naka} & inside & 124 \\
日 & \emph{hi, nichi} & day & 123 \\
自分 & \emph{jibun} & I, me, myself, yourself, oneself (Pronoun) & 122 \\
行く & \emph{iku} & to go & 119 \\
恋 & \emph{koi} & (romantic) love & 117 \\
あなた & \emph{anata} & you & 116 \\
風 & \emph{kaze} & wind & 111 \\
君 & \emph{kimi} & you, buddy, pal & 109 \\
空 & \emph{sora} & sky & 108 \\
時 & \emph{toki} & time, hour, moment & 107 \\
僕 & \emph{boku} &  I, me (Pronoun, Male term) & 103 \\
ずっと & \emph{zutto} & continuously, throughout & 100 \\
未来 & \emph{mirai} & the future (usually distant) & 97 \\
目 & \emph{me} & eye & 96 \\
言う & \emph{iu} & to say & 95 \\
涙 & \emph{namida} & tears & 93 \\
あの & \emph{ano} & that, those, the & 90 \\
忘れる & \emph{wasureru} & to forget & 90 \\
気づく & \emph{kizuku} & to notice, to recognise, to become aware of & 90 \\
道 & \emph{machi} & road, path, street, lane & 90 \\
いつ & \emph{itsu} & when, at what time & 89 \\
前 & \emph{mae} & in front (of), before & 88 \\
好き & \emph{suki} & liked, well-liked, in love (with) & 88 \\
生きる & \emph{ikiru} & to live, to exist & 88 \\
手 & \emph{te} & hand, arm & 87 \\
くれる & \emph{kureru} & to give, to let (one) have & 86 \\
わかる & \emph{wakaru} & 分かる:  to understand, to comprehend, to grasp & 85 \\
今日 & \emph{kyou} & today & 85 \\
信じる & \emph{shinjiru} & to believe, to place trust in, to have faith in & 85 \\
知る & \emph{shiru} &  to be aware of, to know, to be conscious of & 85 \\
胸 & \emph{mune} & chest, breast, heart & 84 \\
まま & \emph{mama} & as it is, as one likes & 82 \\
しまう & \emph{shimau} & to finish, to stop, to put an end to & 80 \\
歩く & \emph{aruku} & to walk & 78 \\
いつも & \emph{itsumo} & always, all the time, at all times & 76 \\
きっと & \emph{kitto} & surely, undoubtedly, almost certainly & 76 \\
いつか & \emph{itsuka} & sometime, someday, one day & 73 \\
思う & \emph{omou} & to think, to consider, to believe, to reckon& 73 \\
見える & \emph{mieru} & to be seen, to be in sight, to appear & 73 \\
すべて & \emph{subete} & 全て:  everything, all, the whole & 72 \\
世界 & \emph{sekai} & the world, society, the universe & 71 \\
声 & \emph{koe} & voice & 69 \\
待つ & \emph{matsu} & to wait & 68 \\
2 & & & 66 \\
度 & \emph{do} & (counter for occurrences) & 66 \\
夜 & \emph{yoru} & evening, night & 65 \\
どんな & \emph{donna} & what kind of, what sort of & 64 \\
みたい & \emph{mitai} & -like, sort of, similar to, resembling & 64 \\
みんな & \emph{minna} & everyone, everybody, all & 63 \\
愛しい & \emph{itoshii} & lovely, dear, beloved, darling, dearest & 63 \\
なれる & \emph{nareru} & 慣れる:  to get used to & 62 \\
欲しい & \emph{hoshii} & wanted, wished for, in need of, desired & 61 \\
そんな & \emph{sonna} & such, that sort of, that kind of & 60 \\
もう & \emph{mou} & already, yet, by now& 60 \\
大人 & \emph{otona} & adult, grown-up & 60 \\
みる & \emph{miru} & 見る:  to see & 58 \\
合う & \emph{au} & to come together, to merge, to unite, to meet & 58 \\
場所 & \emph{basho} & place, location, spot, position & 58 \\
思い出 & \emph{omoide} & memories, recollections, reminiscence & 58 \\
明日 & \emph{ashita, asu} & tomorrow, (only "asu") the near future & 58 \\
すぎる & \emph{sugiru} & to pass through, to pass by & 57 \\
変わる & \emph{kawaru} & to change, to be transformed, to be altered & 57 \\
気持ち & \emph{kimochi} & feeling, sensation, mood, state of mind & 57 \\
なぜ & \emph{naze} & why, how & 55 \\
先 & \emph{saki} & head (of a line), front & 53 \\
友達 & \emph{tomodachi} & friend, companion & 53 \\
気 & \emph{ki} & spirit, mind, heart & 53 \\
言葉 & \emph{kotoba} & word, phrase& 53 \\
ひとつ & \emph{hitotsu} & one (counter) & 52 \\
出す & \emph{dasu} & to take out, to get out & 52 \\
時間 & \emph{jikan} & time & 52 \\
やさしい & \emph{yasashii} & 優しい:  tender, kind, gentle, affectionate & 51 \\
あきらめる & \emph{akirameru} & 諦める:  to give up & 49 \\
く & \emph{ku} & ward, borough, city (in Tokyo) & 49 \\
しあわせ & \emph{shiawase} & 幸せ:  happiness, good fortune, luck, blessing & 49 \\
街 & \emph{gai} & \dots\ street, \dots\ quarter, \dots\ district & 49 \\
もっと & \emph{motto} & (some) more, even more, longer, further & 48 \\
やる & \emph{yaru} & to do, to undertake, to perform & 48 \\
長い & \emph{nagai} & long (distance, length), long (time), prolonged & 47 \\
すぐ & \emph{sugu} & immediately, at once, right away, directly & 46 \\
キス & \emph{kissu} & kiss & 46 \\
探す & \emph{sagasu} & to search for, to look for, to hunt for, to seek & 46 \\
どう & \emph{dou} & how, in what way, how about & 45 \\
太陽 & \emph{taiyou} & sun & 45 \\
教える & \emph{oshieru} & to teach, to instruct & 45 \\

	& & & \\
\end{myLongTable}







\bigskip
\subsubsection{Most used words (band: ``SMAP")}

Question: \\
\emph{``Which words appear most often in songs by the band ``SMAP"? "} \\


\begin{tabular}{| l |  l |}
	\hline
	& \\
	fq: & band\_str:"SMAP" \\
	start, rows: & 0, 0 \\
	facet: & true \\
	facet.field: & lyrics\_by\_str \\
	%	facet.mincount: & 1 \\
	& \\
	\hline
\end{tabular}

\bigskip
Result: \\

"numFound":336

\begin{myLongTable}{Most used words in songs by ``Kaneko Ayano"}
	君 & \emph{kimi} & you, buddy, pal & 201 \\
僕 & \emph{boku} &  I, me (Pronoun, Male term) & 159 \\
人 & \emph{hito} & person & 147 \\
そう & \emph{sou} & looking like & 139 \\
誰 & \emph{dare} & who & 114 \\
何 & \emph{nani} &  what & 113 \\
今 & \emph{ima} & now & 111 \\
いい & \emph{ii} & good, excellent, fine, nice & 107 \\
時 & \emph{toki} & time, hour, moment & 98 \\
中 & \emph{naka} & inside & 95 \\
夢 & \emph{yume} & dream & 95 \\
心 & \emph{kokoro} & heart, mind, spirit& 95 \\
見る & \emph{miru} & to see & 94 \\
日 & \emph{hi, nichi} & day & 93 \\
胸 & \emph{mune} & chest, breast, heart & 89 \\
いつも & \emph{itsumo} & always, all the time, at all times & 81 \\
きっと & \emph{kitto} & surely, undoubtedly, almost certainly & 80 \\
もう & \emph{mou} & already, yet, by now& 79 \\
気持ち & \emph{kimochi} & feeling, sensation, mood, state of mind & 79 \\
空 & \emph{sora} & sky & 78 \\
夜 & \emph{yoru} & evening, night & 77 \\
言う & \emph{iu} & to say & 77 \\
you & \emph{-} & - & 75 \\
く & \emph{ku} & ward, borough, city (in Tokyo) & 75 \\
いつ & \emph{itsu} & when, at what time & 74 \\
行く & \emph{iku} & to go & 73 \\
手 & \emph{te} & hand, arm & 72 \\
生きる & \emph{ikiru} & to live, to exist & 72 \\
あの & \emph{ano} & that, those, the & 71 \\
明日 & \emph{ashita, asu} & tomorrow, (only "asu") the near future & 71 \\
愛 & \emph{ai} & love & 70 \\
まま & \emph{mama} & as it is, as one likes & 69 \\
一 & \emph{ichi} & one & 68 \\
恋 & \emph{koi} & (romantic) love & 68 \\
そんな & \emph{sonna} & such, that sort of, that kind of & 67 \\
今日 & \emph{kyou} & today & 67 \\
笑う & \emph{warau} & to laugh & 67 \\
くる & \emph{kuru} & 来:  to come & 66 \\
二 & \emph{ni} & two & 66 \\
風 & \emph{kaze} & wind & 66 \\
どこ & \emph{doko} & where, what place & 63 \\
忘れる & \emph{wasureru} & to forget & 63 \\
ずっと & \emph{zutto} & continuously, throughout & 62 \\
僕ら & \emph{bokura} & we (Pronoun, Male term) & 61 \\
街 & \emph{gai} & \dots\ street, \dots\ quarter, \dots\ district & 61 \\
涙 & \emph{namida} & tears & 59 \\
目 & \emph{me} & eye & 59 \\
知る & \emph{shiru} &  to be aware of, to know, to be conscious of & 59 \\
ゆく & \emph{iku} & 行く:  to go & 57 \\
自分 & \emph{jibun} & I, me, myself, yourself, oneself (Pronoun) & 57 \\
声 & \emph{koe} & voice & 56 \\
気 & \emph{ki} & spirit, mind, heart & 56 \\
love & \emph{-} & - & 54 \\
変わる & \emph{kawaru} & to change, to be transformed, to be altered & 54 \\
泣く & \emph{naku} & to cry, to weep, to sob & 54 \\
笑顔 & \emph{egaho} & smiling face, smile & 54 \\
言葉 & \emph{kotoba} & word, phrase& 54 \\
少し & \emph{sukoshi} & a little, a few & 53 \\
歩く & \emph{aruku} & to walk & 53 \\
いつか & \emph{itsuka} & sometime, someday, one day & 52 \\
信じる & \emph{shinjiru} & to believe, to place trust in, to have faith in & 52 \\
未来 & \emph{mirai} & the future (usually distant) & 52 \\
いく & \emph{iku} & 行く:  to go & 51 \\
すぐ & \emph{sugu} & immediately, at once, right away, directly & 51 \\
みんな & \emph{minna} & everyone, everybody, all & 51 \\
感じる & \emph{kanjiru} & to feel & 51 \\
見える & \emph{mieru} & to be seen, to be in sight, to appear & 51 \\
くれる & \emph{kureru} & to give, to let (one) have & 50 \\
みる & \emph{miru} & 見る:  to see & 50 \\
好き & \emph{suki} & liked, well-liked, in love (with) & 50 \\
世界 & \emph{sekai} & the world, society, the universe & 49 \\
顔 & \emph{kao} & face, visage & 49 \\
i & \emph{-} & - & 48 \\
oh & \emph{-} & - & 48 \\
前 & \emph{mae} & in front (of), before & 48 \\
来る & \emph{kuru} & to come & 48 \\
s & \emph{-} & - & 46 \\
思う & \emph{omou} & to think, to consider, to believe, to reckon& 46 \\
まだ & \emph{mada} & still, as yet, only & 45 \\
同じ & \emph{onaji} & same, identical, equal, similar, alike & 45 \\
it & \emph{-} & - & 44 \\
しまう & \emph{shimau} & to finish, to stop, to put an end to & 44 \\
どんな & \emph{donna} & what kind of, what sort of & 43 \\
はず & \emph{hazu} & should (be), bound (to be) & 43 \\
抱きしめる & \emph{dakishimeru} & to hug someone close, to hold someone tight& 43 \\
my & \emph{-} & - & 42 \\
待つ & \emph{matsu} & to wait & 42 \\
on & \emph{-} & - & 41 \\
the & \emph{-} & - & 41 \\
愛す & \emph{aisu} & to love & 40 \\
星 & \emph{hoshi} & star & 40 \\
消える & \emph{kieru} & to go out, to vanish, to disappear & 40 \\
ちょっと & \emph{chotto} & a little, a bit, slightly & 39 \\
場所 & \emph{basho} & place, location, spot, position & 39 \\
探す & \emph{sagasu} & to search for, to look for, to hunt for, to seek & 39 \\
日々 & \emph{hibi} & the everyday & 39 \\
続ける & \emph{tsuzukeru} & to continue, to keep up, to keep on & 39 \\
雨 & \emph{ame} & rain& 39 \\
a & \emph{-} & - & 38 \\

	& & & \\
\end{myLongTable}


\bigskip
\bigskip


\newpage




\subsubsection{Most used words (band: ``Ningen isu")} \label{most-used-words-Ningen-Isu}

Question: \\
\emph{``Which words appear most often in songs by the band ``Ningen isu"? "} \\


\begin{tabular}{|l|l|}
	\hline
	& \\
	fq: & band\_str:"人間椅子" \\
	start, rows: & 0, 0 \\
	facet: & true \\
	facet.field: & lyrics\_by\_str \\
	%	facet.mincount: & 1 \\
	& \\
	\hline
\end{tabular}


\bigskip
Result: \\

"numFound":137

\begin{myLongTable}{Most used words in songs by the band ``Ningen isu"}
	人 & \emph{hito} & person & 55 \\
夢 & \emph{yume} & dream & 43 \\
心 & \emph{kokoro} & heart, mind, spirit& 32 \\
愛 & \emph{ai} & love & 32 \\
夜 & \emph{yoru} & evening, night & 31 \\
来る & \emph{kuru} & to come & 31 \\
世界 & \emph{sekai} & the world, society, the universe & 30 \\
誰 & \emph{dare} & who & 30 \\
空 & \emph{sora} & sky & 27 \\
光 & \emph{hikari} & light & 25 \\
時 & \emph{toki} & time, hour, moment & 25 \\
君 & \emph{kimi} & you, buddy, pal & 23 \\
泣く & \emph{naku} & to cry, to weep, to sob & 23 \\
見る & \emph{miru} & to see & 23 \\
闇 & \emph{yami} & darkness, the dark, despair, hopelessness & 23 \\
声 & \emph{koe} & voice & 22 \\
明日 & \emph{ashita, asu} & tomorrow, (only "asu") the near future & 22 \\
月 & \emph{tsuki} & moon & 22 \\
どこ & \emph{doko} & where, what place & 21 \\
地獄 & \emph{jigoku} & hell & 21 \\
中 & \emph{naka} & inside & 20 \\
何 & \emph{nani} &  what & 19 \\
涙 & \emph{namida} & tears & 19 \\
終わる & \emph{owaru} & to end, to come to an end, to close, to finish & 19 \\
花 & \emph{hana} & flower, blossom, bloom, petal & 19 \\
越える & \emph{koeru} & to cross over, to pass through, to go beyond& 19 \\
咲く & \emph{saku} & to bloom, to flower, to blossom & 18 \\
山 & \emph{yama} & mountain & 18 \\
果て & \emph{hote} & the end, the extremity, the limit, the limits & 18 \\
生きる & \emph{ikiru} & to live, to exist & 18 \\
知る & \emph{shiru} &  to be aware of, to know, to be conscious of & 18 \\
くる & \emph{kuru} & 来:  to come & 17 \\
この世 & \emph{konoyo} & this world, this life, world of the living & 17 \\
永遠 & \emph{eien} & eternity, perpetuity, permanence, immortality & 17 \\
目 & \emph{me} & eye & 17 \\
笑う & \emph{warau} & to laugh & 17 \\
道 & \emph{machi} & road, path, street, lane & 17 \\
吹く & \emph{fuku} & to blow (of the wind) & 16 \\
恋 & \emph{koi} & (romantic) love & 16 \\
日 & \emph{hi, nichi} & day & 16 \\
星 & \emph{hoshi} & star & 16 \\
胸 & \emph{mune} & chest, breast, heart & 16 \\
行く & \emph{iku} & to go & 16 \\
ゆく & \emph{iku} & 行く:  to go & 15 \\
待つ & \emph{matsu} & to wait & 15 \\
風 & \emph{kaze} & wind & 15 \\
くれる & \emph{kureru} & to give, to let (one) have & 14 \\
ご & & & 14 \\
すべて & \emph{subete} & 全て:  everything, all, the whole & 14 \\
め & & & 14 \\
僕 & \emph{boku} &  I, me (Pronoun, Male term) & 14 \\
宇宙 & \emph{uchyuu} & blood spilt from the body & 14 \\
歌う & \emph{utau} & to sing & 14 \\
消える & \emph{kieru} & to go out, to vanish, to disappear & 14 \\
落ちる & \emph{ochiru} & to fall down, to drop, to fall (e.g. rain) & 14 \\
血潮 & \emph{chishio} & blood spilt from the body & 14 \\
開く & \emph{hiraku} & to open, to undo, to unseal, to unpack & 14 \\
く & \emph{ku} & ward, borough, city (in Tokyo) & 13 \\
まま & \emph{mama} & as it is, as one likes & 13 \\
みる & \emph{miru} & 見る:  to see & 13 \\
命 & \emph{inochi} & life & 13 \\
影 & \emph{kage} & shadow, silhouette, figure, shape & 13 \\
持つ & \emph{motsu} & to hold (in one's hand), to take, to carry & 13 \\
時代 & \emph{jidai} & period, epoch, era, age & 13 \\
海 & \emph{umi} & sea, ocean, waters & 13 \\
神 & \emph{kami} & god, deity, divinity, spirit & 13 \\
つく & \emph{tsuku} & 着く:  to arrive at, to reach & 12 \\
今日 & \emph{kyou} & today & 12 \\
俺 & \emph{ore} & I, me (Pronoun, Male term, rough or arrogant) & 12 \\
手 & \emph{te} & hand, arm & 12 \\
死 & \emph{shi} & death & 12 \\
いい & \emph{ii} & good, excellent, fine, nice & 11 \\
ひとり & \emph{hitori} & 一人:  one person, alone, oneself; 独り: single & 11 \\
ぶる & \emph{furu} & to assume the air of \dots, to behave like \dots & 11 \\
ぼる & & & 11 \\
今 & \emph{ima} & now & 11 \\
出る & \emph{deru} & 出る:  to leave, to exit, to go out & 11 \\
地平 & \emph{shi} & horizon & 11 \\
彼方 & \emph{achira} & that way, that direction, over there & 11 \\
忘れる & \emph{wasureru} & to forget & 11 \\
悪 & \emph{waru} & wicked person, evil person, scoundrel, bad guy & 11 \\
抱く & \emph{idaku} &  to hold in one's arms (e.g. a baby), to hug & 11 \\
生まれる & \emph{umareru} & to be born & 11 \\
眠る & \emph{nemuru} & to sleep, to die, to rest (in peace), to lie (buried) & 11 \\
笑み & \emph{emi} & smile & 11 \\
色 & \emph{iro} & colour & 11 \\
見える & \emph{mieru} & to be seen, to be in sight, to appear & 11 \\
雨 & \emph{ame} & rain& 11 \\
魂 & \emph{tamashii} & soul, spirit & 11 \\
あなた & \emph{anata} & you & 10 \\
こ & & & 10 \\
ちる & & & 10 \\
やる & \emph{yaru} & to do, to undertake, to perform & 10 \\
一 & \emph{ichi} & one & 10 \\
出す & \emph{dasu} & to take out, to get out & 10 \\
前 & \emph{mae} & in front (of), before & 10 \\
形 & \emph{katachi} & form, shape, figure & 10 \\
彼 & \emph{kare} & him (Pronoun) & 10 \\
懐かしい & \emph{natsukashii} & dear (old), fondly-remembered, missed, nostalgic & 10 \\

	& & & \\
\end{myLongTable}








\bigskip
\subsubsection{Most used words (artist: ``Kaneko Ayano")}

Question: \\
\emph{``Which words appear most often in songs from singer ``Kaneko Ayano"? "} \\


\begin{tabular}{| l |  l |}
	\hline
	& \\
	fq: & band\_str:"カネコアヤノ" \\
	start, rows: & 0, 0 \\
	facet: & true \\
	facet.field: & lyrics\_by\_str \\
	%	facet.mincount: & 1 \\
	& \\
	\hline
\end{tabular}

\bigskip
Result: \\

"numFound":61
\begin{myLongTable}{Most used words in songs by ``Kaneko Ayano"}
	君 & \emph{kimi} & you, buddy, pal & 31 \\
人 & \emph{hito} & person & 24 \\
いい & \emph{ii} & good, excellent, fine, nice & 23 \\
私 & \emph{watashi} & I, me (Pronoun, slightly formal or feminine) & 23 \\
今日 & \emph{kyou} & today & 22 \\
今 & \emph{ima} & now & 18 \\
知る & \emph{shiru} &  to be aware of, to know, to be conscious of & 17 \\
中 & \emph{naka} & inside & 16 \\
あなた & \emph{anata} & you & 15 \\
くる & \emph{kuru} & 来:  to come & 15 \\
みたい & \emph{mitai} & -like, sort of, similar to, resembling & 15 \\
夜 & \emph{yoru} & evening, night & 15 \\
夢 & \emph{yume} & dream & 14 \\
街 & \emph{gai} & \dots\ street, \dots\ quarter, \dots\ district & 14 \\
みる & \emph{miru} & 見る:  to see & 13 \\
忘れる & \emph{wasureru} & to forget & 13 \\
愛 & \emph{ai} & love & 13 \\
思う & \emph{omou} & to think, to consider, to believe, to reckon& 12 \\
日 & \emph{hi, nichi} & day & 12 \\
気持ち & \emph{kimochi} & feeling, sensation, mood, state of mind & 12 \\
見る & \emph{miru} & to see & 12 \\
いつ & \emph{itsu} & when, at what time & 11 \\
いつも & \emph{itsumo} & always, all the time, at all times & 11 \\
しまう & \emph{shimau} & to finish, to stop, to put an end to & 11 \\
朝 & \emph{asa, ashita} & morning & 11 \\
行く & \emph{iku} & to go & 11 \\
誰 & \emph{dare} & who & 11 \\
いく & \emph{iku} & 行く:  to go & 10 \\
くれる & \emph{kureru} & to give, to let (one) have & 10 \\
ゆく & \emph{iku} & 行く:  to go & 10 \\
歩く & \emph{aruku} & to walk & 10 \\
話 & \emph{hanashi} & talk, speech, chat, conversation & 10 \\
そう & \emph{sou} & looking like & 9 \\
わかる & \emph{wakaru} & 分かる:  to understand, to comprehend, to grasp & 9 \\
体 & \emph{karada} & body & 9 \\
前 & \emph{mae} & in front (of), before & 9 \\
少し & \emph{sukoshi} & a little, a few & 9 \\
明日 & \emph{ashita, asu} & tomorrow, (only "asu") the near future & 9 \\
いつか & \emph{itsuka} & sometime, someday, one day & 8 \\
きっと & \emph{kitto} & surely, undoubtedly, almost certainly & 8 \\
まま & \emph{mama} & as it is, as one likes & 8 \\
出る & \emph{deru} & 出る:  to leave, to exit, to go out & 8 \\
外 & \emph{soto} & outside & 8 \\
好き & \emph{suki} & liked, well-liked, in love (with) & 8 \\
恋 & \emph{koi} & (romantic) love & 8 \\
毎日 & \emph{mainichi} & every day & 8 \\
気 & \emph{ki} & spirit, mind, heart & 8 \\
目 & \emph{me} & eye & 8 \\
笑う & \emph{warau} & to laugh & 8 \\
見える & \emph{mieru} & to be seen, to be in sight, to appear & 8 \\
部屋 & \emph{heya} & room & 8 \\
食べる & \emph{taberu} & to eat & 8 \\
ずっと & \emph{zutto} & continuously, throughout & 7 \\
だれ & \emph{dare} & 誰:  who & 7 \\
どこ & \emph{doko} & where, what place & 7 \\
ひとつ & \emph{hitotsu} & one (counter) & 7 \\
一 & \emph{ichi} & one & 7 \\
二 & \emph{ni} & two & 7 \\
何 & \emph{nani} &  what & 7 \\
僕 & \emph{boku} &  I, me (Pronoun, Male term) & 7 \\
先 & \emph{saki} & head (of a line), front & 7 \\
分かる & \emph{wakaru} & to understand, to comprehend, to grasp & 7 \\
変わる & \emph{kawaru} & to change, to be transformed, to be altered & 7 \\
日々 & \emph{hibi} & the everyday & 7 \\
星 & \emph{hoshi} & star & 7 \\
空 & \emph{sora} & sky & 7 \\
考える & \emph{kangaeru} & to think (about, of), to ponder, to contemplate & 7 \\
言う & \emph{iu} & to say & 7 \\
風 & \emph{kaze} & wind & 7 \\
ああ & \emph{aa} & ah!, oh!, alas! & 6 \\
いれる & \emph{ireru} & 入れる:  to put in & 6 \\
つく & \emph{tsuku} & 着く:  to arrive at, to reach & 6 \\
もう & \emph{mou} & already, yet, by now& 6 \\
わたし & \emph{watashi} & 私:  I, me (Pronoun, slightly formal or feminine) & 6 \\
上 & \emph{ue} & above, up, over & 6 \\
会う & \emph{au} & to meet, to encounter, to see & 6 \\
夏 & \emph{natsu} & summer & 6 \\
家 & \emph{ie, uchi} & house & 6 \\
待つ & \emph{matsu} & to wait & 6 \\
心 & \emph{kokoro} & heart, mind, spirit& 6 \\
恋しい & \emph{koishii} & yearned for, longed for, missed & 6 \\
悪い & \emph{warui} & bad, poor, undesirable & 6 \\
悲しい & \emph{kanashii} & sad, miserable, unhappy, sorrowful & 6 \\
方 & \emph{hou} & direction, way, side & 6 \\
月 & \emph{tsuki} & moon & 6 \\
気づく & \emph{kizuku} & to notice, to recognise, to become aware of & 6 \\
終わる & \emph{owaru} & to end, to come to an end, to close, to finish & 6 \\
胸 & \emph{mune} & chest, breast, heart & 6 \\
良い & \emph{yoi} & good, excellent, fine, nice, pleasant, agreeable & 6 \\
色 & \emph{iro} & colour & 6 \\
花 & \emph{hana} & flower, blossom, bloom, petal & 6 \\
言葉 & \emph{kotoba} & word, phrase& 6 \\
走る & \emph{hashiru} & to run & 6 \\
顔 & \emph{kao} & face, visage & 6 \\
きれい & \emph{kirei} & 綺麗:  pretty, lovely, beautiful, fair & 5 \\
く & \emph{ku} & ward, borough, city (in Tokyo) & 5 \\
これから & \emph{kore kara} & from now on & 5 \\
さよなら & \emph{sayonara} & goodbye, so long, farewell & 5 \\
すぎる & \emph{sugiru} & to pass through, to pass by & 5 \\

	& & & \\
\end{myLongTable}





\bigskip
\bigskip
\bigskip




\newpage
\subsection{Most used words, by songwriter} \label{most-used-words-by-songwriter}

Finding the most used words by band is interesting, as it gives an idea of the kind of themes that the band deals with in their songs. \\

What is perhaps even more relevant, however, is to find the most used words by a particular songwriter, as it provides us with some information about their writing style (in terms of lexicon), especially considering that one songwriter sometimes provides lyrics for several bands (or change bands throughout their career). \\

In this section, we will look at the style of the songwriters from ``THE BLUE HEARTS", two of whom went on to form other bands later on (namely, ``The High-Lows" and ``The Cro-Magnons"). \\


\subsubsection{Most used words (writer: Kōmoto Hiroto)}

Question: \\
\emph{``Which words appear most often in songs by 甲本ヒロト (Kōmoto Hiroto), the singer from ``THE BLUE HEARTS"? "} \\

Query:

\begin{tabular}{| l |  l |}
	\hline
	& \\
	fq: & lyrics\_by\_str:"甲本ヒロト" \\
	start, rows: & 0, 0 \\
	facet: & true \\
	facet.field: & lyrics\_by\_str \\
	facet.mincount: & 1 \\
	& \\
	\hline
\end{tabular}


\bigskip
Result: \\

"numFound":45

\begin{myLongTable}{Most used words in songs written by Kōmoto Hiroto}
	僕 & \emph{boku} &  I, me (Pronoun, Male term) & 49 \\
人 & \emph{hito} & person & 39 \\
誰 & \emph{dare} & who & 36 \\
何 & \emph{nani} &  what & 32 \\
夜 & \emph{yoru} & evening, night & 32 \\
そう & \emph{sou} & looking like & 31 \\
中 & \emph{naka} & inside & 31 \\
夢 & \emph{yume} & dream & 31 \\
行く & \emph{iku} & to go & 31 \\
見る & \emph{miru} & to see & 29 \\
もう & \emph{mou} & already, yet, by now& 27 \\
ああ & \emph{aa} & ah!, oh!, alas! & 26 \\
いい & \emph{ii} & good, excellent, fine, nice & 26 \\
まま & \emph{mama} & as it is, as one likes & 26 \\
今 & \emph{ima} & now & 26 \\
空 & \emph{sora} & sky & 26 \\
くれる & \emph{kureru} & to give, to let (one) have & 25 \\
時 & \emph{toki} & time, hour, moment & 25 \\
笑う & \emph{warau} & to laugh & 23 \\
一 & \emph{ichi} & one & 22 \\
俺 & \emph{ore} & I, me (Pronoun, Male term, rough or arrogant) & 22 \\
見える & \emph{mieru} & to be seen, to be in sight, to appear & 22 \\
やる & \emph{yaru} & to do, to undertake, to perform & 21 \\
泣く & \emph{naku} & to cry, to weep, to sob & 21 \\
君 & \emph{kimi} & you, buddy, pal & 20 \\
風 & \emph{kaze} & wind & 20 \\
どこ & \emph{doko} & where, what place & 19 \\
くる & \emph{kuru} & 来:  to come & 17 \\
世界 & \emph{sekai} & the world, society, the universe & 17 \\
事 & \emph{koto} & thing, matter & 17 \\
待つ & \emph{matsu} & to wait & 17 \\
心 & \emph{kokoro} & heart, mind, spirit& 17 \\
恋 & \emph{koi} & (romantic) love & 17 \\
街 & \emph{gai} & \dots\ street, \dots\ quarter, \dots\ district & 17 \\
いく & \emph{iku} & 行く:  to go & 16 \\
二 & \emph{ni} & two & 16 \\
今日 & \emph{kyou} & today & 16 \\
日 & \emph{hi, nichi} & day & 16 \\
死ぬ & \emph{shinu} & to die & 16 \\
目 & \emph{me} & eye & 16 \\
あなた & \emph{anata} & you & 15 \\
前 & \emph{mae} & in front (of), before & 15 \\
星 & \emph{hoshi} & star & 15 \\
遠い & \emph{toi} & far, distant, far away & 15 \\
つく & \emph{tsuku} & 着く:  to arrive at, to reach & 14 \\
今夜 & \emph{konya} & this evening, tonight & 14 \\
全部 & \emph{zenbu} & all, entire, whole, altogether & 14 \\
手 & \emph{te} & hand, arm & 14 \\
涙 & \emph{namida} & tears & 14 \\
燃える & \emph{moeru} & to burn & 14 \\
生きる & \emph{ikiru} & to live, to exist & 14 \\
みんな & \emph{minna} & everyone, everybody, all & 13 \\
ゆく & \emph{iku} & 行く:  to go & 13 \\
欲しい & \emph{hoshii} & wanted, wished for, in need of, desired & 13 \\
知る & \emph{shiru} &  to be aware of, to know, to be conscious of & 13 \\
雨 & \emph{ame} & rain& 13 \\
飛ぶ & \emph{tobu} & to fly, to soar & 13 \\
しまう & \emph{shimau} & to finish, to stop, to put an end to & 12 \\
みる & \emph{miru} & 見る:  to see & 12 \\
声 & \emph{koe} & voice & 12 \\
海 & \emph{umi} & sea, ocean, waters & 12 \\
生まれる & \emph{umareru} & to be born & 12 \\
聞く & \emph{kiku} & to hear& 12 \\
いける & \emph{ikeru} &  & 11 \\
いつ & \emph{itsu} & when, at what time & 11 \\
く & \emph{ku} & ward, borough, city (in Tokyo) & 11 \\
ずっと & \emph{zutto} & continuously, throughout & 11 \\
つける & \emph{tsukeru} &  & 11 \\
どう & \emph{dou} & how, in what way, how about & 11 \\
わかる & \emph{wakaru} & 分かる:  to understand, to comprehend, to grasp & 11 \\
上 & \emph{ue} & above, up, over & 11 \\
乗る & \emph{noru} & to get on (train, plane, bus, ship, etc.) & 11 \\
呼ぶ & \emph{yobu} &  to call out  & 11 \\
命 & \emph{inochi} & life & 11 \\
好き & \emph{suki} & liked, well-liked, in love (with) & 11 \\
忘れる & \emph{wasureru} & to forget & 11 \\
方 & \emph{hou} & direction, way, side & 11 \\
明日 & \emph{ashita, asu} & tomorrow, (only "asu") the near future & 11 \\
歩く & \emph{aruku} & to walk & 11 \\
眠る & \emph{nemuru} & to sleep, to die, to rest (in peace), to lie (buried) & 11 \\
言う & \emph{iu} & to say & 11 \\
道 & \emph{machi} & road, path, street, lane & 11 \\
降る & \emph{furu} & to fall (of rain, snow, ash, etc.), to come down & 11 \\
あげる & \emph{ageru} & 上げる:  to raise, to elevate & 10 \\
あの & \emph{ano} & that, those, the & 10 \\
いま & \emph{ima} & 今:  now & 10 \\
かける & \emph{kakeru} & & 10 \\
すぐ & \emph{sugu} & immediately, at once, right away, directly & 10 \\
ちゃう & \emph{chau} & to do completely & 10 \\
みたい & \emph{mitai} & -like, sort of, similar to, resembling & 10 \\
オレ & \emph{ore} & 俺:  I, me (Pronoun, Male term, rough or arrogant) & 10 \\
会う & \emph{au} & to meet, to encounter, to see & 10 \\
本当 & \emph{hontou} & truth, reality, actuality, fact & 10 \\
気 & \emph{ki} & spirit, mind, heart & 10 \\
線 & \emph{sen} & line, stripe & 10 \\
顔 & \emph{kao} & face, visage & 10 \\
いつか & \emph{itsuka} & sometime, someday, one day & 9 \\
そんな & \emph{sonna} & such, that sort of, that kind of & 9 \\
やさしい & \emph{yasashii} & 優しい:  tender, kind, gentle, affectionate & 9 \\

	& & & \\
\end{myLongTable}




\bigskip
\subsubsection{Most used words (writer: Mashima Masatoshi)}


Question: \\
\emph{``Which words appear most often in songs by 真島昌利 (Mashima Masatoshi), the guitarist from ``THE BLUE HEARTS"? "} \\

Query:

\begin{tabular}{| l |  l |}
	\hline
	& \\
	fq: & lyrics\_by\_str:"真島昌利" \\
	start, rows: & 0, 0 \\
	facet: & true \\
	facet.field: & lyrics\_by\_str \\
	facet.mincount: & 1 \\
	& \\
	\hline
\end{tabular}


\bigskip
Result: \\

"numFound":50 \\

\begin{myLongTable}{Most used words in songs written by Mashima Masatoshi}
	いい & \emph{ii} & good, excellent, fine, nice & 58 \\
何 & \emph{nani} &  what & 54 \\
事 & \emph{koto} & thing, matter & 44 \\
風 & \emph{kaze} & wind & 44 \\
人 & \emph{hito} & person & 40 \\
いく & \emph{iku} & 行く:  to go & 38 \\
もう & \emph{mou} & already, yet, by now& 38 \\
行く & \emph{iku} & to go & 35 \\
く & \emph{ku} & ward, borough, city (in Tokyo) & 34 \\
俺 & \emph{ore} & I, me (Pronoun, Male term, rough or arrogant) & 34 \\
一 & \emph{ichi} & one & 32 \\
今 & \emph{ima} & now & 32 \\
誰 & \emph{dare} & who & 32 \\
笑う & \emph{warau} & to laugh & 30 \\
ちゃう & \emph{chau} & to do completely & 28 \\
やる & \emph{yaru} & to do, to undertake, to perform & 28 \\
夏 & \emph{natsu} & summer & 27 \\
そう & \emph{sou} & looking like & 26 \\
今日 & \emph{kyou} & today & 26 \\
夜 & \emph{yoru} & evening, night & 26 \\
中 & \emph{naka} & inside & 25 \\
時 & \emph{toki} & time, hour, moment & 25 \\
見る & \emph{miru} & to see & 25 \\
わかる & \emph{wakaru} & 分かる:  to understand, to comprehend, to grasp & 23 \\
僕 & \emph{boku} &  I, me (Pronoun, Male term) & 23 \\
言う & \emph{iu} & to say & 23 \\
どこ & \emph{doko} & where, what place & 22 \\
君 & \emph{kimi} & you, buddy, pal & 22 \\
死ぬ & \emph{shinu} & to die & 22 \\
目 & \emph{me} & eye & 21 \\
雨 & \emph{ame} & rain& 21 \\
くる & \emph{kuru} & 来:  to come & 20 \\
明日 & \emph{ashita, asu} & tomorrow, (only "asu") the near future & 20 \\
空 & \emph{sora} & sky & 20 \\
どう & \emph{dou} & how, in what way, how about & 19 \\
上 & \emph{ue} & above, up, over & 19 \\
日 & \emph{hi, nichi} & day & 19 \\
生きる & \emph{ikiru} & to live, to exist & 19 \\
言葉 & \emph{kotoba} & word, phrase& 19 \\
吹く & \emph{fuku} & to blow (of the wind) & 18 \\
月 & \emph{tsuki} & moon & 18 \\
飛ぶ & \emph{tobu} & to fly, to soar & 18 \\
楽しい & \emph{tanoshii} & enjoyable, fun, pleasant & 17 \\
飲む & \emph{nomu} & to drink & 17 \\
いつ & \emph{itsu} & when, at what time & 16 \\
出る & \emph{deru} & 出る:  to leave, to exit, to go out & 16 \\
夢 & \emph{yume} & dream & 16 \\
手 & \emph{te} & hand, arm & 16 \\
走る & \emph{hashiru} & to run & 16 \\
くれる & \emph{kureru} & to give, to let (one) have & 15 \\
まま & \emph{mama} & as it is, as one likes & 15 \\
世界 & \emph{sekai} & the world, society, the universe & 15 \\
遠い & \emph{toi} & far, distant, far away & 15 \\
音 & \emph{oto} & sound, noise & 15 \\
すぎる & \emph{sugiru} & to pass through, to pass by & 14 \\
でる & \emph{deru} & 出る:  to leave, to exit, to go out & 14 \\
オレ & \emph{ore} & 俺:  I, me (Pronoun, Male term, rough or arrogant) & 14 \\
ギター & \emph{gitaa} & guitar & 14 \\
乗る & \emph{noru} & to get on (train, plane, bus, ship, etc.) & 14 \\
冷たい & \emph{tsumetai} & cold (to the touch) & 14 \\
歌う & \emph{utau} & to sing & 14 \\
気 & \emph{ki} & spirit, mind, heart & 14 \\
海 & \emph{umi} & sea, ocean, waters & 14 \\
見える & \emph{mieru} & to be seen, to be in sight, to appear & 14 \\
輝く & \emph{kagayaku} & to shine, to glitter, to sparkle & 14 \\
降る & \emph{furu} & to fall (of rain, snow, ash, etc.), to come down & 14 \\
食べる & \emph{taberu} & to eat & 14 \\
そんな & \emph{sonna} & such, that sort of, that kind of & 13 \\
つく & \emph{tsuku} & 着く:  to arrive at, to reach & 13 \\
つける & \emph{tsukeru} &  & 13 \\
太陽 & \emph{taiyou} & sun & 13 \\
忘れる & \emph{wasureru} & to forget & 13 \\
意味 & \emph{imi} & meaning, significance, sense & 13 \\
時間 & \emph{jikan} & time & 13 \\
自分 & \emph{jibun} & I, me, myself, yourself, oneself (Pronoun) & 13 \\
踊る & \emph{odoru} & to dance & 13 \\
達 & \emph{tachi} & (pluralizing suffix) & 13 \\
顔 & \emph{kao} & face, visage & 13 \\
あの & \emph{ano} & that, those, the & 12 \\
少し & \emph{sukoshi} & a little, a few & 12 \\
悪い & \emph{warui} & bad, poor, undesirable & 12 \\
星 & \emph{hoshi} & star & 12 \\
者 & \emph{mono, sha} & person & 12 \\
金 & \emph{kane} & money & 12 \\
いつも & \emph{itsumo} & always, all the time, at all times & 11 \\
光る & \emph{hikaru} & to shine, to glitter, to be bright & 11 \\
揺れる & \emph{yureru} & to shake, to sway, to waver & 11 \\
昨日 & \emph{kinou} & yesterday & 11 \\
来る & \emph{kuru} & to come & 11 \\
涙 & \emph{namida} & tears & 11 \\
聞く & \emph{kiku} & to hear& 11 \\
あがる & \emph{agaru} & 上がる:  to rise, to go up & 10 \\
うまい & \emph{umai} & skillful, clever; delicious & 10 \\
さん & \emph{san} & Mr., Mrs., Miss, Ms. & 10 \\
しまう & \emph{shimau} & to finish, to stop, to put an end to & 10 \\
みたい & \emph{mitai} & -like, sort of, similar to, resembling & 10 \\
みる & \emph{miru} & 見る:  to see & 10 \\
もっと & \emph{motto} & (some) more, even more, longer, further & 10 \\
バカ & \emph{baka} & ばか, 馬鹿:  idiot & 10 \\

	& & & \\
\end{myLongTable}










\bigskip
\subsubsection{Most used words (writer: Kawaguchi Junnosuke)}


Question: \\
\emph{``Which words appear most often in songs by 河口純之助 (Kawaguchi Junnosuke), the bassist from ``THE BLUE HEARTS"? "} \\

\begin{tabular}{| l |  l |}
	\hline
	& \\
	fq: & lyrics\_by\_str:"河口純之助" \\
	start, rows: & 0, 0 \\
	facet: & true \\
	facet.field: & lyrics\_by\_str \\
	facet.mincount: & 1 \\
	& \\
	\hline
\end{tabular}


\bigskip
Result: \\

"numFound":5

\begin{myLongTable}{Most used words in songs written by Kawaguchi Junnosuke}
	心 & \emph{kokoro} & heart, mind, spirit& 3 \\
胸 & \emph{mune} & chest, breast, heart & 3 \\
きっと & \emph{kitto} & surely, undoubtedly, almost certainly & 2 \\
どこ & \emph{doko} & where, what place & 2 \\
みえる & \emph{mieru} & 見える:  to be seen, to be in sight & 2 \\
ゆく & \emph{iku} & 行く:  to go & 2 \\
中 & \emph{naka} & inside & 2 \\
会える & \emph{aeru} & (be able) to meet, to encounter (会う) & 2 \\
僕 & \emph{boku} &  I, me (Pronoun, Male term) & 2 \\
前 & \emph{mae} & in front (of), before & 2 \\
君 & \emph{kimi} & you, buddy, pal & 2 \\
声 & \emph{koe} & voice & 2 \\
夢 & \emph{yume} & dream & 2 \\
奥 & \emph{oku} & inner part, interior & 2 \\
少し & \emph{sukoshi} & a little, a few & 2 \\
形 & \emph{katachi} & form, shape, figure & 2 \\
描く & \emph{egaku} & to draw & 2 \\
星 & \emph{hoshi} & star & 2 \\
気 & \emph{ki} & spirit, mind, heart & 2 \\
気持ち & \emph{kimochi} & feeling, sensation, mood, state of mind & 2 \\
流れる & \emph{nagareru} & to stream, to flow (liquid, time, etc.) & 2 \\
行く & \emph{iku} & to go & 2 \\
雨 & \emph{ame} & rain& 2 \\
雲 & \emph{kumo} & cloud & 2 \\
あける & \emph{akeru} & 開ける:  to open (a door, etc.) & 1 \\
あげる & \emph{ageru} & 上げる:  to raise, to elevate & 1 \\
あたたかい & \emph{atatakai} & 暖かい:  warm & 1 \\
あの & \emph{ano} & that, those, the & 1 \\
いま & \emph{ima} & 今:  now & 1 \\
うた & \emph{uta} &  歌:  song (歌う) & 1 \\
うねる & \emph{uneru} & 畝る:  to undulate & 1 \\
おかしな & \emph{okashina} & ridiculous, odd & 1 \\
おちる & \emph{ochiru} & to fall down & 1 \\
おなじ & \emph{onaji} & 同じ:  same, identical, equal, similar & 1 \\
かえる & \emph{kaeru} & 帰る:  to return, to go home & 1 \\
かならず & \emph{kanarazu} & 必ず:  always, without exception & 1 \\
かわる & \emph{kawaru} & 変わる:  to change, to be transformed & 1 \\
きこえる & \emph{kikoeru} & 聞こえる:  to be heard, to be audible & 1 \\
きみ & \emph{kimi} & 君:  you, buddy, pal & 1 \\
きれい & \emph{kirei} & 綺麗:  pretty, lovely, beautiful, fair & 1 \\
く & \emph{ku} & ward, borough, city (in Tokyo) & 1 \\
くる & \emph{kuru} & 来:  to come & 1 \\
ぐち & \emph{guchi} & 愚痴:   idle complaint, grumble & 1 \\
こわれる & \emph{kowareru} & 壊れる:   to be broken, to break & 1 \\
すぎる & \emph{sugiru} & to pass through, to pass by & 1 \\
ずっと & \emph{zutto} & continuously, throughout & 1 \\
そのうち & \emph{sono uchi} & before very long, soon, someday & 1 \\
そば & \emph{soba} & near, close & 1 \\
だす & \emph{dasu} & 出す:  to take out, to get out & 1 \\
ちっぽけ & \emph{chippoke} & very small, tiny & 1 \\
つくる & \emph{tsukuru} & 作る:  to make, to produce & 1 \\
つくろう & \emph{tsukuruu} & 繕う:  to mend, to patch up, to fix & 1 \\
つぶる & \emph{tsuburu} & to close (one's eyes), to shut & 1 \\
とおりぬける & \emph{toorinukeru} & 通り抜ける:  to go through & 1 \\
とける & \emph{tokeru} & 解ける:  to be solved, to be resolved & 1 \\
とびだす & \emph{tobidasu} & 飛び出す:  to jump out, to rush out & 1 \\
とれる & \emph{toreru} & 取れる:  to come off, to be removed & 1 \\
なに & \emph{nani} & 何:  what & 1 \\
なれる & \emph{nareru} & 慣れる:  to get used to & 1 \\
のぞく & \emph{nozoku} & 除く:  to remove, to eliminate & 1 \\
はいる & \emph{hairu} & 入る:  to enter, to go into & 1 \\
はじめ & \emph{hajime} & 始め:  beginning, start, first (time) & 1 \\
はず & \emph{hazu} & should (be), bound (to be) & 1 \\
びら & & & 1 \\
ふる & \emph{furu} & 降る:  to fall (e.g. rain, snow, etc.) & 1 \\
ふるえる & \emph{furueru} & 震える:  to shiver, to shake & 1 \\
ぼく & \emph{boku} &  僕:  I, me (Pronoun, Male term) & 1 \\
まぎれこむ & \emph{magirekomu} & 紛れ込む:  to disappear into, to slip into & 1 \\
まとも & \emph{matomo} & honesty, decency & 1 \\
まま & \emph{mama} & as it is, as one likes & 1 \\
まわす & \emph{mawasu} & 回す:  to turn, to rotate & 1 \\
まんなか & \emph{mannaka} & 真ん中:  middle, centre & 1 \\
みる & \emph{miru} & 見る:  to see & 1 \\
もどる & \emph{modoru} & 戻る:  to turn back & 1 \\
やわらかい & \emph{yawarakai} & 柔らかい:  soft, tender & 1 \\
ゆれる & \emph{yureru} & 揺れる:  to shake, to sway & 1 \\
よみがえる & \emph{yomigaeru} & 蘇る:   to be resurrected & 1 \\
よろこぶ & \emph{yorokobi} & to be delighted & 1 \\
アンテナ & \emph{ANTENA} & \emph{antenna} & 1 \\
インスピレーション & \emph{INSUPIREESHION} & \emph{inspiration} & 1 \\
エネルギ & \emph{ENERUGI} & \emph{energy} & 1 \\
カード & \emph{KAADO} & \emph{card} & 1 \\
カーニバル & \emph{KAANIBARU} & \emph{cannibal} & 1 \\
ガラス & \emph{GARASU} & \emph{grass} & 1 \\
キミ & \emph{kimi} & 君:  you, buddy, pal & 1 \\
キー & \emph{KII} & \emph{key} & 1 \\
サビ & \emph{SABI} & hook (high point of a song) & 1 \\
シンデレラ & \emph{SHINDERERA} & Cinderella & 1 \\
チャンネル & \emph{CHANERU} & \emph{channel} & 1 \\
テレフォン & \emph{TEREFUON} & \emph{telephone} & 1 \\
ドア & \emph{DOA} & \emph{door} & 1 \\
ドレス & \emph{DORESU} & \emph{dress} & 1 \\
ボク & \emph{boku} &  僕:  I, me (Pronoun, Male term) & 1 \\
メッセージ & \emph{MESSEEGI} & \emph{message} & 1 \\
上 & \emph{ue} & above, up, over & 1 \\
下 & \emph{shita} & below, down, under, & 1 \\
不思議 & \emph{fushigi} & wonderful, marvelous, amazing & 1 \\

	& & & \\
\end{myLongTable}


\bigskip
\bigskip



% Copyright 2022 Pierre S. Caboche. All rights reserved.

\subsection{Words usage, across bands} \label{word-usage}

So far we were trying to determine which words appear most often in Japanese songs. Now we can do it the other way round: take a word, and see how often this word is used by different bands (or songwriters). \\

We can start with the more popular words, and then try with words that appear far less often:

\begin{itemize}
	\item most used words:
	\begin{itemize}
		\item most used words in the dataset (section~\ref{most-used-words-dataset})
		
		\item most used words by band/songwriter, for each band/songwriter (section~\ref{most-used-words-by-band})
	\end{itemize}

	\item lesser used words:
	\begin{itemize}
		\item using the list of \emph{``interesting terms"} from the \MLT\ feature (section \ref{mlt}) on a particular song (e.g. a popular song, a song you like or find the lyrics interesting) can help reveal the terms which make that song unique
				
		\item through personal experience. When listening to some song's lyrics, some terms may stand out for different reasons (e.g. uncommon terms, derogatory terms, etc.). 
		
		Thanks to Solr, we can determine how often these terms are used, by which bands, in which songs\dots\ (section \ref{derogatory-terms})
	\end{itemize}
\end{itemize}


%To get the lesser used words, we would need to take some specific songs (starting with the popular ones, or your favourite songs), list which terms appear in them (using \faceting\ again, but filtering on a specific song), and find the terms that stand out (for not appearing in our lists of most used words). \\



\newpage


\subsubsection{``Dream" (夢)} \label{word-usage-dream}

Question: \\
\emph{``Which bands use the word ``dream" (夢 --- yume) the most ?"} \\


\begin{tabular}{|l|l|}
	\hline
	& \\
	q: & lyrics\_txt\_ja:"夢" \\
	q.op: & OR \\
	start, rows: & 0, 0 \\
	facet: & true \\
	facet.field: & band\_str \\
	facet.mincount: & 0 \\
	& \\
	\hline
\end{tabular}

%\bigskip
%\url{http://localhost:8983/solr/#/songs_jp/query?q=lyrics_txt_ja:%22%E5%A4%A2%22&q.op=OR&indent=true&facet=true&facet.field=band_str&facet.mincount=1&rows=0}

\bigskip
Results: \\

"numFound": 
\input{content/results/wordcounts/lex-夢-numFound.tex} \\

Facets:

\begin{longtable}{|l l|r|}
	\hline
	\multicolumn{1}{|c}{Band} & & 何曲\\
	\hline
	& & \\
	\endfirsthead
	
	\hline
	& & \\
	\endhead
	
	\hline
	\endfoot
	
	AKB48 & & 160 \\
サザンオールスターズ & \emph{Southern All Stars} & 144 \\
JAM Project & & 117 \\
徳永英明 & \emph{Tokunaga Hideaki} & 117 \\
松任谷由実 & \emph{Matsutōya Yumi} & 105 \\
SMAP & & 95 \\
B'z & & 89 \\
いきものがかり & \emph{Ikimonogakari} & 81 \\
Mr.Children & & 80 \\
fripSide & & 76 \\
石川さゆり & \emph{Ishikawa Sayuri} & 75 \\
ゆず & \emph{Yuzu} & 72 \\
BEGIN & & 66 \\
都はるみ & \emph{Miyako Harumi} & 63 \\
DA PUMP & & 58 \\
aiko & & 56 \\
GReeeeN & & 52 \\
GACKT & & 50 \\
RADWIMPS & & 49 \\
LiSA & & 45 \\
人間椅子 & \emph{Ningen Isu} & 43 \\
Aimer & & 41 \\
L'Arc~en~Ciel & & 40 \\
ゴールデンボンバー & \emph{Golden Bomber} & 38 \\
ORANGE RANGE & & 36 \\
BUMP OF CHICKEN & & 33 \\
Kiroro & & 31 \\
DREAMS COME TRUE & & 25 \\
MAN WITH A MISSION & & 25 \\
Superfly & & 24 \\
米津玄師 & \emph{Younezu Kenshi} & 24 \\
ザ・クロマニヨンズ & \emph{The Cro-Magnons} & 23 \\
MONGOL800 & & 17 \\
THE BLUE HEARTS & & 15 \\
back number & & 14 \\
カネコアヤノ & \emph{Kaneko Ayano} & 14 \\
ザ・ハイロウズ & \emph{The High-Lows} & 13 \\
河島英五 & \emph{Kawashima Eigo} & 13 \\
BABYMETAL & & 12 \\
ONE OK ROCK & & 12 \\
Xmas Eileen & & 3 \\
Ado & & 2 \\
YOASOBI & & 2 \\

	& & \\
\end{longtable}


\bigskip
\bigskip
\bigskip


As we can see, the word for ``dream" appears in 160 songs by AKB48, out of the 376 songs for this band (section \ref{query-list-of-bands}). This means that around 42.5\% of all song by AKB48 contain the word ``dream".  \\

There is no way to calculate this kind of ratio directly in Solr (or ElasticSearch). Even if there was a way to perform some sort of ``subquery", it would be prohibitively expensive. In the end it is better to perform several queries (for the numerator, and then the denominator). \\




\newpage
\wordstats{Heart}{心}{
	心 --- \emph{kokoro} \\
}


\newpage
\wordstats{Chest, Heart}{胸}{
	胸 --- \emph{mune} \\
		
	This is the word for ``chest" (the body part) but can also be used for ``heart", with a less literal meaning than 心. \\
}


\newpage
\wordstats{Love}{愛}{
	愛 --- \emph{ai} \\
}


\newpage
\wordstats{Romantic love}{恋}{
	恋 --- \emph{koi} \\
}



\newpage
\wordstats{Person}{人}{
	人 --- \emph{hito} \\
}

\newpage
\wordstats{Lover, sweetheart}{恋人}{
	恋人 --- \emph{koibito} --- lover, sweetheart, boyfriend, girlfriend \\
}


\newpage
\wordstats{(To be) liked}{好き}{
	好き --- \emph{suki} (NA-adjective) \\
}


\newpage
\wordstats{Kiss}{キス}{
	キス --- \emph{kisu} \\
}

\newpage
\wordstats{Feeling}{気持ち}{
	気持ち --- \emph{kimochi} \\
}




\newpage
\wordstats{Tears}{涙}{
	涙 --- \emph{namida, nada} \\
}

\newpage
\wordstats{Sky}{空}{
	空 --- \emph{sora} \\
}

\newpage
\wordstats{Wind}{風}{
	風 --- \emph{kaze} \\
}

\newpage
\wordstats{Rain}{雨}{
	雨 --- \emph{ame} \\
}

\newpage
\wordstats{Snow}{雪}{
	雪 --- \emph{yuki} \\
}

\newpage
\wordstats{Night}{夜}{
	夜 --- \emph{yoru} \\
}



\newpage
\wordstats{Time, hour, moment}{時}{
	時  --- \emph{toki} \\
}

\newpage
\wordstats{Moment, instant}{瞬間}{
	瞬間  --- \emph{shunkan} \\
}


\newpage
\wordstats{Always}{いつも}{
	いつも --- \emph{itsumo} \\
}


\newpage
\wordstats{Continuously}{ずっと}{
	ずっと --- \emph{zutto} --- continuously in some state \\
}



\newpage
\wordstats{Eternity}{永遠}{
	永遠  --- \emph{eien} \\
}

\newpage
\wordstats{Period, epoch, era}{時代}{
	時代  --- \emph{jidai} \\
}



\newpage
\wordstats{To laugh}{笑う}{
	笑う --- \emph{warau} \\
}

\newpage
\wordstats{To cry}{泣く}{
	泣く --- \emph{naku} \\
}

\newpage
\wordstats{Voice}{声}{
	声 --- \emph{koe} \\
}

\newpage
\wordstats{World}{世界}{
	世界 --- \emph{sekai} \\
}


\newpage
\wordstats{To end}{変わる}{
	変わる --- \emph{owaru} \\
}


\newpage
\wordstats{Future}{未来}{
	未来 --- \emph{mirai} \\
}

\newpage
\wordstats{To think}{思う}{
	思う --- \emph{omou} \\
}

\newpage
\wordstats{To remember}{思い出す}{
	思い出す --- \emph{omoidasu} \\
}

\newpage
\wordstats{A memory}{思い出}{
	思い出 --- \emph{omoide} \\
}

\newpage
\wordstats{To forget}{忘れる}{
	忘れる --- \emph{wasureru} \\
}

\newpage
\wordstats{To disappear}{消える}{
	消える --- \emph{kieru} \\
}


\newpage
\wordstats{To believe, to trust}{信じる}{
	信じる --- \emph{shinjiru} \\
}


\newpage
\wordstats{Hope, Wish, Aspiration}{希望}{
	希望 --- \emph{kibou} \\
}


\newpage
\wordstats{Wish, Desire, Hope}{望み}{
	望み --- \emph{nozomi} \\
}



\newpage
\wordstats{God (kami)}{神}{
	神 --- \emph{kami} \\
}

\newpage
\wordstats{God (kami-sama)}{神様}{
	神様 --- \emph{kami-sama} \\
}

\newpage
\wordstats{Hand}{手}{
	手 --- \emph{te} \\
}

\newpage
\wordstats{Light}{光}{
	光 --- \emph{hikari} \\
}

\newpage
\wordstats{Fire}{火}{
	火 --- \emph{hi} \\
}


\newpage
\wordstats{Flower}{花}{
	花 --- \emph{hana} \\
}

\newpage
\wordstats{Fireworks}{花火}{
	花火 --- \emph{hanabi} \\
}

\newpage
\wordstats{Flower petal}{花びら}{
	花びら --- \emph{hanabira} \\
}

\newpage
\wordstats{Cherry blossom}{桜}{
	桜 --- \emph{sakura} \\
}

\newpage
\wordstats{Mountain}{山}{
	山 --- \emph{yama} \\
}

\newpage
\wordstats{Man}{男}{
	男 --- \emph{otoko} \\
}

\newpage
\wordstats{Woman}{女}{
	女 --- \emph{onna} \\
}

\newpage
\wordstats{Child}{子}{
	子 --- \emph{ko} \\
}

\newpage
\wordstats{Cute}{可愛い, かわいい, カワイイ}{
	可愛い --- \emph{kawaii} \\
}

\newpage
\wordstats{Innocent}{初}{
	初, はつ, うい --- \emph{hatsu, ui} --- first, new
	
	うぶ, 初, 初心 --- \emph{ubu} (NA-adjective) --- innocent, naive, inexperienced \\
	
%	``Innocent" in the sense of: naive, unsophisticated, inexperienced. \\
}

\newpage
\wordstats{To be born}{生まれる}{
	生まれる --- \emph{umareru} \\
}

\newpage
\wordstats{To live}{生きる}{
	生きる --- \emph{ikiru} \\
}


\newpage
\wordstats{(Human) life}{人生}{
	人生 --- \emph{jinsei} \\
}


\newpage
\wordstats{Life, lifeforce}{命}{
	命 --- \emph{inochi} \\
}

\newpage
\wordstats{Death}{死}{
	死 --- \emph{shi} \\
}

\newpage
\wordstats{To die}{死ぬ}{
	死ぬ --- \emph{shinu} \\
}

\newpage
\wordstats{To be separated}{離れる}{
	離れる --- \emph{hanareru} \\
}

\newpage
\wordstats{Alcohol}{酒}{
	酒 --- \emph{sake} \\
}

\newpage
\wordstats{Sadness}{悲しみ}{
	悲しみ --- \emph{kanashimi} \\
}

\newpage
\wordstats{Sad, miserable}{悲しい}{
	悲しい --- \emph{kanashii} \\
}

\newpage
\wordstats{Happiness}{幸せ}{
	幸せ --- \emph{shiawase} \\
}


\newpage
\wordstats{Joy, delight}{喜び}{
	喜び --- \emph{Yorokobi} \\
}


\newpage
\wordstats{Happy}{嬉しい}{
	嬉しい --- \emph{ureshii} \\
}


\newpage
\wordstats{Enjoyable, pleasant}{楽しい}{
	楽しい --- \emph{tanoshii} \\
}


\newpage
\wordstats{To enjoy (oneself)}{楽しむ}{
	楽しむ --- \emph{tanoshimu} \\
}


\newpage
\wordstats{Promise}{約束, やくそく}{
	約束 --- \emph{yakusoku} \\
}


\newpage
\wordstats{Train station}{駅}{
	駅 --- \emph{eki} \\
}


\newpage
\wordstats{Heaven}{天国}{
	天国  --- \emph{tengoku} \\
}


\newpage
\wordstats{Hell}{地獄}{
	地獄  --- \emph{jigoku} \\
}









\newpage
\wordstats{\emph{komorebi}}{木漏れ日}{	
	\emph{``komorebi"} (木漏れ日) is a Japanese term, difficult to translate in English, which represents \emph{``the sunlight that is filtered through the trees".} \\
	
	Only a handful of artists make use of this uniquely Japanese term: \\
}





%%% Not enough songs in the dataset
% Missing song from AKB48 (野菜シスターズ)

%\newpage
%\wordstats{Pumpkin}{カボチャ, 南瓜}{}
%
%List of songs:
%
%\begin{longtable}{|l l|p{6cm}|}
%	\hline
%	\multicolumn{2}{|c|}{band} & 
%	\multicolumn{1}{|c|}{title}
%	\\
%	\hline
%	& & \\
%	\endhead
%	
%	\input{content/results/contains/contains-カボチャ, 南瓜.tex}
%	
%	& & \\
%	\hline
%\end{longtable}




\newpage
\subsubsection{Derogatory terms\dots (\ruby{軽蔑的}{けいべつてき}な\ruby{言葉}{ことば}。。。)} \label{derogatory-terms}



In this section, we will search lyrics for a few bad words, see which bands use them, and how often.


\begin{itemize}
	\item The following search term is basically the \emph{S-word} (and sometimes the \emph{F-word}) of the Japanese language
\end{itemize}


\begin{tabular}{|l|l|}
	\hline
	& \\
	q: & lyrics\_txt\_ja:"クソ", \\
	& lyrics\_txt\_ja:"くそ", \\
	& lyrics\_txt\_ja:"糞"     \\
	q.op: & OR \\
	rows: & 10 \\
	facet: & true \\
	facet.field: & band\_str \\
	facet.mincount: & 1 \\
	& \\
	\hline
\end{tabular}

\bigskip
Results: \\



Facets:
\begin{longtable}{|l l|r|}
	\hline
	& & \\
	\endhead
	
	\hline
	\endfoot
	
	Ado & & 2 \\
ゴールデンボンバー & \emph{Golden Bomber} & 2 \\
人間椅子 & \emph{Ningen Isu} & 2 \\
BEGIN & & 1 \\
DREAMS COME TRUE & & 1 \\
GReeeeN & & 1 \\
JAM Project & & 1 \\
ONE OK ROCK & & 1 \\
ORANGE RANGE & & 1 \\
back number & & 1 \\
ゆず & \emph{Yuzu} & 1 \\
サザンオールスターズ & \emph{Southern All Stars} & 1 \\

	& & \\
	
\end{longtable}


List of songs:

\begin{longtable}{|l l|p{6cm}|}
	\hline
	\multicolumn{2}{|c|}{band} & 
	\multicolumn{1}{|c|}{title}
	\\
	\hline
	& & \\
	\endhead
	
	
	Ado & & うっせぇわ \\
Ado & & レディメイド \\
back number & & サマーワンダーランド \\
BEGIN & & ソウセイ \\
DREAMS COME TRUE & & 普通の今夜のことを ー let tonight be forever remembered ー \\
GReeeeN & & 歩み \\
JAM Project & & ピアノ狂奏曲第4番~怪人賛歌 \\
ONE OK ROCK & & Re:make \\
ORANGE RANGE & & マイ・ライフル~feat.ペチュニアロクッス~ \\
ゴールデンボンバー & \emph{Golden Bomber} & SHINE \\
ゴールデンボンバー & \emph{Golden Bomber} & さよなら、さよなら、さよなら \\
サザンオールスターズ & \emph{Southern All Stars} & Let It Boogie \\
ゆず & \emph{Yuzu} & 手紙 \\
人間椅子 & \emph{Ningen Isu} & 地獄 \\
人間椅子 & \emph{Ningen Isu} & 地獄への招待状 \\

	
	& & \\
	\hline
\end{longtable}


\bigskip
\bigskip
\bigskip


\begin{itemize}
	\item The following search term is a derogatory word for \emph{``a stupid or otherwise undesirable person"}
\end{itemize}




\begin{tabular}{|l|l|}
	\hline
	& \\
	q: & lyrics\_txt\_ja:"クソッタレ", \\
	   & lyrics\_txt\_ja:"くそったれ", \\
	   & lyrics\_txt\_ja:"糞たれ"     \\
	q.op: & OR \\
	rows: & 10 \\
	facet: & true \\
	facet.field: & band\_str \\
	facet.mincount: & 1 \\
	& \\
	\hline
\end{tabular}

%\bigskip
%\url{http://localhost:8983/solr/#/songs_jp/query?q=lyrics_txt_ja:%22%E3%82%AF%E3%82%BD%E3%83%83%E3%82%BF%E3%83%AC%22,%20lyrics_txt_ja:%22%E3%81%8F%E3%81%9D%E3%81%A3%E3%81%9F%E3%82%8C%22,%20lyrics_txt_ja:%22%E7%B3%9E%E3%81%9F%E3%82%8C%22&q.op=OR&indent=true&facet=true&facet.field=band_str&facet.mincount=1}

\bigskip
Results: \\



Facets:
\begin{longtable}{|l l|r|}
	\hline
	& & \\
	\endhead
	
	\hline
	\endfoot
	
	THE BLUE HEARTS & & 3 \\
人間椅子 & \emph{Ningen Isu} & 2 \\
ゆず & \emph{Yuzu} & 1 \\
ゴールデンボンバー & \emph{Golden Bomber} & 1 \\

	& & \\
\end{longtable}


List of songs:


\begin{longtable}{|l l|l|}
	\hline
	\multicolumn{2}{|c|}{band} & 
	\multicolumn{1}{|c|}{title}
	\\
	\hline
	& & \\
	\endhead
	
	THE BLUE HEARTS & & 僕はここに立っているよ \\
THE BLUE HEARTS & & 少年の詩 \\
THE BLUE HEARTS & & 終わらない歌 \\
ゴールデンボンバー & \emph{Golden Bomber} & SHINE \\
ゆず & \emph{Yuzu} & 濃 \\
人間椅子 & \emph{Ningen Isu} & 地獄 \\
人間椅子 & \emph{Ningen Isu} & 地獄への招待状 \\

	& & \\
	\hline
\end{longtable}


%
%\begin{longtable}{|l|l|}
%	\hline
%	\multicolumn{1}{|c|}{band} & 
%	\multicolumn{1}{|c|}{title}
%	\\
%	\hline
%	& \\
%	
%	
%	THE BLUE HEARTS & 僕はここに立っているよ \emph{(I'm standing here)} \\
%	THE BLUE HEARTS & 終わらない歌 \hfill \emph{(A Never-ending Song)} \\
%	THE BLUE HEARTS & 少年の詩 \hfill \emph{(Teenage Boy Poem)} \\
%	人間椅子 (Ningen isu) & 地獄への招待状 \hfill \emph{(Invitation to Hell)} \\
%	人間椅子 (Ningen isu) & 地獄 \hfill \emph{(Hell)} \\
%	& \\
%	
%	
%	\hline
%\end{longtable}




\newpage
\subsection{Pronouns usage}

Japanese has a variety of pronouns, especially for ``I" and ``we". \\

Some pronouns are used more often by males (male term), some are used more often by females (female term), some can be used by either genders. 
Pronouns have different levels of politeness (formal, very formal, casual, or downright derogatory).
Some pronouns are used in very specific contexts (e.g. 我々, わたくしたち). \\

Studying the pronouns usage can give us an idea of the level of language used in the songs.



\wordstats{you}{君}{
	\emph{``kimi"} --- ``you", male term, familiar language \\
	
	This is the most used pronoun in the whole dataset, and usually\footnote{the ranking changes as we add new songs to the dataset} in the top 3 most used words, together with 人 (\emph{hito} -- person), and 今 (\emph{ima} -- now). \\
}


\wordstats{you}{あなた}{
	\emph{``anata"} --- ``you" \\
	
	May also mean ``dear", ``honey" (especially in \emph{enka} songs). \\
}


\wordstats{you (derogatory)}{お前, おまえ}{
	\emph{``omae"} --- ``you", male term, derogatory (used to be honorific) \\
}



\wordstats{you (derogatory)}{てめえ}{
	\emph{``temee"} --- ``you", derogatory. \\
	
	This term was added for completeness, but yields no results: \\
}

%\wordstats{you}{あなたがた, あなた方, 貴方方, 貴方々}{}



\wordstats{I}{私, わたし}{
	\emph{``watashi"} --- ``I" \\
}

\wordstats{I}{あたし}{
	\emph{``atashi"} --- ``I", female term \\
}

\wordstats{I}{僕, ぼく, ボク}{
	\emph{``boku"} --- ``I", male term \\
}

\wordstats{I}{俺,おれ,オレ}{
	\emph{``ore"} --- ``I", male term, rough or arrogant \\
}

\wordstats{I}{我, われ, ワレ}{
	\emph{``ware"} --- ``I" \\
}

\wordstats{I}{わたくし}{
	\emph{``watakushi"} --- ``I", very formal (or snob) \\
}

\wordstats{we}{私たち, 私達, わたしたち}{
	\emph{``watashitachi"} --- ``we", plural of 私 \\
}

\wordstats{we}{僕ら, ぼくら, ボクラ}{
	\emph{``bokura"} --- ``we", plural of 僕 (male term) \\
}

\wordstats{we}{俺たち, 俺達, おれたち, オレタチ}{
	\emph{``oretachi"} --- ``we", plural of 俺 (male term, rough or arrogant) \\
}

\wordstats{we}{我々, 我我, われわれ}{
	\emph{``wareware"} --- a collective ``we" to represent a group of people (as in \emph{``we the people\dots"}) \\
}

\wordstats{we}{わたくしたち}{
	\emph{``watakushitachi"} --- ``we", plural of わたくしたち (very formal or snob) \\
}








\label{analysing-lyrics-end}

